%\begin{CVbody}

\section{Publications}

\printbibliography[ heading=pubtype, type=thesis, title={\pubtitle{thesis}} ]

\printbibliography[ heading=pubtype, type=book, title= {\pubtitle{book}} ]

\printbibliography[ heading=pubtype, type=article, title={\pubtitle{ArTiClE}} ]

\printbibliography[ heading=pubtype, type=report, title={\pubtitle{report}} ]

\printbibliography[ heading=pubtype, type=software, title={\pubtitle{software}} ]

\printbibliography[ heading=pubtype, type=inproceedings, title={\pubtitle{inproceedings}} ]

\printbibliography[ heading=pubtype, keyword={video}, title={\pubtitle[Videos][\faFilm]{custom}} ]

\printbibliography[ heading=pubtype, keyword={website}, title={\pubtitle{online}} ]

\printbibliography[ heading=pubtype, type=manual, title={\pubtitle{manual}} ]

\printbibliography[ heading=pubtype, type=patent, title={\pubtitle{patent}} ]


% INFORMATION ABOUT PUBTITLE COMMAND:
%-------------------------------
% Produces a symbol with icon color and \large and bold text.
% To have a custom symbol and text, use \pubtitle[<text>][<symbol>]{custom}
% \pubtitle{custom} have default values, and you can use \pubtitle[<text>]{custom}
% to keep the default symbol and change the text or \pubtitle[<symbol>]{custom} to
% change the symbol and keep the default text (where symbol is a single token/
% braced group, i.e. a command like \faBeer). 
%
% Supported entry types:
% - custom
% - article
% - book
% - thesis
% - report
% - manual 
% - online
% - software
% - datatype
% - patent
% - conferance
% - inproceedings
% - masterthesis
% - phdthesis


%\end{CVbody}