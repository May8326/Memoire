%Document vide aux normes de l'École nationale des Chartes
%crée par J.B. Camps
%Dernières modifications E. Rouquette (03/2025)

%%%%%%%%%%%%%%%%%%%%%% PRÉAMBULE


%%%%%%%%%%%%%% partie obligatoire du préambule
\documentclass[12pt,twoside]{book}
\usepackage{fontspec}
\usepackage{xunicode}
\usepackage{polyglossia}
\setmainlanguage{french}%indiquer la langue principale du document
%\setotherlanguage{} %indiquer les autres langues utilisée


%%%%%%%%%%%%%%%%%%%%%%%%%%%%%%%%% PACKAGES UTILISÉS

\usepackage{csquotes} % les guillemets français
\usepackage{lettrine} %faire une lettrine (pas obligatoire)

\usepackage[style=biblatex-enc/enc,sorting=nyt,maxbibnames=10]{biblatex}%charger le style de l'EnC (téléchargeable ici https://ctan.org/pkg/biblatex-enc)
\addbibresource{} %le fichier bibliograhique. Exemple de chemin à partir du dossier où se trouve le document maître:Exemple ./dossierA/fichier.bib
%\defbibheading{}{\subsection*{}} Si l'on veut changer le titre de la/les bibliographie(s)


%%%Faire un ou plusieurs index

%\usepackage{imakeidx} %pour faire un ou plusieurs index
%\makeindex %commande pour générer l'index


%RAJOUTEZ ICI VOS PACKAGES




%%%%%%%%%%%%%%%%%%%%%%%%%%%%%%%%% CONFIGURATION DE MISE EN PAGE

%%%%%% Les compteurs (sections, subsections, etc)


%%%%%% Les compteurs (sections, subsections, etc)
\renewcommand{\thesection}{\Roman{section}.}%On ne fait apparaître que le numéro de la section
\renewcommand{\thesubsection}{\arabic{subsection}.}%subsection en chiffres arabes
\renewcommand{\thesubsubsection}{\alph{subsubsection}.}%subsubsection en lettres minuscules
%Si l'on veut faire apparaître les subsubsection dans le table des matières (à commenter sinon)
\setcounter{tocdepth}{3}
\setcounter{secnumdepth}{3}  % La subsubsection (profondeur=3 dans la table des matières) apparait numérotée dans la TdM



%%%%%  Configurer le document selon les normes de l'école

\usepackage[margin=2.5cm]{geometry} %marges
\usepackage{setspace} % espacement qui permet ensuite de définir un interligne
\onehalfspacing % interligne de 1.5
\setlength\parindent{1cm} % indentation des paragraphes à 1 cm


%%%%% Mise en forme des headers (haut de page)

\usepackage{fancyhdr} %package utilisé pour modifier les headers
\pagestyle{fancy} %utiliser ses propres choix de mise en page et non ceux par défaut du package

\setlength\headheight{16pt}%la hauteur des headers

%%la façon dont les sections apparaissent dans les en-tête:

\renewcommand{\sectionmark}[1]{\markright{\small\textit{\thesection~\  #1}}}%Faire apparaître dans les headers les sections en  petit et en italiques
%\renewcommand{\sectionmark}[1]{}%Commenter la lign précédetne et mettre celle-ci pour ne pas avoir le titre des sections dans le header

%% réglages propres à frontmatter

\appto\frontmatter{\pagestyle{fancy}%
	\renewcommand{\chaptermark}[1]{\markboth{\small\textit{#1}}{}}% ne pas faire apparaître de <<numéro>> de chapitre dans les chapitres non numérotés (front: l'introduction, els remerciement, etc)
}

%% réglages propres à mainmatter

\appto\mainmatter{
\renewcommand{\chaptermark}[1]{\markboth{\small\chaptername~\thechapter~--\ \textit{#1}}{}}%faire apparaître dans les headers les sections en  petit et en italiques
%\renewcommand{\chaptermark}[1]{}%Commenter la ligne précédente et mettre celle-ci pour ne pas avoir le titre des chapitres  dans le header
}

%% réglages propres aux annexes

\appto\appendix{
	\renewcommand{\chaptermark}[1]{\markboth{\small~Annexe \thechapter~--\ \textit{#1}}{}}%faire apparaître dans les headers le nom des annexes
	%\renewcommand{\chaptermark}[1]{}%Commenter la ligne précédente et mettre celle-ci pour ne pas avoir le titre des chapitres  dans le header
}




%indiquer des règles d'hyphénation pour des mots précis si besoin
%\begin{hyphenrules}{french}
%	\hyphenation{}
%\end{hyphenrules}


%%%%%%% Package hyperref

% A mettre après les autres appels de packages car redéfinit certaines commandes).

\usepackage[colorlinks=false, breaklinks=true, pdfusetitle, pdfsubject ={Mémoire HN}, pdfkeywords={les mots-clés}]{hyperref} %
\usepackage[numbered]{bookmark}%va avec hyperref; marche mieux pour les signets. l'option numbered: les signets dans le pdf sont numérotés

% Compléter pdfsubjet et pdfkeywords
%Explication des options de hyperref (modifiables)
% hyperindex=false
% colorlinks=false: pour que le cadre des liens n'apparaisse pas à l'impression
% breaklinks permet d'avoir des liens allant sur pusieurs lignes
%pdfusetitle: utiliser \author et \title pour produire le nom et le titre du pdf


%avec overleaf, utiliser :
%\usepackage[xetex]{hyperref}
%\hypersetup{
	%	pdfauthor = {Prénom Nom},
	%	pdftitle = {titre},
	%	pdfsubject = {sujet},
	%	pdfkeywords = {premier mot-clé} {deuxième mot-clé} {troisième mot-clé} {etc}
	%}



%%%%%%%%%%%%%%%%%%%% Package glossaries

%Exception: il faut le charger APRÈS hyperref
%\usepackage[toc=true]{glossaries}
%\makeglossaries
%avec TexStudio: F9 pour compiler le glossaire (s'il y a aussi un index)

%mettre les entrées du glossaire ici ou les mettre dans un fichier à part que l'on appelle ici par \loadglsentries{nom_du_fichier.tex}

%Structure d'une entrée de glossaire
%\newglossaryentry{}{%
%	name={},%
%	description={}
%}



%%%%%%%%%%%%%%%%%% DÉFINITION DES COMMANDES ET ENVIRONNMENTS







 %%%%%%%%%%%%%% INFORMATIONS POUR LA PAGE DE TITRE
 
\author{Prénom \textsc{Nom} - M2 TNAH}
\title{Titre du mémoire}

%%%%%%%%%%%%%%%%%%%%%% DOCUMENT
\begin{document}
	\begin{titlepage}
		\begin{center}
			
			\bigskip
			
			\begin{large}				
				ÉCOLE NATIONALE DES CHARTES\\
				UNIVERSITÉ PARIS, SCIENCES \& LETTRES
			\end{large}
			\begin{center}\rule{2cm}{0.02cm}\end{center}
			
			\bigskip
			\bigskip
			\bigskip
			\begin{Large}
				\textbf{Prénom Nom}\\
			\end{Large}
		%selon le cas
			\begin{normalsize} \textit{licencié.e ès lettres}\\
				\textit{diplômé.e de master}
			\end{normalsize}
			
			\bigskip
			\bigskip
			\bigskip
			
			\begin{Huge}
				\textbf{TITRE DU MÉMOIRE}\\
			\end{Huge}
			\bigskip
			\bigskip
			\begin{LARGE}
				\textbf{SOUS-TITRE DU MÉMOIRE}\\
			\end{LARGE}
			
			\bigskip
			\bigskip
			\bigskip
			\begin{large}
			\end{large}
			\vfill
			
			\begin{large}
				Mémoire 
				pour le diplôme de master \\
				\enquote{Technologies numériques appliquées à l'histoire} \\
				\bigskip
				2025
			\end{large}
			
		\end{center}
	\end{titlepage}

	\thispagestyle{empty}	
	\cleardoublepage
	
\frontmatter

	\chapter{Résumé}
\medskip
	Résumé du mémoire en français. Cette page ne doit pas dépasser une page.\\
	
	\textbf{Mots-clés:} une liste de mots-clés~; séparés par des points-virgules.
	
	\textbf{Informations bibliographiques:} Prénom Nom, \textit{Titre du mémoire. Sous-titre du mémoire}, mémoire de master \enquote{Technologies numériques appliquées à l'histoire}, dir. [Noms des directeurs.trices], École nationale des chartes, 20245.
	
		\newpage{\pagestyle{empty}\cleardoublepage}
	
	\chapter{Remerciements}
	
\lettrine{M}es remerciements vont tout d'abord à\dots
	\newpage{\pagestyle{empty}\cleardoublepage}
	
%%%%%%%%%%%% \bibliographie ici (normes de l'EnC)
%\printbibliography

	
\chapter{Introduction}	
Mon introduction 
%si l'introduction est longue, elle peut être importée avec input

\newpage{\pagestyle{empty}\cleardoublepage}

%%%%%%%%%%%%%%%%%Le corps du mémoire
	\mainmatter
%Trier par dossiers si besoin (front, main,annexes,), se crérer un docuemnt .tex par structure (section ou chapter selon la taille et la pertinence) Exemple de chemin à partir du dossier où se trouve le document maître: ./dossierA/fichier.tex

	

	\part{Une partie}
\chapter{Mon premier chapitre}
%\input{exemple_section1.tex} %Input: importer un fichier


%etc
%Ou bien: ne pas mettre chapter, et importer le chapitre complet:
%\input{mon_chapitre_1.tex}
%\input{mon_chapitre_2.tex}	
%etc
	
	\part{Une autre partie}
	
	%etc.
	
	
	\chapter*{Conclusion}
	\addcontentsline{toc}{chapter}{Conclusion}
\newpage{\pagestyle{empty}\cleardoublepage}
	






%%%%%%%%%%%%%%%%%%

\appendix %Des appendices: tables figures, etc

\chapter[Titre court]{Le titre très long de la première annexe}

%\input{fichier.tex}

\newpage{\pagestyle{empty}\cleardoublepage}

%%%%%%%%%%%%%%%%%%

\backmatter % glossaire, index, table des figures, table des matières.. (la bibliographie a déjà été appelée)

%\printindex
%\printglossaries[title=Glossaire]
%\listoftables
%\listoffigures
\tableofcontents
\end{document}
