\begin{longtable}{B C}
	\hline\hline
	\textbf{Année} & \textbf{Événement} \\
	\hline
	\endfirsthead
	
	\hline\hline
	\textbf{Année} & \textbf{Événement} \\
	\hline
	\endhead
	
	\hline
	\endfoot
	
	\hline\hline
	\endlastfoot
	
	
	1857 & Jules-François Dupuis-Delcourt propose pour la première fois l'idée d'un musée aéronautique. \\
	1863 & Gustave de Ponton d'Amécourt reprend l'idée dans La Conquête de l'Air. \\
	1879 & Échec d'une tentative de création d'un musée par l'Académie d'Aérostation météorologique. \\
	
	\rowcolor{lightgray}
	12 nov. 1918 & Albert Caquot désigne le capitaine Hirschauer pour organiser un conservatoire des matériels aéronautiques.\\
	\rowcolor{lightgray} 26 déc. 1918 & Le projet est officiellement approuvé par le ministère de la Guerre. \\
	\rowcolor{lightgray} 8 avr. 1919 & Installation des premières collections dans un hangar de Gabriel Voisin à Issy-les-Moulineaux. \\
	\rowcolor{lightgray} automne 1919 & Présentation partielle au Grand Palais lors du 6e Salon de l'aéronautique. \\
	\rowcolor{lightgray} 1920-1921 & Déménagement à Chalais-Meudon suite à une crue de la Seine. \\
	\rowcolor{lightgray} 23 nov. 1921 & Inauguration officielle du musée à Chalais-Meudon. \\
	
	sept. 1928 & Création du ministère de l'Air. \\
	1932 & Affectation de locaux au 28 boulevard Victor (Paris XV). \\
	20 nov. 1936 & Ouverture des installations parisiennes. \\
	1937 & Inauguration de l'aérogare du Bourget conçue par Georges Labro. \\
	1939 & Fermeture du musée à cause de la guerre. \\
	mars 1940 & Bombardement du bâtiment parisien. \\
	été 1940 & Saisie des collections entreposées à Amboise par l'occupant allemand. \\
	
	\rowcolor{lightgray} 7 oct. 1945 & Fermeture définitive du site du boulevard Victor. Transfert des collections à Chalais-Meudon. \\
	\rowcolor{lightgray} 1945--1961 & Musée fermé au public, accessible sur demande. \\
	\rowcolor{lightgray} 17 déc. 1951 & Fondation de l'AAMA (Association des Amis du Musée de l'Air). \\
	\rowcolor{lightgray} 1952--1972 & 21 projets d'implantation étudiés (Champ-de-Mars, Orly, Issy, Grand Palais, Versailles, etc.). \\
	\rowcolor{lightgray} 1961 & Réouverture au public à Chalais-Meudon. \\
	\rowcolor{lightgray} 1963 & Proposition d'un terrain à Orly, abandonnée pour raisons techniques. \\
	\rowcolor{lightgray} 1965--1972 & Projet du « Palais de l'Air et de l'Espace » à Issy, abandonné en 1972. \\
	
	15 fév. 1973 & Acceptation du transfert au Bourget. \\
	19 oct. 1973 & Le prototype Concorde 001 est remis au musée. \\
	27 mai 1975 & Inauguration du premier hall (Seconde Guerre mondiale). \\
	30 mai 1975 & Visite présidentielle de Valéry Giscard d'Estaing. \\
	1977 & Ouverture du hall A (1919--1939). \\
	1979 & Ouverture des halls C et D. \\
	1981 & Ouverture du hall E ; fin des vols commerciaux au Bourget ; fermeture définitive de Chalais-Meudon. \\
	1982 & Transfert de la direction et de la documentation au Bourget. \\
	
	\rowcolor{lightgray}
	1983 & Inauguration du hall de l'Espace. Le musée devient officiellement le « Musée de l'Air et de l'Espace ». \\
	\rowcolor{lightgray} 1984--1994 & Création des réserves et ateliers de restauration à Dugny. \\
	\rowcolor{lightgray} 1985 & Ouverture du Planétarium. \\
	\rowcolor{lightgray} 2 juin 1987 & Inauguration de la Grande Galerie (origines à 1918). \\
	\rowcolor{lightgray} 30 juin 1994 & Inscription de l'aérogare de 1937 aux Monuments historiques. \\
	\rowcolor{lightgray} 3 mai 1995 & Arrivée en vol du Dassault Mercure 100. Fondation de l'association IT Mercure. \\
	\rowcolor{lightgray} 1998 & Fin du transfert des collections vers Le Bourget après construction d'un atelier à Dugny. \\
	\rowcolor{lightgray} 2000 & Déploiement de Micromusée pour la gestion des collections. \\
	\rowcolor{lightgray} 2002 & Le musée obtient le label « Musée de France ». \\
	\rowcolor{lightgray} 2008 & Fondation de l'association Les Ailes de la Ville. \\
	
	
	2011 & Lancement d'une campagne de rénovation et d'extension. \\
	2013 & Rénovation et inauguration de la salle des Huit Colonnes. \\
	2016 & Déploiement de l'e-médiathèque en ligne. \\
	2017 & Inauguration de la réserve climatisée Jean-Paul Béchat à Dugny. \\
	2019-12-09 & Inauguration de la Grande Galerie rénovée. \\
	2020 & Ouverture au public de la tour de contrôle historique. \\
	2022 & Début de la construction de la réserve des aéronefs de grand format à Dugny. \\
	2023 & Ouverture de la médiathèque. \\
	2025 & Migration vers Clade/Koha pour la bibliothèque et vers Archange pour la gestion des collections. \\
	202? & Mise en service prévue de la ligne 17 du Grand Paris Express, desservant le musée. \\
	
\end{longtable}