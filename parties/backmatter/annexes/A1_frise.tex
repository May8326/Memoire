\begin{longtable}{X X X X X X X}
\toprule
\textbf{Période} & \textbf{Année} & \textbf{Contexte} & \textbf{Histoire du musée} & \textbf{Localisation du musée} & \textbf{Histoire des collections} & \textbf{Agrandissements} \\
\midrule
\endfirsthead
\toprule
\textbf{Période} & \textbf{Année} & \textbf{Contexte} & \textbf{Histoire du musée} & \textbf{Localisation du musée} & \textbf{Histoire des collections} & \textbf{Agrandissements} \\
\midrule
\endhead
\bottomrule
\endfoot
\bottomrule
\endlastfoot
\multirow{3}{*}{Avant la création officielle} & 1857 &  & Jules-François Dupuis-Delcourt propose pour la première fois l'idée d'un musée aéronautique &  &  &  \\
\addlinespace
 & 1863 &  & Gustave de Ponton d'Amécourt reprend l'idée dans La Conquête de l'Air &  &  &  \\
\addlinespace
 & 1879 &  & Échec d'une tentative de création d'un musée par l'Académie d'Aérostation météorologique &  &  &  \\
\addlinespace
\multirow{4}{*}{Création du musée} & 1918 & Fin de la Première Guerre Mondiale & Albert Caquot désigne le capitaine Hirschauer pour organiser un conservatoire des matériels aéronautiques (12 novembre), Le projet est officiellement approuvé par le ministère de la Guerre (26 décembre) &  &  &  \\
\addlinespace
 & 1919 &  &  & Installation des premières collections dans un hangar de Gabriel Voisin à Issy-les-Moulineaux (8 avril) & Présentation partielle des collections au Grand Palais lors du 6e Salon de l'aéronautique (automne) &  \\
\addlinespace
 & 1920-1921 &  &  & Déménagement à Chalais-Meudon suite à une crue de la Seine (hiver) &  &  \\
\addlinespace
 & 1921 &  & Inauguration officielle du musée à Chalais-Meudon (23 novembre) &  &  &  \\
\addlinespace
\multirow{6}{*}{Expansion à Paris et premières implantations} & 1928 & Création du ministère de l'Air (septembre) &  &  &  &  \\
\addlinespace
 & 1932 &  & Affectation de locaux au 28 boulevard Victor (Paris XV) &  &  &  \\
\addlinespace
 & 1936 &  & Ouverture des installations parisiennes (20 novembre) &  &  &  \\
\addlinespace
 & 1937 & Inauguration de l'aérogare du Bourget conçue par Georges Labro &  &  &  &  \\
\addlinespace
 & 1939 & IIème Guerre Mondiale & Fermeture du musée à cause de la guerre &  &  &  \\
\addlinespace
 & 1940 &  & Bombardement du bâtiment parisien (3 juin) &  & Saisie des collections entreposées à Amboise par l'occupant allemand (été) &  \\
\addlinespace
\multirow{7}{*}{Après-guerre et recherches de site} & 1945 &  & Fermeture définitive du site du boulevard Victor. Transfert des collections à Chalais-Meudon (10 juillet) &  & Transfert des collections à Chalais-Meudon (10 juillet) &  \\
\addlinespace
 & 1945-1961 &  & Musée fermé au public, accessible sur demande &  &  &  \\
\addlinespace
 & 1951 & Fondation de l'AAMA (Association des Amis du Musée de l'Air) (17 décembre) &  &  &  &  \\
\addlinespace
 & 1952-1972 &  &  & 21 projets d'implantation étudiés (Champ-de-Mars, Orly, Issy, Grand Palais, Versailles, etc.) &  &  \\
\addlinespace
 & 1961 &  & Réouverture au public à Chalais-Meudon &  &  &  \\
\addlinespace
 & 1963 &  &  & Proposition d'un terrain à Orly, abandonnée pour raisons techniques &  &  \\
\addlinespace
 & 1965-1972 &  &  & Projet du « Palais de l'Air et de l'Espace » à Issy, abandonné en 1972 &  &  \\
\addlinespace
\multirow{6}{*}{Implantation au Bourget} & 1973 &  &  & Acceptation du transfert au Bourget (15 février) & Le prototype Concorde 001 est remis au musée (19 octobre) &  \\
\addlinespace
 & 1975 &  & Visite présidentielle de Valéry Giscard d'Estaing (30 mai) &  &  & Inauguration du premier hall (Seconde Guerre mondiale) (27 mai) \\
\addlinespace
 & 1977 &  &  &  &  & Ouverture du hall A (1919–1939) \\
\addlinespace
 & 1979 &  &  &  &  & Ouverture des halls C et D \\
\addlinespace
 & 1981 &  & Fin des vols commerciaux au Bourget  fermeture définitive de Chalais-Meudon &  &  & Ouverture du hall E \\
\addlinespace
 & 1982 &  &  & Transfert de la direction et de la documentation au Bourget &  &  \\
\addlinespace
\multirow{10}{*}{Développement du Musée de l'Air et de l'Espace} & 1983 &  & Le musée devient officiellement le « Musée de l'Air et de l'Espace » &  &  & Inauguration du hall de l'Espace \\
\addlinespace
 & 1984-1994 &  &  &  &  & Création des réserves et ateliers de restauration à Dugny \\
\addlinespace
 & 1985 &  &  &  &  & Ouverture du Planétarium \\
\addlinespace
 & 1987 &  &  &  &  & Inauguration de la Grande Galerie (origines à 1918) (2 juin) \\
\addlinespace
 & 1994 & Inscription de l'aérogare de 1937 aux Monuments historiques (30 juin) &  &  &  &  \\
\addlinespace
 & 1995 & Fondation de l'association IT Mercure pour la conservation du Mercure 100 &  &  & Arrivée en vol du Dassault Mercure 100 (3 mai) &  \\
\addlinespace
 & 1998 &  &  & Fin du transfert des collections vers Le Bourget après construction d'un atelier à Dugny &  & Construction d'un atelier à Dugny \\
\addlinespace
 & 2000 &  & Déploiement de Micromusée pour la gestion des collections &  &  &  \\
\addlinespace
 & 2002 &  & Le musée obtient le label « Musée de France » &  &  &  \\
\addlinespace
 & 2008 & Fondation de l'association Les Ailes de la Ville &  &  &  &  \\
\addlinespace
\multirow{10}{*}{Rénovations et innovations} & 2011 &  &  &  &  & Lancement d'une campagne de rénovation et d'extension \\
\addlinespace
 & 2013 &  &  &  &  & Rénovation et inauguration de la salle des Huit Colonnes \\
\addlinespace
 & 2016 &  & Déploiement de l'e-médiathèque en ligne &  &  &  \\
\addlinespace
 & 2017 &  &  &  &  & Inauguration de la réserve climatisée Jean-Paul Béchat à Dugny \\
\addlinespace
 & 2019 &  &  &  &  & Inauguration de la Grande Galerie rénovée (9 décembre) \\
\addlinespace
 & 2020 &  &  &  &  & Ouverture au public de la tour de contrôle historique \\
\addlinespace
 & 2022 &  &  &  &  & Début de la construction de la réserve des aéronefs de grand format à Dugny \\
\addlinespace
 & 2023 &  & Ouverture de la médiathèque &  &  & Ouverture de la médiathèque \\
\addlinespace
 & 2025 &  & Migration vers Clade/Koha pour la bibliothèque et vers Archange pour la gestion des collections &  &  &  \\
\addlinespace
 & 202? & Mise en service prévue de la ligne 17 du Grand Paris Express, desservant le musée &  &  &  &  \\
\addlinespace
\end{longtable}