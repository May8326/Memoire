\begin{ltablex}{\textwidth}{Y Y Z}
\hline\hline
\textbf{Période} & \textbf{Année} & \textbf{Evénement} \\
\hline
\endfirsthead
	
\hline\hline
\textbf{Période} & \textbf{Année} & \textbf{Evénement} \\
\hline
\endhead
	
\hline
\endfoot
	
\hline\hline
\endlastfoot


\multirow{3}{*}{Prémices du musée} & 1857 & Jules-François Dupuis-Delcourt propose pour la première fois l’idée d’un musée aéronautique. \\
& 1863 & Gustave de Ponton d’Amécourt reprend l’idée dans La Conquête de l’Air. \\
& 1879 & Échec d'une tentative de création d’un musée par l’Académie d’Aérostation météorologique. \\
\multirow{6}{*}{Création du musée} & 1918-11-12 & Albert Caquot désigne le capitaine Hirschauer pour organiser un conservatoire des matériels aéronautiques.\\
& 1918-12-26 & Le projet est officiellement approuvé par le ministère de la Guerre. \\
& 1919-04-08 & Installation des premières collections dans un hangar de Gabriel Voisin à Issy-les-Moulineaux. \\
& 1919-Automne & Présentation partielle au Grand Palais lors du 6e Salon de l’aéronautique. \\
& 1920-1921 & Déménagement à Chalais-Meudon suite à une crue de la Seine. \\
& 1921-11-23 & Inauguration officielle du musée à Chalais-Meudon. \\
\multirow{7}{*}{Expansion à Paris et premières implantations} & 1928-09 & Création du ministère de l’Air. \\
& 1932 & Affectation de locaux au 28 boulevard Victor (Paris XV). \\
& 1936-11-20 & Ouverture des installations parisiennes. \\
& 1937 & Inauguration de l’aérogare du Bourget conçue par Georges Labro. \\
& 1939 & Fermeture du musée à cause de la guerre. \\
& 1940-06-03 & Bombardement du bâtiment parisien. \\
& 1940-Été & Saisie des collections entreposées à Amboise par l’occupant allemand. \\
\multirow{7}{*}{Après guerre et recherche d'un site} & 1945-07-10 & Fermeture définitive du site du boulevard Victor. Transfert des collections à Chalais-Meudon. \\
& 1945–1961 & Musée fermé au public, accessible sur demande. \\
& 1951-12-17 & Fondation de l’AAMA (Association des Amis du Musée de l’Air). \\
& 1952–1972 & 21 projets d’implantation étudiés (Champ-de-Mars, Orly, Issy, Grand Palais, Versailles, etc.). \\
& 1961 & Réouverture au public à Chalais-Meudon. \\
& 1963 & Proposition d’un terrain à Orly, abandonnée pour raisons techniques. \\
& 1965–1972 & Projet du « Palais de l’Air et de l’Espace » à Issy, abandonné en 1972. \\
\multirow{8}{*}{Implantation au Bourget} & 1973-02-15 & Acceptation du transfert au Bourget. \\
& 1973-10-19 & Le prototype Concorde 001 est remis au musée. \\
& 1975-05-27 & Inauguration du premier hall (Seconde Guerre mondiale). \\
& 1975-05-30 & Visite présidentielle de Valéry Giscard d’Estaing. \\
& 1977 & Ouverture du hall A (1919–1939). \\
& 1979 & Ouverture des halls C et D. \\
& 1981 & Ouverture du hall E ; fin des vols commerciaux au Bourget ; fermeture définitive de Chalais-Meudon. \\
& 1982 & Transfert de la direction et de la documentation au Bourget. \\
\multirow{10}{*}{Développement du \ac{mae}} & 1983 & Inauguration du hall de l’Espace. Le musée devient officiellement le « Musée de l’Air et de l’Espace ». \\
& 1984–1994 & Création des réserves et ateliers de restauration à Dugny. \\
& 1985 & Ouverture du Planétarium. \\
& 1987-06-02 & Inauguration de la Grande Galerie (origines à 1918). \\
& 1994-06-30 & Inscription de l’aérogare de 1937 aux Monuments historiques. \\
& 1995-05-03 & Arrivée en vol du Dassault Mercure 100. Fondation de l’association IT Mercure. \\
& 1998 & Fin du transfert des collections vers Le Bourget après construction d’un atelier à Dugny. \\
& 2000 & Déploiement de Micromusée pour la gestion des collections. \\
& 2002 & Le musée obtient le label « Musée de France ». \\
& 2008 & Fondation de l’association Les Ailes de la Ville. \\
\multirow{10}{*}{Rénovations et Innovations} & 2011 & Lancement d’une campagne de rénovation et d’extension. \\
& 2013 & Rénovation et inauguration de la salle des Huit Colonnes. \\
& 2016 & Déploiement de l’e-médiathèque en ligne. \\
& 2017 & Inauguration de la réserve climatisée Jean-Paul Béchat à Dugny. \\
& 2019-12-09 & Inauguration de la Grande Galerie rénovée. \\
& 2020 & Ouverture au public de la tour de contrôle historique. \\
& 2022 & Début de la construction de la réserve des aéronefs de grand format à Dugny. \\
& 2023 & Ouverture de la médiathèque. \\
& 2025 & Migration vers Clade/Koha pour la bibliothèque et vers Archange pour la gestion des collections. \\
& 202? & Mise en service prévue de la ligne 17 du Grand Paris Express, desservant le musée. \\

\end{ltablex}