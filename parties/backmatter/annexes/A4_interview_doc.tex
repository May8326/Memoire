%%%%%%%%%%%%%%%%%%%%%%%%%%%%%%%%%%%%%%%%%%%%%%%%%%%%%%%%%%%%%%%%%%%%
%%%%%%%%%%%%%%%%%%%%%%% ANNEXE - INTERVIEWS %%%%%%%%%%%%%%%%%%%%%%
%%%%%%%%%%%%%%%%%%%%%%%%%%%%%%%%%%%%%%%%%%%%%%%%%%%%%%%%%%%%%%%%%%%%
\subsection*{A FAIRE VALIDER PAR VINCENT}

\textit{Entretien réalisé le 20 mai 2025}
\indent Présents : \textit{Maëlys Gioan, Vincent Dhorne}

\subsubsection*{La bibliothèque et ses thésaurus}

\question{Le fichier que j'ai en ma possession comprend-il l'ensemble des données du thésaurus de la bibliothèque ? N'y a-t-il pas de définitions supplémentaires, de langues différentes, etc. ?}

\reponse{Oui, le fichier est complet. Il n'y a pas d'éléments supplémentaires.}

\question{Lorsque le thésaurus a été conçu, une structure particulière a-t-elle été suivie ? Comment décidiez-vous qu'un terme devait être en descripteur principal plutôt que terme générique d'un descripteur plus spécifique ?}

\reponse{[Cette question n'a pas reçu de réponse précise lors de l'entretien]}

\question{Quelles seront les applications utilisées par les agents et par les utilisateurs après la migration ?}

\reponse{Koha pour la gestion et Clade pour la diffusion.}

\question{Selon vous, qui est compétent dans le service pour le thésaurus ? Quelles sont les spécificités du thésaurus de la bibliothèque par rapport aux autres ?}

\reponse{C'est moi qui m'en occupe. Notre thésaurus est peut-être plus intéressant pour les avions récents.}

\question{D'où viennent les données des thésaurus ? Qui les saisit ? Qui les valide ?}

\reponse{Les documentalistes font des propositions de candidats en cataloguant, c'est moi qui m'occupe de les valider pour les intégrer au thésaurus.}


\subsubsection*{L'e-médiathèque}

\question{Qu'est-ce qui relève du thésaurus à proprement parler, et qu'est-ce qui relève plutôt de la normalisation de l'indexation ?}

\reponse{Ce qui constitue les mots clés (lieux, aéronefs...) constituent le thésaurus. Les personnes, événements et valeurs ne sont pas des thésaurus à proprement parler et plutôt des listes contrôlées.}

\question{Sur quoi ce thésaurus fait-il autorité ?}

\reponse{Selon [la gestionnaire de base de données], sur les constructeurs. Pour ma part, je ne vois pas d'interaction avec les autres thésaurus.}

\question{N'y aurait-il pas des données qui gagneraient à être récupérées depuis d'autres référentiels ? Par exemple pour les valeurs ou les lieux, et ne garder que les termes propres au musée à gérer ?}

\reponse{Pour les lieux, ce serait possible. Pour les autres éléments, c'est trop spécifique au musée.}

\question{D'où viennent les données des thésaurus ? Qui les saisit ? Qui les valide ?}

\reponse{[Un documentaliste] fait des ajouts en tant que candidat, puis je valide et j'ajoute. Pour les avions, je vérifie par rapport aux ouvrages de référence.}

\question{Où sont-elles hébergées ?}

\reponse{Le prestataire Cegedim stocke les données. Une autre prestation assure la maintenance de la base et du logiciel de l'e-médiathèque.}

\subsubsection*{Historique des thésaurus}

\question{Quand chaque thésaurus a-t-il été créé, approximativement ?}

\reponse{Alexandrie date de 1994. Pour Micromusée, il y a eu des réunions thésaurus vers 2000 avec les documentalistes lors de l'import des photos. Des ajouts ont été faits au fur et à mesure par les chargés de collections selon leurs besoins, jusqu'en 2016 pour la fin des photos.}

\question{Aujourd'hui, quels sont les liens entre les thésaurus ? Y a-t-il des interactions pour demander quel terme utiliser ?}

\reponse{Aucune interaction, sauf peut-être une vérification dans Alexandrie avant de créer un terme dans l'e-médiathèque. Depuis au moins le Covid où nous n'avions plus accès au serveur de Micromusée, l'habitude de vérifier les données avant de les ajouter aux thésaurus du DRD s'est perdue.}