%%%%%%%%%%%%%%%%%%%%%%%%%%%%%%%%%%%%%%%%%%%%%%%%%%%%%%%%%%%%%%%%%%%%
%%%%%%%%%%%%%%%%%%%%%%% ANNEXE - INTERVIEWS %%%%%%%%%%%%%%%%%%%%%%
%%%%%%%%%%%%%%%%%%%%%%%%%%%%%%%%%%%%%%%%%%%%%%%%%%%%%%%%%%%%%%%%%%%%
\section*{Entretiens avec les professionnels}

\subsection*{Entretien avec Vincent Dhorne, documentaliste}

\textit{Entretien réalisé le 20 mai 2025}
\indent Présents : \textit{Maëlys Gioan, Vincent Dhorne}

\subsubsection*{La bibliothèque et ses thésaurus}

\question{Le fichier que j'ai en ma possession comprend-il l'ensemble des données du thésaurus de la bibliothèque ? N'y a-t-il pas de définitions supplémentaires, de langues différentes, etc. ?}

\reponse{Oui, le fichier est complet. Il n'y a pas d'éléments supplémentaires.}

\question{Lorsque le thésaurus a été conçu, une structure particulière a-t-elle été suivie ? Comment décidiez-vous qu'un terme devait être en descripteur principal plutôt que terme générique d'un descripteur plus spécifique ?}

\reponse{[Cette question n'a pas reçu de réponse précise lors de l'entretien]}

\question{Quelles seront les applications utilisées par les agents et par les utilisateurs après la migration ?}

\reponse{Koha pour la gestion et Clade pour la diffusion.}

\question{Selon vous, qui est compétent dans le service pour le thésaurus ? Quelles sont les spécificités du thésaurus de la bibliothèque par rapport aux autres ?}

\reponse{C'est moi qui m'en occupe. Notre thésaurus est peut-être plus intéressant pour les avions récents.}

\question{D'où viennent les données des thésaurus ? Qui les saisit ? Qui les valide ? Où sont-elles hébergées ?}

\reponse{Il n'y a aucune communication au niveau du musée sur ces aspects. Il faudrait voir avec le ministère de la Défense.}

\question{Comment arrivent-elles sur Koha ? Dans quel format ? Y a-t-il des conversions ?}

\reponse{Les fichiers CSV sont transformés en MARC XML.}

\subsubsection*{L'e-médiathèque}

\question{Qu'est-ce qui relève du thésaurus à proprement parler, et qu'est-ce qui relève plutôt de la normalisation de l'indexation ?}

\reponse{Les lieux, les aéronefs (constructeurs) et les noms propres constituent le thésaurus. Les personnes, événements et valeurs ne sont pas des thésaurus à proprement parler.}

\question{Sur quoi ce thésaurus fait-il autorité ?}

\reponse{Selon Camille, sur les mots-clés. Pour ma part, je ne vois pas d'interaction avec les autres thésaurus.}

\question{N'y aurait-il pas des données qui gagneraient à être récupérées depuis d'autres référentiels ? Par exemple pour les valeurs ou les lieux, et ne garder que les termes propres au musée à gérer ?}

\reponse{Pour les lieux, ce serait possible. Pour les autres éléments, c'est trop spécifique au musée.}

\question{D'où viennent les données des thésaurus ? Qui les saisit ? Qui les valide ?}

\reponse{Claire, une prestataire externe, fait des ajouts en tant que candidat, puis je valide et j'ajoute. Pour les avions, je vérifie par rapport aux ouvrages de référence.}

\question{Où sont-elles hébergées ?}

\reponse{Le prestataire Cegedim stocke les données. Une autre prestation assure la maintenance de la base et du logiciel de l'e-médiathèque.}

\subsubsection*{Historique des thésaurus}

\question{Quand chaque thésaurus a-t-il été créé, approximativement ?}

\reponse{Alexandrie date de 1994. Pour Micromusée, il y a eu des réunions thésaurus vers 2000 avec les documentalistes lors de l'import des photos. Des ajouts ont été faits au fur et à mesure par les chargés de collections selon leurs besoins, jusqu'en 2016 pour la fin des photos.}

\question{Qui étaient les principales personnes à la tête du projet, particulièrement motivées ?}

\reponse{Les documentalistes, qui avaient le savoir-faire et qui ont alimenté Micromusée.}

\question{Aujourd'hui, quels sont les liens entre les thésaurus ? Y a-t-il des interactions pour demander quel terme utiliser ?}

\reponse{Aucune interaction, sauf peut-être une vérification dans Alexandrie avant de créer un terme dans l'e-médiathèque.}

\subsubsection*{Observations complémentaires}

Lors de cet entretien, plusieurs éléments ont été clarifiés concernant l'organisation des données :
\begin{itemize}
	\item Les \enquote{noms communs} désignent tout ce qui n'est pas un sous-ensemble d'aéronef
	\item Une vérification sur Alexandrie est parfois effectuée avant l'ajout d'un nouveau terme, surtout pour l'aviation moderne
	\item L'aviation ancienne est plutôt vérifiée sur Micromusée
	\item Pendant le confinement, l'accès à distance à Micromusée n'était pas possible
	\item Un export de Micromusée a été réalisé en 2017
\end{itemize}