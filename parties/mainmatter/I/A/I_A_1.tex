\section{\label{I-A-1}Un enjeu de représentation nationale et une autorité auprès des musées similaires}

\subsection{La lente construction du \ac{mae}}\footnote{Voir la chronologie de l'histoire du musée en Annexe  \ref{Ax-A}.}

L’histoire du Musée de l’Air et de l’Espace est celle d’un projet persistant, sans cesse reporté et modifié, qui trouve ses racines dans les aspirations d'associations ou de personnalités liées à l'aéronautique dès la fin du XIXe siècle\footcite{terrier_eroport_2019}. Aujourd'hui encore il ne cesser d'évoluer, et cette année 2025 aura vu, en plus de modernisations importantes de son environnement logiciel, l'inauguration d'un nouvel espace d'exposition permanente mettant en valeur la tour de contrôle de l'aéroport historique du Bourget\footcite{museedelairetdelespaceHallNavigationAerienne2025}. C'est en effet dans ces locaux que le musée s'est installé depuis 1973, après une longue période de recherches pour une installation définitive. Confronté aux évènements du XXe siècle, aux difficultés de conservation d'objets volumineux et techniques ou encore aux changements de tutelle, c'est principalement grâce à l'impulsion de militaires, de passionnés d'aéronautique et à sa fonction de représentation d'un savoir faire technique français qu'il a pu voir le jour.

C’est après la Première Guerre Mondiale, premier conflit armé ayant reconnu l'importance stratégique de l'aviation, que la décision devient effective. Un conservatoire de l'aéronautique est confié au capitaine Hirschauer à l’initiative d’Albert Caquot. En avril 1919, quelques aéronefs trouvent refuge dans un hangar à Issy-les-Moulineaux, avant qu’une crue de la Seine ne contraigne leur repli à Chalais-Meudon. Le 23 novembre 1921, le musée y est solennellement inauguré : l’institution est née, mais son ancrage reste fragile.

Pendant l’entre-deux-guerres, le musée s’essaie à d’autres implantations, notamment à Paris, boulevard Victor. Ces locaux sont inaugurés en 1936, mais fermés trois ans plus tard à l’aube de la Seconde Guerre Mondiale. Les bombardements puis la saisie allemande de collections en transit brisent cet élan. À la Libération, le musée réintègre Chalais-Meudon, mais il reste fermé au public pendant plus de quinze ans.

L’histoire qui suit est celle d’une errance administrative et territoriale. Vingt-et-un sites sont envisagés entre 1952 et 1972\footcite{terrier_eroport_2019} : Orly, Versailles, Issy-les-Moulineaux… Aucun ne fait l’unanimité. En 1961, le musée rouvre au public à Meudon, mais c’est une solution provisoire. Le « Palais de l’Air et de l’Espace » continue à chercher les locaux qui pourront mettre en valeur ses collections dans tout leur volume, et il faudra attendre 1973 pour que le Bourget soit retenu comme implantation définitive.

Dès son ouverture au Bourget, le musée témoigne d'un lien privilégié avec l'industrie aéronautique française et l'état : le prototype du Concorde 001 lui est donné pour son ouverture. Dès lors, les collections sont progressivement transférées depuis Chalais-Meudon qui ferme en 1981, la direction rejoint les locaux du Bourget, et de nouveaux halls sont inaugurés au fil de la mise à disposition et le rachat de nouveaux bâtiments libérés par l'aéroport. En 1983, à l'occasion de l'inauguration d'un nouveau hall consacré aux collections spatiales, le musée prend officiellement le nom de musée de l'Air et de l'Espace, qu'il conserve jusqu'à ce jour.

L’ouverture du Planétarium (1985), la création de réserves et d'un atelier de restauration sur la commune de Dugny, l’inscription de l’aérogare aux Monuments historiques (1994), l'informatisation du musée avec l’arrivée de Micromusée et du catalogue de la bibliothèque en ligne à la fin des années 1990, de l’e-médiathèque en 2016 jalonnent ce mouvement de consolidation du musée. Celui-ci devient une institution muséale à part entière, labellisée « Musée de France » depuis 2002.

Aujourd’hui, l’établissement poursuit son renouvellement avec la mise en ligne de d'outils bibliothéconomiques et muséaux, le développement de nouveaux espaces de conservation et d'exposition, ou encore son intégration prochaine au réseau du Grand Paris Express. L’histoire du Musée de l’Air et de l’Espace est celle d'un projet, qui s’est construit autour de ses collections et non à partir d’un site, entièrement dédié à la mémoire du ciel.

\subsection{Autre}
C’est donc à partir des années 1980 que le musée s’est véritablement construit et transformé, sous l’effet conjoint d’une reconnaissance de l’importance culturelle du secteur aéronautique et d’une volonté institutionnelle de l'inscrire dans le réseau national des musées. Le déménagement du musée au Bourget est révélateur de la double fonction qu’il occupe aujourd’hui : à la fois conservatoire historique d’un patrimoine technique unique et vitrine nationale d’une industrie stratégique. L’aéroport du Bourget, premier aérodrome civil français, constitue un lieu hautement symbolique, qui inscrit le musée dans la géographie de l'histoire de l'aviation française et assied sa légitimité. Par ailleurs, son lien étroit avec le \ac{siae} qu'il accueille tous les deux ans lui confère également une dimension promotionnelle, mêlant l’histoire à la modernité, et la culture au dynamisme industriel.

Cette proximité soulève néanmoins une question essentielle : dans quelle mesure le Musée de l’Air et de l’Espace, en étant si étroitement associé à une manifestation commerciale, peut-il conserver sa posture de conservateur impartial et de référence scientifique ? TODO: Renseignements sur les changements d'organigramme.
%Cette interrogation traverse les pratiques et les choix stratégiques du musée, notamment dans ses efforts pour se professionnaliser et renforcer son expertise muséale. La question de la tutelle et des modes de gestion n’est pas moins cruciale : longtemps administré par des instances militaires, le musée a dû repenser ses organigrammes, en répartissant clairement les responsabilités entre Recherche, Documentation et Conservation, regroupées aujourd’hui dans un département unique, le DSC (Département des Collections). Ce regroupement vise à favoriser les synergies, mais il pose aussi des défis, notamment en termes de gestion des archives, où les moyens restent limités.

C'est au cœur de cette organisation complexe, que se trouvent les archives du musée, traditionnellement négligées mais désormais reconnues comme des composantes fondamentales de la mémoire aéronautique. Or, la gestion de ces fonds pâtit encore d’un déficit de personnel spécialisé : depuis 2024, une seule archiviste se consacre principalement aux archives privées, tandis que les archives courantes sont en grande partie gérées via des serveurs informatiques et des missions ponctuelles. TODO: Comment pallier ce manque de ressources humaines et mettre en valeur ce patrimoine/garantir la pérennité de l'institution
%Ce constat soulève une autre question majeure : comment garantir la pérennité et la valorisation d’un patrimoine documentaire aussi riche avec des ressources humaines aussi restreintes ?

Par-delà son histoire et sa structure, ce qui distingue avant tout le Musée de l’Air et de l’Espace, c’est la richesse et la diversité de ses collections, qui en font une référence nationale sans équivalent. On y trouve, bien sûr, des avions historiques, des moteurs, des équipements techniques — objets dont la conservation requiert des conditions très spécifiques et une expertise rare. Cette particularité technique, sans précédent dans les musées français, impose une gestion adaptée et des vocabulaires spécialisés. Mais la collection ne se limite pas à ces objets spectaculaires. S’y ajoutent des maquettes, des estampes, des objets d’art, et des ensembles communs aux musées militaires, tels que uniformes. La prise en compte, plus récente, des collections civiles — vêtements d’aviateurs civils, objets du quotidien — témoigne d’une évolution de la politique muséale vers une approche plus anthropologique, qui valorise l’histoire sociale et humaine de l’aéronautique\footnote{Cette évolution se manifeste principalement dans TODO: formuler "le renommage de chargé de collections anthropo"}.

À côté de cette richesse matérielle, la documentation constitue un pilier essentiel : la base exhaustive de périodiques aéronautiques, les publications techniques, les archives photographiques et audiovisuelles illustrent la volonté du musée d’être aussi un centre de recherche et de diffusion du savoir. L’organisation interne, qui regroupe collections, documentation et recherche sous une même direction, traduit une conception intégrée du patrimoine aéronautique, mais elle fait également apparaître les différences fondamentales entre les métiers concernés — différence qui, si elle est source de richesse, génère aussi des tensions et complexifie le fonctionnement quotidien.

Enfin, cette singularité du musée reflète une réalité plus large, celle des musées techniques, qui occupent une place particulière dans le paysage muséal français. Souvent mis à l’écart au profit des musées beaux-arts, ces établissements rencontrent des difficultés spécifiques. Leurs chargés de collections doivent être formés à la fois aux savoirs techniques et aux pratiques muséales. Ces musées doivent sans cesse composer avec la nature même de leurs collections — objets souvent volumineux, complexes à conserver et à exposer — ce qui impose des méthodes innovantes et une adaptation constante.

Ainsi, le Musée de l’Air et de l’Espace s’inscrit dans cette catégorie d’institutions où l’expertise technique se mêle à l’exigence muséale, conférant à l’établissement un statut d’autorité et de référence dans son domaine. Cette position, fragile et exigeante, le place au carrefour des enjeux de représentation nationale, de conservation patrimoniale, et d’innovation culturelle.
