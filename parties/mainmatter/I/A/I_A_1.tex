\section{\label{I-A-1}Un enjeu de représentation nationale et une autorité auprès des musées similaires}

\subsection{La lente construction du \ac{mae}}\footnote{Voir la chronologie de l'histoire du musée en Annexe \ref{Ax-A}.}

L’histoire du \acf{mae} est celle d’un projet persistant, sans cesse reporté et modifié, qui trouve ses racines dans les aspirations d'associations ou de personnalités liées à l'aéronautique dès la fin du XIXe siècle\footcite{terrier_eroport_2019}. Aujourd'hui encore il ne cesser d'évoluer, et cette année 2025 aura vu, en plus de modernisations importantes de son environnement logiciel, l'inauguration d'un nouvel espace d'exposition permanente mettant en valeur la tour de contrôle de l'aéroport historique du Bourget\footcite{museedelairetdelespaceHallNavigationAerienne2025}. C'est en effet dans ces locaux que le musée s'est installé en 1973, après une longue période de recherches pour une installation définitive. Confronté aux évènements du XXe siècle, aux difficultés de conservation d'objets volumineux et techniques et aux aléas des tutelles ministérielles, c'est principalement grâce à l'impulsion de militaires, de passionnés d'aéronautique et à sa fonction de représentation d'un savoir faire technique français qu'il a pu voir le jour.

La décision devient effective après la Première Guerre Mondiale, premier conflit armé à reconnaître l'importance stratégique de l'aviation. A l'initiative d'Albert Caquot, un conservatoire de l'aéronautique est confié au capitaine Hirschauer et quelques aéronefs trouvent refuge dans un hangar à Issy-les-Moulineaux avant qu'une crue de la Seine ne contraigne leur repli à Chalais-Meudon. Le musée est officiellement inauguré le 23 novembre 1921 : l’institution est née, mais son ancrage reste fragile.
Pendant l’entre-deux-guerres, le musée s’essaie à d’autres implantations, notamment à Paris, boulevard Victor. Ces locaux sont inaugurés en 1936, mais fermés trois ans plus tard à l’aube de la Seconde Guerre Mondiale. Les bombardements puis la saisie allemande de collections en transit brisent son élan : à la Libération, le musée réintègre Chalais-Meudon, mais il reste fermé au public pendant plus de quinze ans.

L’histoire qui suit est celle d’une errance administrative et territoriale. Vingt-et-un sites sont envisagés entre 1952 et 1972\footcite{terrier_eroport_2019} : Orly, Versailles, Issy-les-Moulineaux… Aucun ne fait l’unanimité. En 1961, le musée rouvre au public à Meudon, mais c’est une solution provisoire. Le « Palais de l’Air et de l’Espace » continue à chercher les locaux qui pourront mettre en valeur ses collections dans tout leur volume, et il faudra attendre 1973 pour que l'ancien aéroport du Bourget, dont l'activité vient d'être drastiquement réduite après l’essor d'Orly, soit retenu comme implantation définitive.

Dès son ouverture au Bourget, le musée témoigne d'un lien privilégié avec l'industrie aéronautique française et l'état : pour son ouverture, le prototype du Concorde 001 lui est donné. Dès lors, les collections sont progressivement transférées depuis Chalais-Meudon qui ferme en 1981, la direction rejoint les locaux du Bourget, et de nouveaux halls sont inaugurés au fil de la mise à disposition et le rachat de nouveaux bâtiments libérés par l'aéroport. En 1983, à l'occasion de l'inauguration d'un nouveau hall consacré aux collections spatiales, le musée prend officiellement son nom actuel de \acf{mae}.

Ce mouvement de consolidation du musée et d'intégration dans le réseau des musées techniques français se consolide à cette période, notamment avec l'ouverture du Planétarium (1985), la création de réserves et d'un atelier de restauration sur la commune de Dugny,  ou l'informatisation du musée. Dès la fin des années 1990 sont en effet mis en place Micromusée, logiciel de gestion des collections et le \ac{sigb} Alexandrie. Cette mise en place est suivie à partir de 2016 par la mise en ligne d'un logiciel exclusivement dédié à la gestion et la diffusion de fichiers audiovisuels : l'e-médiathèque du musée. Celui-ci devient une institution muséale à part entière qui se professionnalise petit à petit, le \ac{mae} est labellisé « Musée de France » depuis 2002.

Aujourd’hui, l’établissement poursuit son développement en renouvelant ses outils bibliothéconomiques et muséaux, en inaugurant de nouveaux espaces de conservation et d'exposition, et son intégration prochaine au réseau du Grand Paris Express laisse espérer un accroissement de son attractivité auprès du public. L’histoire du \ac{mae} est ainsi celle d'un projet, qui s’est construit autour de ses collections et non à partir d’un site, entièrement dédié à la mémoire du ciel.

\subsection{Une institution complexe qui fait référence}
C’est donc à partir des années 1980 que le musée s’est véritablement construit et transformé, sous l’effet conjoint d’une reconnaissance de l’importance culturelle du secteur aéronautique, du renouvellement de la réflexion sur la muséologie et d’une volonté d'inscrire l'institution dans le réseau national des musées. Le déménagement du musée au Bourget est révélateur de la double fonction qu’il occupe aujourd’hui : à la fois conservatoire historique d’un patrimoine technique, et vitrine nationale d’une industrie stratégique. L’aéroport du Bourget, premier aérodrome civil français est en effet un lieu hautement symbolique qui inscrit le musée dans la géographie comme dans l'histoire de l'aviation française. Cette localisation unique confère encore davantage de légitimité au musée. Son lien étroit avec le \ac{siae} qu'il accueille tous les deux ans assied également sa dimension promotionnelle, qui mêle l’histoire à la modernité, et la culture au dynamisme industriel.



C'est au cœur de cette organisation complexe, que se trouvent les archives du musée, traditionnellement négligées mais désormais reconnues comme des composantes fondamentales de la mémoire aéronautique. Or, la gestion de ces fonds pâtit encore d’un déficit de personnel spécialisé : depuis 2024, une seule archiviste se consacre principalement aux archives privées, tandis que les archives courantes sont en grande partie gérées via des serveurs informatiques et des missions ponctuelles. TODO: Comment pallier ce manque de ressources humaines et mettre en valeur ce patrimoine/garantir la pérennité de l'institution
%Ce constat soulève une autre question majeure : comment garantir la pérennité et la valorisation d’un patrimoine documentaire aussi riche avec des ressources humaines aussi restreintes ?

Par-delà son histoire et sa structure, ce qui distingue avant tout le Musée de l’Air et de l’Espace, c’est la richesse et la diversité de ses collections, qui en font une référence nationale sans équivalent. On y trouve, bien sûr, des avions historiques, des moteurs, des équipements techniques — objets dont la conservation requiert des conditions très spécifiques et une expertise rare. Cette particularité technique, sans précédent dans les musées français, impose une gestion adaptée et des vocabulaires spécialisés. Mais la collection ne se limite pas à ces objets spectaculaires. S’y ajoutent des maquettes, des estampes, des objets d’art, et des ensembles communs aux musées militaires, tels que uniformes. La prise en compte, plus récente, des collections civiles — vêtements d’aviateurs civils, objets du quotidien — témoigne d’une évolution de la politique muséale vers une approche plus anthropologique, qui valorise l’histoire sociale et humaine de l’aéronautique\footnote{Cette évolution se manifeste principalement dans TODO: formuler "le renommage de chargé de collections anthropo"}.

À côté de cette richesse matérielle, la documentation constitue un pilier essentiel : la base exhaustive de périodiques aéronautiques, les publications techniques, les archives photographiques et audiovisuelles illustrent la volonté du musée d’être aussi un centre de recherche et de diffusion du savoir. L’organisation interne, qui regroupe collections, documentation et recherche sous une même direction, traduit une conception intégrée du patrimoine aéronautique, mais elle fait également apparaître les différences fondamentales entre les métiers concernés — différence qui, si elle est source de richesse, génère aussi des tensions et complexifie le fonctionnement quotidien.

Enfin, cette singularité du musée reflète une réalité plus large, celle des musées techniques, qui occupent une place particulière dans le paysage muséal français. Souvent mis à l’écart au profit des musées beaux-arts, ces établissements rencontrent des difficultés spécifiques. Leurs chargés de collections doivent être formés à la fois aux savoirs techniques et aux pratiques muséales. Ces musées doivent sans cesse composer avec la nature même de leurs collections — objets souvent volumineux, complexes à conserver et à exposer — ce qui impose des méthodes innovantes et une adaptation constante.

Ainsi, le Musée de l’Air et de l’Espace s’inscrit dans cette catégorie d’institutions où l’expertise technique se mêle à l’exigence muséale, conférant à l’établissement un statut d’autorité et de référence dans son domaine. Cette position, fragile et exigeante, le place au carrefour des enjeux de représentation nationale, de conservation patrimoniale, et d’innovation culturelle.
