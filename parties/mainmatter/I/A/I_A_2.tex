\section{\label{I-A-2}La recherche : le rôle déterminant d'un musée technique}

\begin{quote}
	\og Le \acf{mae} doit développer ses réseaux \textelp{} dans les domaines civils et militaires. Pour ce dernier domaine, ils répondent à son statut de musée du ministère des Armées et par extension de sa sensibilisation de la société au monde militaire et à son histoire.\\
	Pour la sphère civile, cette influence est essentiellement réalisée autour des industriels afin de les sensibiliser au patrimoine de la troisième dimension, leur patrimoine ; de l’Enseignement Supérieur et des jeunes pour développer des synergies avec les formations en lien avec le milieu culturel et proposer un accès aux cursus disponibles dans les domaines de l’aéronautique et du spatial. Enfin \textelp{} le musée doit conserver son rôle d’institution de référence dans le domaine de la conservation du patrimoine de la troisième dimension auprès des associations aéronautiques et spatiales\footnote{\Ac{psc} 2020 du \mae}.\fg
\end{quote}


Le rôle du \ac{mae} ne se cantonne pas seulement à la seule conservation d’objets : celui-ci s'impose en effet comme un acteur essentiel de la recherche au carrefour entre histoire technique, aéronautique, sciences sociales et muséologie.  Le \ac{psc} du \acf{mae}, remanié en 2020, est un précieux témoin du difficile équilibre recherché par le musée pour assurer la visibilité et valorisation de ses fonds auprès du grand public comme de la communauté scientifique.

\subsection{Un acteur central dans les réseaux de recherche aéronautique}

Le \ac{mae} se retrouve en effet, comme bien des musées techniques ou beaux-arts, pris entre différents mondes : musées, bibliothèques, archives, centres de recherches, associations de passionnés... Une grande partie de sa mission consiste donc à assurer la communication entre ces différents qui échangent savoir, pratiques et innovations dans un réseau national comme international.

Ce rôle se manifeste dans la multiplication des expositions temporaires à dimension internationale : l’exposition \emph{Flight}, fruit d’un partenariat avec le Parque de las Ciencias de Grenade, le centre Techmania de Plzeň et l’Institut royal des Sciences naturelles de Belgique, illustre ainsi la capacité du musée à fédérer des acteurs divers autour d’une réflexion sur le vol humain et animal\footnote{Voir \href{https://www.museeairespace.fr/agenda/exposition-flight}{https://www.museeairespace.fr/agenda/exposition-flight}}. De même, des journées d’études, comme celle organisée en 2019 pour le centenaire de l’aviation civile\footnote{Programme disponible sur le site du musée \cite{19192019CentAns}}, réunissent des universitaires, des conservateurs, des ingénieurs ou des amateurs, se retrouvant pour croiser les regards et les méthodes pour améliorer notre compréhension de l'histoire ou encore de la sociologie de l'aéronautique.

L’engagement du musée ne s’arrête pas à la diffusion : il participe à des projets de recherche interdisciplinaires comme le programme C-ADER déposé auprès de l'\ac{anr}, et qui fédère le \ac{c2rmf}, l’Institut de recherche de chimie Paris, l’université de Lorraine et l’Institut de soudure, autour de la  question de la conservation des aéronefs exposés en extérieur\footnote{Voir  \href{https://anr.fr/Projet-ANR-22-CE27-0025}{https://anr.fr/Projet-ANR-22-CE27-0025}}. Ce projet transversal conduira entre autres à l’élaboration d’un thésaurus partagé : celui-ci est une nécessité pragmatique, mais aussi un acte intellectuel qui permet de faire dialoguer chimistes, restaurateurs, conservateurs et historiens avec une même langue. Le musée exerce pleinement dans ce programme son rôle de médiateur et de catalyseur de la recherche.

\subsection{Un réseau diversifié : répondre aux exigences de tous}

Le \ac{psc} du \ac{mae} dresse la liste des multiples acteurs avec lesquels l'établissement collabore au quotidien. Cette diversité — chercheurs, passionnés d'aéronautique, industriels, grand public — impose au musée de jongler entre des vocabulaires spécialisés et des attentes très différentes. Comment transmettre les mêmes connaissances techniques à un ingénieur Dassault et à un visiteur occasionnel ? La question se pose d'autant plus que le musée fait face, comme beaucoup d'institutions patrimoniales, à un paysage informationnel éclaté où la spécialisation des vocabulaires et la multiplication des bases de données compliquent la transmission du savoir.

\begin{figure}[htbp]
	\begin{adjustbox}{width=\textwidth,center}
	\begin{tikzpicture}[
		% Style 
		central/.style={
			text width=3cm, align=center,
			font=\bfseries
		},
		category/.style={
			text width=2.5cm, align=center,
			font=\small\bfseries
		},
		subcategory/.style={
			text width=2.2cm, align=center,
			font=\footnotesize
		},
		% Style pour les connexions
		connector/.style={-, thin, black},
		% Distance standards
		node distance=3cm and 2cm
		]
		
		% Nœud central
		\node[central] (centre) {\acf{mae}};
		
		% Catégories principales
		\node[category, above=5cm of centre] (armee) {Armée et\\Défense};
		\node[category, above right=3.5cm and 4.5cm of centre] (industrie) {Industrie};
		\node[category, below=5cm of centre] (culture) {Culture et\\Patrimoine};
		\node[category, below right=3.5cm and 4.5cm of centre] (medias) {Médias et\\Formation};
		\node[category, below left=3.5cm and 4.5cm of centre] (recherche) {Enseignement\\et Recherche};
		\node[category, above left=3.5cm and 4.5cm of centre] (associations) {Associations\\aéronautiques};
		
		% Sous-catégories pour Armée et Défense (5 nœuds)
		\node[subcategory, above left=3cm and 3cm of armee] (armee1) {Armée de l'Air};
		\node[subcategory, above=3cm of armee] (armee2) {Ministère des\\Armées};
		\node[subcategory, above right=3cm and 3cm of armee] (armee4) {École de l'air};
		\node[subcategory, left=3cm of armee] (armee5) {SIRPA Air,\\CERPA};
		\node[subcategory, right=3cm of armee] (armee6) {Délégation militaire\\départementale (93)};
		
		% Sous-catégories pour Industrie (5 nœuds)
		\node[subcategory, above=3cm of industrie] (ind1) {GIFAS};
		\node[subcategory, above right=3cm and 3cm of industrie] (ind2) {Sociétés\\aéronautiques};
		\node[subcategory, right=4cm of industrie] (ind3) {Airbus Group};
		\node[subcategory, below right=3cm and 3cm of industrie] (ind4) {Safran};
		\node[subcategory, below=3cm of industrie] (ind5) {Dassault Aviation};
		
		% Sous-catégories pour Médias et Formation (2 nœuds)
		\node[subcategory, above right=3cm and 3cm of medias] (med1) {Magazine\\Aviation \& Pilote};
		\node[subcategory, below right=3cm and 3cm of medias] (med2) {Pôle de compétitivité\\Astech Paris Région};
		
		% Sous-catégories pour Culture et Patrimoine (5 nœuds)
		\node[subcategory, below left=3cm and 3cm of culture] (cult1) {Musée national\\de la Marine};
		\node[subcategory, below left=3cm and 1cm of culture] (cult2) {Musées de\\l'armée};
		\node[subcategory, below right=3cm and 1cm of culture] (cult3) {Archives\\Nationales};
		\node[subcategory, below right=3cm and 3cm of culture] (cult4) {\ac{bnf}};
		\node[subcategory, below=3cm of culture] (cult5) {Médiathèques locales};
		
		% Sous-catégories pour Enseignement et Recherche (5 nœuds)
		\node[subcategory, above=3cm of recherche] (rech5) {Campus\\Condorcet};
		\node[subcategory, above left=3cm and 3cm of recherche] (rech4) {INP};
		\node[subcategory, left=4cm of recherche] (rech3) {Académie\\de Créteil};
		\node[subcategory, below left=3cm and 3cm of recherche] (rech2) {Éducation\\nationale};
		\node[subcategory, below=3cm of recherche] (rech1) {IHEDN – Trinôme\\académique IdF};
		
		% Sous-catégories pour Associations aéronautiques (5 nœuds)
		\node[subcategory, above=3cm of associations] (assoc1) {AAMA – Amis du\\Musée de l'Air};
		\node[subcategory, above left=3cm and 3cm of associations] (assoc2) {Cerfs-Volants\\Historiques};
		\node[subcategory, left=4cm of associations] (assoc3) {Aéromodélisme\\Club de Saint-Denis};
		\node[subcategory, below left=3cm and 3cm of associations] (assoc4) {Mémorial Flight};
		\node[subcategory, below=3cm of associations] (assoc5) {Ailes anciennes};
		
		% Connexions principales (centre vers catégories)
		\draw[connector] (centre) -- (armee);
		\draw[connector] (centre) -- (industrie);
		\draw[connector] (centre) -- (medias);
		\draw[connector] (centre) -- (culture);
		\draw[connector] (centre) -- (recherche);
		\draw[connector] (centre) -- (associations);
		
		% Connexions pour Armée et Défense
		\draw[connector] (armee) -- (armee1);
		\draw[connector] (armee) -- (armee2);
		\draw[connector] (armee) -- (armee4);
		\draw[connector] (armee) -- (armee5);
		\draw[connector] (armee) -- (armee6);
		
		% Connexions pour Industrie
		\draw[connector] (industrie) -- (ind1);
		\draw[connector] (industrie) -- (ind2);
		\draw[connector] (industrie) -- (ind3);
		\draw[connector] (industrie) -- (ind4);
		\draw[connector] (industrie) -- (ind5);
		
		% Connexions pour Médias et Formation
		\draw[connector] (medias) -- (med1);
		\draw[connector] (medias) -- (med2);
		
		% Connexions pour Culture et Patrimoine
		\draw[connector] (culture) -- (cult1);
		\draw[connector] (culture) -- (cult2);
		\draw[connector] (culture) -- (cult3);
		\draw[connector] (culture) -- (cult4);
		\draw[connector] (culture) -- (cult5);
		
		% Connexions pour Enseignement et Recherche
		\draw[connector] (recherche) -- (rech1);
		\draw[connector] (recherche) -- (rech2);
		\draw[connector] (recherche) -- (rech3);
		\draw[connector] (recherche) -- (rech4);
		\draw[connector] (recherche) -- (rech5);
		
		% Connexions pour Associations aéronautiques
		\draw[connector] (associations) -- (assoc1);
		\draw[connector] (associations) -- (assoc2);
		\draw[connector] (associations) -- (assoc3);
		\draw[connector] (associations) -- (assoc4);
		\draw[connector] (associations) -- (assoc5);
		
	\end{tikzpicture}
\end{adjustbox}
	\caption{Diversité des partenaires du \ac{mae} (liste non exhaustive établie à partir du \ac{psc} 2020).}
	\label{fig:schem_partenaires}
\end{figure}

Pour mieux répondre à ces exigences, le musée a restructuré son organigramme\footnote{Voir Annexe \ref{Ax-B}} à la fin des années 2010, répartissant ses missions entre trois départements : collections, publics, fonctions support. Cette nouvelle organisation, selon le \ac{psc} 2020, \enquote{amorce le développement de rapports plus transversaux entre les équipes du musée, qui ont ainsi gagné en efficacité}. Le regroupement de la conservation, de la recherche et de la documentation au sein du \ac{dsc} intègre au même département toute la gestion du patrimoine aéronautique, que ce soit pour la gestion des collections, de la documentation, de la recherche  ou des archives du musée. Le \ac{drd}, avec une base exhaustive de périodiques, de publications techniques, d'archives photographiques et audiovisuelles fait ainsi du \mae un centre unique de recherche et de diffusion du savoir aéronautique.

Cette réorganisation présente néanmoins ses écueils. La séparation entre fonctions patrimoniales et médiation culturelle génère des difficultés de communication, notamment lors de la préparation d'expositions où les deux départements doivent collaborer étroitement. Au sein du \ac{dsc}, la documentation cumule des missions très diverses : gestion de la bibliothèque, accompagnement des chercheurs, traitement des images produites par la régie, conservation des archives. Cette accumulation de responsabilités, qui reflète la richesse du fonds documentaire, soulève aussi des questions sur la reconnaissance de métiers aux exigences techniques très spécifiques. L'organisation actuelle révèle ainsi les tensions entre l'efficacité recherchée par le regroupement et les spécificités de cultures professionnelles distinctes.