\section{\label{I-A-2}La recherche : le rôle déterminant d'un musée technique}


%Intérêt et nécessité pour la visibilité du musée et son rôle dans la recherche d'adopter une stratégie 1) pour l'interopérabilité de ses vocabulaires contrôlés 2) pour une gestion efficace de ses archives numériques liées aux œuvres 
%
%Avant tout, l’aéronautique, puis tout ce qui s’y rapporte : entre archives, musées, bibliothèques, centres de recherche, activités de passionnés d’aéronautique. 
%
%Un réseau national et international de musées et d’associations liées à l’aéronautique. 
%
%La place des musées dans la recherche : aide à la recherche, organisation des expositions, journées d’études, ... 


Le rôle du \ac{mae} ne se cantonne pas seulement à la seule conservation d’objets : celui-ci s'impose en effet comme un acteur essentiel de la recherche au carrefour entre histoire technique, aéronautique, sciences sociales et muséologie.  Le \ac{psc} du \acf{mae}, remanié en 2020, est un précieux témoin du difficile équilibre recherché par le musée pour assurer la visibilité et valorisation de ses fonds auprès du grand public comme de la communauté scientifique.

\subsection{Un acteur central dans les réseaux de recherche aéronautique}

Le \ac{mae} se retrouve en effet, comme bien des musées techniques ou beaux-arts, pris entre différents mondes : musées, bibliothèques, archives, centres de recherches, associations de passionnés... Une grande partie de sa mission consiste donc à assurer la communication entre ces différents qui échangent savoir, pratiques et innovations dans un réseau national comme international.

Ce réseau est mobilisé notamment lors d'expositions temporaires : c'est par exemple le cas de l'exposition \emph{Flight} réalisée en partenariat avec le Parque de las Ciencias de Grenade, le centre Techmania de Plzen (République Tchèque) et l’Institut royal de Sciences naturelles
de Belgique à Bruxelles, sur les mécanismes du vol humain mis en relations avec ceux du vol animal\footnote{cf. \href{https://www.museeairespace.fr/agenda/exposition-flight}{https://www.museeairespace.fr/agenda/exposition-flight}}. Des journées d'études, bien qu'ouvertes à tous, sont adressées à un public plus averti : en 2019, musée proposait ainsi une journée d'étude pour commémorer le centenaire de l'aviation civile et commerciale en France\footcite{19192019CentAns}. Les intervenants regroupaient divers profils comme chercheur d'université, historien pour une association, conservateurs...

C'est surtout dans des projets de recherche comme C-ADER [TODO: ajout docu CADER et partenariats + production d'instruments de recherche et documentation particulière, savoirs faire très différents] = production d'instrument de recherche transversal par le musée, pour divers acteurs.
%La spécificité du \ac{mae} réside également dans sa position de carrefour entre différents mondes : musées, bibliothèques, centres de recherche, associations de passionnés. Sa mission ne se limite pas à la préservation, elle s’étend à l’animation d’un vaste réseau national et international, où s’échangent savoirs, pratiques, et innovations\footcite{collectifMuseeLairLespace2023,yarrowBibliothequesPubliquesArchives2008a}. 

\subsection{Les outils de la recherche au \ac{mae}}

Dans un paysage patrimonial où la profusion de bases de données et de systèmes d’information tend à fragmenter le savoir, l’interopérabilité des vocabulaires contrôlés s’impose comme une exigence majeure. Pour le \ac{mae}, l’enjeu est double : il s’agit d’une part de permettre aux chercheurs d’interroger, de croiser et de réutiliser les données du musée dans des contextes variés ; d’autre part, de garantir que le vocabulaire propre à l’aéronautique, d’une technicité souvent extrême, ne constitue pas un obstacle mais un vecteur d’accès à l’information\footcite{hudonISO25964Pour2012a,chichereauNormesConceptionGestion2007}. 

L’adoption de normes partagées – à l’instar du \ac{skos} ou de l’ISO 25964 – ainsi que la réflexion sur les passerelles entre les thésaurus internes et les référentiels nationaux (RAMEAU, IdRef) participent de cette dynamique. Ce travail, encore largement en cours, conditionne l’intégration du musée dans les grands réseaux de la recherche et favorise sa participation à des projets collectifs, tant nationaux qu’internationaux\footcite{nouvelThesaurusPACTOLSSysteme2019}.


L’autre versant de cette stratégie documentaire concerne la gestion des archives numériques liées aux œuvres. La numérisation massive des collections, des dossiers d’œuvre et des ressources iconographiques, si elle constitue un progrès indéniable pour la conservation et la diffusion, fait peser sur l’institution une responsabilité nouvelle : garantir la traçabilité, l’intégrité et la pérennité de ces données\footcite{ministeredelacultureDocumenterArchiverMusee2020,bechardArchivesElectroniques2020a}. 

La mise en place de procédures d’archivage conformes aux référentiels nationaux\footcite{comiteinterministerielauxarchivesdefranceReferentielGeneralGestion}, l’adoption de solutions logicielles adaptées (GED, SAE), et la sensibilisation des personnels à ces problématiques sont autant de conditions pour que le musée ne se contente pas d’accumuler des fichiers, mais inscrive son action dans une politique raisonnée de gestion de l’information. Cette maîtrise documentaire, loin d’être un luxe, est le socle même de la valeur scientifique des collections.


\subsection{Conclusion : De la maîtrise documentaire à l’autorité scientifique}

Ainsi, loin de n’être qu’un conservatoire, le Musée de l’Air et de l’Espace construit son rôle dans la recherche sur la maîtrise et la valorisation de ses outils documentaires. L’interopérabilité des vocabulaires, la gestion raisonnée des archives numériques, et l’inscription dans des réseaux collaboratifs constituent autant de leviers pour affirmer sa légitimité scientifique et sa capacité à irriguer, au-delà de ses murs, la réflexion sur le patrimoine aéronautique.