\section{\label{I-A-2}La recherche : le rôle déterminant d'un musée technique}

Le rôle du \ac{mae} ne se cantonne pas seulement à la seule conservation d’objets : celui-ci s'impose en effet comme un acteur essentiel de la recherche au carrefour entre histoire technique, aéronautique, sciences sociales et muséologie.  Le \ac{psc} du \acf{mae}, remanié en 2020, est un précieux témoin du difficile équilibre recherché par le musée pour assurer la visibilité et valorisation de ses fonds auprès du grand public comme de la communauté scientifique.

\subsection{Un acteur central dans les réseaux de recherche aéronautique}

Le \ac{mae} se retrouve en effet, comme bien des musées techniques ou beaux-arts, pris entre différents mondes : musées, bibliothèques, archives, centres de recherches, associations de passionnés... Une grande partie de sa mission consiste donc à assurer la communication entre ces différents qui échangent savoir, pratiques et innovations dans un réseau national comme international.

Ce rôle se manifeste dans la multiplication des expositions temporaires à dimension internationale : l’exposition \emph{Flight}, fruit d’un partenariat avec le Parque de las Ciencias de Grenade, le centre Techmania de Plzeň et l’Institut royal des Sciences naturelles de Belgique, illustre ainsi la capacité du musée à fédérer des acteurs divers autour d’une réflexion sur le vol humain et animal\footnote{Voir \href{https://www.museeairespace.fr/agenda/exposition-flight}{https://www.museeairespace.fr/agenda/exposition-flight}}. De même, des journées d’études, comme celle organisée en 2019 pour le centenaire de l’aviation civile\footnote{Programme disponible sur le site du musée \cite{19192019CentAns}}, réunissent des universitaires, des conservateurs, des ingénieurs ou des amateurs, se retrouvant pour croiser les regards et les méthodes pour améliorer notre compréhension de l'histoire ou encore de la sociologie de l'aéronautique.

L’engagement du musée ne s’arrête pas à la diffusion : il participe à des projets de recherche interdisciplinaires comme le programme C-ADER déposé auprès de l'\ac{anr}, et qui fédère le \ac{c2rmf}, l’Institut de recherche de chimie Paris, l’université de Lorraine et l’Institut de soudure, autour de la  question de la conservation des aéronefs exposés en extérieur\footnote{Voir  \href{https://anr.fr/Projet-ANR-22-CE27-0025}{https://anr.fr/Projet-ANR-22-CE27-0025}}. Ce projet transversal conduira entre autres à l’élaboration d’un thésaurus partagé : celui-ci est une nécessité pragmatique, mais aussi un acte intellectuel qui permet de faire dialoguer chimistes, restaurateurs, conservateurs et historiens avec une même langue. Le musée exerce pleinement dans ce programme son rôle de médiateur et de catalyseur de la recherche.

\subsection{Le réseau du \ac{mae} : mettre en place une politique et des outils documentaires adaptés}

\begin{quote}
	\og Le \acf{mae} doit développer ses réseaux \textelp{} dans les domaines civils et militaires. Pour ce dernier domaine, ils répondent à son statut de musée du ministère des Armées et par extension de sa sensibilisation de la société au monde militaire et à son histoire.
	Pour la sphère civile, cette influence est essentiellement réalisée autour des industriels afin de les sensibiliser au patrimoine de la troisième dimension, leur patrimoine ; de l’Enseignement Supérieur et des jeunes pour développer des synergies avec les formations en lien avec le milieu culturel et proposer un accès aux cursus disponibles dans les domaines de l’aéronautique et du spatial. Enfin \textelp{} le musée doit conserver son rôle d’institution de référence dans le domaine de la conservation du patrimoine de la troisième dimension auprès des associations aéronautiques et spatiales\footnote{\ac{psc} du \ac{mae}}.\fg
\end{quote}

Le \ac{psc} du \mae résume bien dans sa dernière
Cette vocation scientifique du \ac{mae} ne saurait prospérer sans une réflexion exigeante sur les outils qui la rendent possible. Le \ac{mae}, à l’instar des grandes institutions patrimoniales, se confronte à un paysage éclaté, où la profusion des bases de données, la diversité des formats et la spécialisation extrême des vocabulaires font facilement obstacle à l’intelligibilité du savoir. C’est pourquoi l’interopérabilité des vocabulaires contrôlés lui est la condition même d’une recherche efficace, capable de relier, d’interroger, de transmettre.

Adopter des normes partagées – \ac{skos}, ISO 25964 – et travailler à l’articulation entre les thésaurus internes et les grands référentiels nationaux comme RAMEAU ou IdRef, c’est inscrire le musée dans une dynamique de réseau : permettre à ses données de circuler, de s’enrichir, d’être réutilisées, c’est-à-dire de vivre\footcite{hudonISO25964Pour2012a,chichereau_normes_2007,nouvel_thesaurus_2019}. Ce chantier, encore inachevé, appelle la mobilisation de toutes les compétences, la prise en compte des spécificités du vocabulaire aéronautique et la volonté de ne jamais sacrifier la précision technique à la seule facilité d’alignement. Il s’agit moins de proclamer une normalisation absolue que d’élaborer des passerelles, des zones de contact intelligentes où la diversité des pratiques s’ordonne sans se dissoudre.

L’autre versant de cette politique documentaire concerne la gestion des archives numériques liées aux œuvres. La numérisation systématique des ressources iconographiques, la production croissante de dossiers de collection et de documentation, loin de n’être qu’un progrès matériel, font peser sur l’institution une responsabilité nouvelle : garantir la pérennité, l’intégrité, la traçabilité de ces flux d’information\footcite{ministere_de_la_culture_documenter_2020,bechard_archives_2020}. S’accumuler n’est pas conserver : le musée doit s’astreindre à élaborer des procédures d’archivage conformes aux référentiels nationaux\footcite{comite_interministeriel_aux_archives_de_france_referentiel_nodate}, à choisir des solutions logicielles adaptées (GED, SAE), à former son personnel à la rigueur de la gestion documentaire.

Cette maîtrise n’est pas un luxe mais une exigence : elle fonde la valeur scientifique des collections, assure leur transmissibilité, et garantit au musée sa capacité à irriguer, au-delà de ses murs, la réflexion sur le patrimoine aéronautique. Ainsi, le \ac{mae}, loin de se contenter d’accumuler des objets ou des fichiers, s’impose par la cohérence de ses choix et l’exemplarité de ses pratiques comme une référence nationale et un acteur majeur de la recherche.
