\chapter[Acteurs et dépendances]{\label{I-B}De nombreux acteurs et dépendances ministérielles }


\lettrine{L}e \maelong occupe une place singulière parmi les musées français, tant par la richesse de ses collections que par son rattachement institutionnel particulier. Étroitement lié au \minarm, il doit conjuguer son rôle de conservateur du patrimoine aéronautique avec les exigences et les contraintes propres à son statut d’établissement public sous tutelle militaire.

Cette situation institutionnelle n'est pas anecdotique : elle conditionne directement les choix documentaires du musée, de l'organisation de ses métadonnées aux outils de diffusion imposés par la tutelle. L'analyse de ces contraintes permettra de comprendre comment les enjeux institutionnels façonnent les pratiques informationnelles et génèrent des besoins spécifiques en matière de vocabulaires contrôlés.

\section{\label{I-B-1}A musée d’exception, contraintes d’exception : un musée étroitement dépendant du ministère de la Défense. }

Le Musée de l’Air et de l’Espace, en tant que musée d’exception, ne jouit paradoxalement d’aucune autonomie véritable. Placé sous l’autorité du ministère des Armées, il est pris dans un faisceau de décisions ministérielles qui conditionnent ses évolutions, qu’elles soient techniques, documentaires ou institutionnelles. Cela vaut tant pour les choix informatiques (comme la solution d’archivage SAE) que pour les outils de catalogage ou de diffusion des collections. L’intégration d’un outil tel qu’Opentheso dans Koha, par exemple, ne saurait être envisagée sans considérer son impact sur l’ensemble des musées et bibliothèques relevant de la même tutelle. Le musée partage son infrastructure avec d'autres entités défendant un patrimoine militaire et technique, et c’est donc à léchelle ministérielle que les décisions sont prises, parfois au prix d’une adaptation imparfaite aux besoins spécifiques du site du Bourget.\footnote{Par exemple, la cartographie des flux de données de thésaurus au \mae en Annexe \ref{Ax-C} montre bien le rôle central du ministère des armées comme validateur et gestionnaire des bases de données publiques du musée (Archange et Clade).}

Ces décisions techniques sont également influencées par les priorités politiques du ministère, notamment en ce qui concerne la présence du musée dans les grands événements comme le salon du Bourget. L’ANSI (Agence du Numérique de la Sécurité Informatique) joue par ailleurs un rôle déterminant dans la validation des solutions techniques, imposant des contraintes de sécurité parfois difficilement conciliables avec les outils de la recherche ou du patrimoine. La gestion documentaire, l’interopérabilité des thésaurus, ou encore la diffusion des collections ne peuvent donc être pensées uniquement à léchelle du musée lui-même. À cela s’ajoute un panorama complexe de tutelles multiples : entre la Direction de la Mémoire, de la Culture et des Archives (DMCA), la Délégation à l’Information et à la Communication de la Défense (DICoD), et les directions propres à chaque arme, les marges de manœuvre apparaissent étroites.



\section{\label{I-B-2}Un musée qui s'inclut dans un ensemble de choix politiques qui lui sont indépendants}

Ici, mon texte


\bigskip
\bigskip
\bigskip

Le \maelong, musée de France doté d’un statut prestigieux et d’un patrimoine exceptionnel, évolue ainsi dans un cadre institutionnel et administratif fortement marqué par sa dépendance au ministère des Armées. Cette situation conditionne ses choix stratégiques, ses ressources, mais aussi son identité culturelle : le tout reflétant à la fois les contraintes spécifiques d’un établissement public militaire et les enjeux généraux liés à la conservation et à la valorisation de la mémoire aéronautique civile et militaire française. 

\bigskip
\bigskip
\bigskip