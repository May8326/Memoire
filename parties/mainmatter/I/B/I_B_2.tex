\section{\label{I-B-2}Un musée qui s'inclut dans un ensemble de choix politiques qui lui sont indépendants}

Le Musée de l’Air et de l’Espace s’inscrit dans une stratégie culturelle définie à l’échelle ministérielle, et dont les conséquences se répercutent jusque dans les moindres aspects de la mise à disposition de ses collections. Le portail CLADE (Catalogue des bibliothèques de la Défense), outil de diffusion mis en place par le ministère, constitue à ce titre un cas emblématique. Ce portail n’offre pas actuellement la possibilité d’effectuer une recherche par mots-clés contrôlés, alors même que des mois de travail ont été consacrés à la constitution d’un thésaurus documentaire structurant. Cette absence d’interopérabilité réduit à néant l’objectif même du thésaurus, en invisibilisant les efforts de structuration et en limitant l’accès intellectuel aux collections pour les chercheurs comme pour le grand public.

Plus largement, les choix politiques concernant les mutations institutionnelles, les déménagements ou la mutualisation des moyens entre musées dépendent de stratégies extérieures au musée. Le lien avec les autres institutions militaires patrimoniales (comme les Invalides, le musée de la Marine ou les musées de l’Armée de terre) est plus ou moins actif selon les moments, les projets ou les arbitrages. Or ces musées partagent souvent les mêmes problématiques d’interopérabilité, de gestion des archives numériques et de diffusion des données, sans pour autant pouvoir harmoniser leurs pratiques. Le Musée de l’Air se retrouve ainsi au carrefour de décisions nationales, dont il subit les conséquences sans toujours pouvoir peser sur leur orientation. C’est dans cette tension entre particularité scientifique forte et soumission à des logiques ministérielles générales que se joue l’avenir de ses systèmes d’information.
