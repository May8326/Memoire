\section{\label{I-B-2}Conséquences de cette dépendances sur le quotidien du musée}



Lors des missions qui ont été réalisées durant ce stage, ce sont surtout dans les choix d'outils informatiques de catalogage et de diffusion des collections que c'est ressentie cette contrainte. Les chantiers de mise en place de nouveaux logiciels qui se sont achevé cet été 2025 ont par exemple tous deux été pilotés par le ministère de la Défense : la mise en place du logiciel de gestion des collections \textit{Archange} (\textit{S/Museum} de \textit{Skinsoft}) s'inclut dans un projet progressif d'intégration des musées du ministère sur une même plateforme de gestion des collections. La migration vers le \ac{sigb} \textit{Koha} pour la bibliothèque, et la diffusion de ses collections sur la plateforme \textit{Clade} s'inscrit dans un projet similaire pour toutes les bibliothèques du ministère, afin d'unifier leur gestion mais aussi d'améliorer leur accessibilité en permettant à l'utilisateur d'interroger sur le même portail l'ensemble des \gls{bibmusee}.


Le musée peut ainsi se retrouver dans des positions délicates : dans le cas de la migration vers \gls{clade}, à laquelle j'ai pu être confrontée lors de mon stage, les responsables du \ac{drd} ont dû faire face à des difficultés particulières, directement causées par l'intégration de la bibliothèque du \mae à un réseau qui lui est plus large. D'un côté en effet, ce projet offre de grands avantages en matière d'accessibilité des catalogues : centralisation de la recherche, recherche par mots-clés, intégration de documents numériques téléchargeables par exemple\footcite{ministeredesarmeesKitCommunicationCLADE}. Elle permet aussi à des institutions plus petites de participer à un projet qu'elles n'auraient peut-être pas eu les ressources de mener indépendamment. Cependant, cela signifie également qu'il devient plus délicat de s'adapter aux exigences et aux habitudes de gestion de chacun, et cela devient problématique dans le cas de bibliothèques spécialisées comme celle du \mae. Par exemple, le projet est géré à l'échelle du ministère : cela signifie que les agents du musée, et plus particulièrement du \ac{drd}, n'ont que très peu de visibilité sur le déroulé de la migration : ceux-ci n'ont par exemple jamais eu accès à au cahier des charges du projet. Il devient donc difficile pour les utilisateurs d'identifier d'éventuels problèmes pendant la migration : ce n'est ainsi qu'à la toute fin de la phase de test qu'il a été découvert que la structure du fichier d'import du thésaurus avait été mal comprise par les chargés de la migration des données, et que ceci avait causé des inexactitudes lors de l'import, extrêmement difficiles à corriger après coup.