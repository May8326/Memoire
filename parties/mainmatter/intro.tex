\chapter{Introduction}	

\begin{quote}
    \og La Bibliothèque comporte toutes les structures verbales, toutes les variations que permettent les vingt-cinq symboles orthographiques, mais point un seul non-sens absolu [...] Je ne puis combiner une série quelconque de caractères, par exemple	\textit{Dhcmrlchtdj} que la divine Bibliothèque n’ait déjà prévue, et qui dans quelqu'une de ses langues secrètes ne renferme
	une signification terrible\footcite{borgesBibliothequeBabel2011}.\fg
	
	Cette proximité soulève néanmoins une question essentielle : dans quelle mesure le Musée de l’Air et de l’Espace, en étant si étroitement associé à une manifestation commerciale, peut-il conserver sa posture de conservateur impartial et de référence scientifique ? TODO: Renseignements sur les changements d'organigramme.
	%Cette interrogation traverse les pratiques et les choix stratégiques du musée, notamment dans ses efforts pour se professionnaliser et renforcer son expertise muséale. La question de la tutelle et des modes de gestion n’est pas moins cruciale : longtemps administré par des instances militaires, le musée a dû repenser ses organigrammes, en répartissant clairement les responsabilités entre Recherche, Documentation et Conservation, regroupées aujourd’hui dans un département unique, le DSC (Département des Collections). Ce regroupement vise à favoriser les synergies, mais il pose aussi des défis, notamment en termes de gestion des archives, où les moyens restent limités.
	
	
\end{quote}