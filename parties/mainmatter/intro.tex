\chapter{Introduction}	
\enquote{Face au potentiel de reproductibilité à l’infini du numérique, comment anticiper l’explosion des documents et des ressources ? Le défi du numérique pour les musées est peut être d’envisager une stratégie globale destinée à contrôler la croissance exponentielle des documents physiques et numériques et de leurs métadonnées par la mise en place d’une véritable gouvernance de l’information patrimoniale et renforcer au sein du musée un processus collectif, mutualisé et transversal de la chaîne documentaire\footcite{rizzaDocumentAuCoeur2014}.}

%Ce constat soulève une autre question majeure : comment garantir la pérennité et la valorisation d’un patrimoine documentaire aussi riche avec des ressources humaines aussi restreintes ?
Définir ce qu'est un thésaurus
\begin{quote}
    \og La Bibliothèque comporte toutes les structures verbales, toutes les variations que permettent les vingt-cinq symboles orthographiques, mais point un seul non-sens absolu [...] Je ne puis combiner une série quelconque de caractères, par exemple	\textit{Dhcmrlchtdj} que la divine Bibliothèque n’ait déjà prévue, et qui dans quelqu'une de ses langues secrètes ne renferme
	une signification terrible\footcite{borgesBibliothequeBabel1990}.\fg
\end{quote}	
	%	Cette proximité soulève néanmoins une question essentielle : dans quelle mesure le Musée de l’Air et de l’Espace, en étant si étroitement associé à une manifestation commerciale, peut-il conserver sa posture de conservateur impartial et de référence scientifique ? TODO: Renseignements sur les changements d'organigramme.
	%Cette interrogation traverse les pratiques et les choix stratégiques du musée, notamment dans ses efforts pour se professionnaliser et renforcer son expertise muséale. La question de la tutelle et des modes de gestion n’est pas moins cruciale : longtemps administré par des instances militaires, le musée a dû repenser ses organigrammes, en répartissant clairement les responsabilités entre Recherche, Documentation et Conservation, regroupées aujourd’hui dans un département unique, le DSC (Département des Collections). Ce regroupement vise à favoriser les synergies, mais il pose aussi des défis, notamment en termes de gestion des archives, où les moyens restent limités.
	

	%		Les constats établis dans cette première partie guident la méthodologie adoptée dans ce mémoire : l'analyse des pratiques documentaires actuelles du MAE (Partie II) permettra d'identifier les besoins réels des utilisateurs, tandis que l'étude comparative avec d'autres institutions similaires (Partie III) révélera les solutions méthodologiques envisageables pour concilier spécialisation et interopérabilité.
	



% PROPOSITION A S'APPROPRIER,RETRAVAILLER,MODIFIER --- TEXT FILLER

\lettrine{C}{ette} vision borgésienne d'une bibliothèque infinie, où chaque livre contient toutes les combinaisons possibles de caractères, trouve un écho singulier dans les défis contemporains des institutions patrimoniales. Face à la prolifération exponentielle de l'information numérique, les musées se trouvent confrontés à leur propre « bibliothèque de Babel » : une accumulation de données, de vocabulaires spécialisés et d'archives numériques qui menace de rendre leurs collections aussi inaccessibles que les livres impossibles à déchiffrer de l'univers borgésien.

Le \acf{mae} du Bourget incarne parfaitement cette problématique contemporaine. Institution technique aux collections exceptionnelles, il articule recherche spécialisée, conservation patrimoniale et médiation culturelle dans un contexte institutionnel contraint par sa tutelle militaire. Cette situation génère des enjeux documentaires spécifiques : comment organiser l'information pour qu'elle soit simultanément accessible aux ingénieurs aéronautiques, aux historiens, aux conservateurs et au grand public ? Comment concilier la richesse sémantique des vocabulaires techniques avec les contraintes d'interopérabilité imposées par les réseaux ministériels ?

\section*{Problématique}

Ces constats nous amènent à formuler la question centrale de cette recherche : \textbf{comment les musées peuvent-ils repenser leur gouvernance de l'information face à la prolifération des données et des vocabulaires spécialisés ?} Cette problématique sera illustrée à travers le cas du musée de l'Air et de l'Espace et l'enjeu de l'interopérabilité de ses thésaurus et de sa politique d'archivage numérique.

Cette question dépasse le cadre technique de la gestion documentaire : elle interroge les modalités contemporaines de production, d'organisation et de transmission du savoir dans les institutions patrimoniales. À l'ère du numérique, les musées techniques comme le MAE doivent inventer de nouvelles formes de gouvernance informationnelle qui préservent la spécificité de leurs missions tout en s'intégrant dans des écosystèmes documentaires plus larges.

\section*{Enjeux et hypothèses}

Trois hypothèses principales guident cette réflexion. \textbf{Premièrement}, les contraintes institutionnelles et la diversité des publics des musées techniques génèrent des besoins documentaires spécifiques qui ne peuvent être satisfaits par les approches traditionnelles de gestion de l'information. \textbf{Deuxièmement}, la prolifération des vocabulaires contrôlés et des archives numériques en contexte muséal révèle l'inadéquation des outils et méthodes actuels face au volume et à la complexité des données patrimoniales. \textbf{Troisièmement}, les solutions techniques émergentes (modélisation conceptuelle, intelligence artificielle, outils collaboratifs) offrent des perspectives prometteuses pour repenser la gouvernance de l'information, à condition d'être adaptées aux spécificités des institutions patrimoniales.

\section*{Méthodologie et corpus}

Cette recherche s'appuie sur une approche empirique fondée sur l'observation participante réalisée lors d'un stage de six mois au sein du \ac{drd} du MAE. L'analyse porte sur un corpus constitué des trois thésaurus principaux de l'institution (collections muséales, bibliothèque, e-médiathèque), des archives numériques liées aux œuvres, et des outils de gestion documentaire actuellement en usage. Cette étude de cas est complétée par une analyse comparative avec d'autres institutions patrimoniales confrontées à des défis similaires.

L'approche privilégie le dialogue entre théorie et pratique : aux observations de terrain s'articulent des expérimentations techniques (modélisation conceptuelle, traitement automatisé des données) et une réflexion méthodologique sur les enjeux de l'interopérabilité en contexte patrimonial.

\section*{Plan}

Cette démonstration s'organise en trois temps. La \textbf{première partie} établit le contexte institutionnel spécifique du MAE et montre comment ses particularités (tutelle militaire, diversité des collections, multiplicité des publics) génèrent des enjeux documentaires particuliers. Cette contextualisation révèle que les défis de gouvernance informationnelle ne peuvent être appréhendés indépendamment des contraintes institutionnelles qui les façonnent.

La \textbf{seconde partie} analyse les manifestations concrètes de la prolifération informationnelle au MAE : fragmentation des vocabulaires contrôlés, difficultés de gestion des archives numériques, différences d'appréhension selon les métiers. Cette analyse diagnostique révèle l'ampleur des dysfonctionnements actuels et la nécessité d'une approche globale de la gouvernance informationnelle.

La \textbf{troisième partie} explore les outils et méthodes susceptibles de répondre à ces défis : modélisation conceptuelle, processus collaboratifs d'unification des thésaurus, apport potentiel de l'intelligence artificielle. Cette partie prospective évalue les solutions techniques à l'aune des contraintes institutionnelles identifiées dans les parties précédentes.

\section*{Portée et limites}

Cette recherche vise à contribuer aux réflexions contemporaines sur la transformation numérique des institutions patrimoniales. Si le cas du MAE présente des spécificités liées à son statut et à ses collections, les enjeux de gouvernance informationnelle qu'il révèle concernent l'ensemble des musées techniques confrontés à la massification des données numériques. Les solutions méthodologiques explorées dans ce mémoire ont vocation à nourrir les pratiques professionnelles au-delà du seul contexte aéronautique.

Toutefois, cette étude présente des limites qu'il convient de souligner. L'approche monographique, si elle permet une analyse fine des enjeux institutionnels, limite la généralisation des conclusions. Par ailleurs, les expérimentations techniques réalisées dans le cadre de ce stage demeurent partielles et nécessiteraient des développements ultérieurs pour mesurer pleinement leur efficacité opérationnelle.

Malgré ces limites, cette recherche entend démontrer que la gouvernance de l'information en contexte patrimonial constitue un enjeu stratégique majeur pour l'avenir des institutions culturelles, et que sa maîtrise conditionne leur capacité à remplir leurs missions de conservation, de recherche et de médiation dans l'écosystème numérique contemporain.