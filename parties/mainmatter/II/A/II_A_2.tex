\section{\label{II-A-2}Des conséquences importantes : quand la prolifération devient paralysie documentaire}

L’exercice de diagnostic mené sur les thésaurus du \mae~révèle que plus l’institution enrichit ses descriptions, plus elle complexifie l’accès à ses propres collections. Cette prolifération de vocabulaires entraîne des complications, incitant les chargés de collections et documentalistes à reconsidérer leurs méthodes actuelles. L’observation de terrain révèle deux manifestations principales de cette dégradation : l’invisibilisation progressive des collections et la saturation dans l’organisation des équipes, ces deux facteurs contribuant à rendre plus difficile encore toute rationalisation de l’accumulation d’information au musée.

\subsection{L’invisibilisation documentaire : effet direct de la fragmentation des vocabulaires}

L’un des défis relevé à plusieurs reprises dans les échanges avec les agents du \mae~est la difficulté croissante à retrouver certains documents ou objets, malgré l’enrichissement constant des vocabulaires. Cette situation ne tient pas seulement à la prolifération des termes, mais aussi à leur organisation disjointe : la coexistence de thésaurus parallèles et non coordonnés enferme les informations dans des silos que rien ne relie. Cette fragmentation est notamment étudiée par Richard Gartner et Raphaëlle Mouren dans un article sur la méthodologie mise en place pour éviter ces écueils à la Warburg Institute Library. Bien que le contexte technologique ne soit pas exactement le même, cette analyse s’applique également au \mae : les différences de vocabulaires y sont surtout liées aux différences de conception des métadonnées en général entre les métiers des archives, des musées et des bibliothèques. Même si le musée n’utilise pas les mêmes standards, la citation reste pertinente :  

\begin{myquote}
	{All three standards have achieved some degree of success within their respective sectors in enabling the sharing and transmission of metadata. There is, however, little interoperability between these three approaches, and consequently between their respective communities, owing to their differing underlying architectures which in turn owe their origins to the very different approaches to metadata that have applied for centuries within each sector. Without this interoperability it is difficult to enable cross-sector or cross-community discoverability, to allow the valuable heritage materials held within, for instance, the library sector to be accessible to users in archives or museums. One of the key challenges in coming years must be to find ways to bridge the gaps between these diverse approaches in order to allow the holdings of the cultural heritage sector as a whole to be discoverable whatever the community to which an individual researcher belongs.\footnote{\enquote{\textit{Les trois standards ont chacun rencontré un certain succès dans leurs secteurs respectifs en facilitant le partage et la transmission des métadonnées. Toutefois, l’interopérabilité entre ces trois approches demeure limitée, tout comme entre les communautés qui leur sont associées, en raison de leurs architectures sous-jacentes distinctes, elles-mêmes héritées de conceptions très différentes des métadonnées, enracinées depuis des siècles dans chaque domaine. Cette absence d’interopérabilité complique la découvrabilité intersectorielle ou intercommunautaire, empêchant par exemple les utilisateurs des archives ou des musées d’accéder aisément aux ressources patrimoniales précieuses conservées dans les bibliothèques. L’un des principaux défis des années à venir consistera à trouver des moyens de combler les écarts entre ces approches hétérogènes, afin de rendre les collections du secteur du patrimoine culturel accessibles, quelle que soit la communauté à laquelle appartient un chercheur.}}}}{gartnerArchivesMuseumsLibraries2019}
\end{myquote}

Au \mae, cette situation se concrétise par le fait qu’une information (par exemple, un modèle d’avion rattaché à son constructeur) pourra se retrouver dans le thésaurus des aéronefs de l’\gls{emediatheque} et non dans la table des constructeurs de \gls{micromusee}, et le chercheur voulant accéder à la totalité des connaissances détenues par l’institution sur le sujet doit penser aux différents moyens d’accès disponibles dans chaque cas.

La manière d’organiser l’information est tout aussi importante que son contenu : au \mae, où de nombreuses imprécisions ou erreurs sont restées sans correction, il devient difficile d’accéder à certains documents en utilisant des mots-clés qui seraient pourtant intuitifs. Par exemple, pour une recherche géographique dans l’\gls{emediatheque}, une photographie indexée sous un nom de ville devrait pouvoir apparaître lors d’une recherche par région. Or, dans la branche « lieux » des mots-clés de l’\gls{emediatheque}, il existe des cas où le terme générique direct est un pays et non une région. Il en résulte qu’une photographie indexée sous « Pont-l’Évêque » rattaché directement à « France » ne sera pas retrouvée par un chercheur interrogeant le fonds sous « Normandie » ou « Calvados ». Sous un autre aspect, de nombreux objets techniques, catalogués avec des dénominations précises propres au monde de l’aéronautique, échappent aux requêtes du grand public, qui n’en connaît pas le vocabulaire.

\begin{figure}[htbp]
	\begin{adjustbox}{width=\textwidth,center}
	\begin{tikzpicture}[
		level distance=1.6cm,
		sibling distance=2.5cm,
		every node/.style={font=\sffamily, align=center, rounded corners, draw=lightgray!80, fill=lightgray!30, minimum width=2.6cm, minimum height=0.9cm},
		edge from parent/.style={draw,-latex},
		xshift=0cm
		]
		
		% Arbre bien structuré à gauche
		\node (fr) {France}
		child {node (norm) {Normandie}
			child {node (calv) {Calvados}
				child {node (pont) {Pont-l’Évêque}}
			}
		};
		
		% Arbre mal structuré à droite
		\node[right=7cm of fr] (frbad) {France}
		child {node (normbad) {Normandie}
			child {node (calvbad) {Calvados}}
		}
		child {node (pontbad) {Pont-l’Évêque}};
		
		% Titres sous les arbres
		\node[draw=none, fill=none, below=2.1cm of calv, font=\small\bfseries] (label1) {Hiérarchie conforme\\(recherche par région possible)};
		\node[draw=none, fill=none, below=2.1cm of calvbad, font=\small\bfseries] (label2) {Organisation défaillante\\(Pont-l’Évêque invisible par région)};
		
		% Légende explication
		\node[draw=none, fill=none, above=0.6cm of fr, font=\small] {Arborescence idéale};
		\node[draw=none, fill=none, above=0.6cm of frbad, font=\small] {Arborescence problématique};
		
	\end{tikzpicture}
\end{adjustbox}
	\caption{Exemple de rattachement problématique dans le thésaurus : « Pont-l’Évêque » rattaché directement à « France », ce qui empêche sa remontée lors d'une recherche par région (« Normandie », « Calvados »).}
	\label{fig:thesaurus_geo_pb}
\end{figure}


Cette problématique rejoint une exigence ancienne du métier de documentaliste formulée par Magdeleine Moureau dès 1968 :

\begin{myquote}
	{Chaque document doit pouvoir satisfaire aux deux objectifs documentaires : diffusion systématique et recherche sur question, et pouvoir restituer le même document lors d’une question générique ou lors d’une question spécifique. Cette possibilité de répondre à plusieurs niveaux pourra s’obtenir de deux façons : soit par l’indexateur humain qui rajoutera pour chaque document particulier le thème général dont il procède, soit par la machine qui associera automatiquement certaines notions génériques à certaines notions spécifiques, par exemple Europe à France ou aromatique à benzène}{moureauProblemesPosesPar1968}
\end{myquote}

Le \gls{thesaurus} est l’outil qui a été choisi dans les années 1990 par le \mae~pour répondre à ces objectifs. Son passage au second plan pendant les dernières décennies impacte donc directement ces deux objectifs documentaires et muséaux.

Cette invisibilisation n’est pas une fatalité : elle invite à repenser la coordination des vocabulaires et à développer des outils de recherche capables de traverser ces silos, en mobilisant les méthodes de normalisation et d’interopérabilité proposées par les dernières normes ISO\footcite{chichereauNormesConceptionGestion2007}.

\subsection{La saturation des métiers : quand gérer devient ingérable}

Au quotidien, on observe que l’adéquation des outils de gestion documentaire reste un enjeu central pour les équipes du musée. Par exemple, la version 6 de \gls{micromusee}, logiciel développé dans les années 1990 et resté en usage jusqu’en 2025, a été souvent décrite comme « antique » par ses utilisateurs : ce cas illustre bien les difficultés d’adaptation et de gestion auxquelles ont été confrontés les agents face aux limites ou à la présentation de certains outils. Ce logiciel présentait toutes les fonctionnalités nécessaires pour la gestion des collections : gestion des notices, modification par lot, gestion de thésaurus... mais il était devenu difficile à lancer sur des systèmes d’exploitation actuels, très peu ergonomique, et surtout n’offrait aucune possibilité d’interopérabilité avec d’autres logiciels. La migration vers \gls{archange} a ainsi nécessité l’engagement d’un renfort informatique pour traiter les nombreux fichiers texte en ASCII exportés de la base et les transposer dans des fichiers csv plus lisibles et transmissibles au prestataire de migration.

Comme le souligne Hélène Vassal dans l'ouvrage \citetitle{merleau-pontyDocumenterCollectionsMusees2016}, \enquote{l’accès aux œuvres passe aussi par l’accès à leur connaissance}, et cet accès passe aujourd’hui par des outils informatiques qui \enquote{[rendent] l’utilisation des bases de données et de leurs applications indissociables des pratiques quotidiennes des professionnels de la documentation et de la régie\footcite{merleau-pontyDocumenterCollectionsMusees2016}.} Le volume des données concernées, souvent en milliers de mots, n’est pas toujours traitable pièce à pièce par l’humain.

Cependant, comme le rappelle Maryse Rizza dans \citetitle{rizzaDocumentAuCoeur2014}, le numérique n’est pas un but en soi mais il est \enquote{au service du musée et de ses missions\footcite{ricardRGPDArchives2018a}}. Cette évolution transforme néanmoins profondément l’exercice des métiers muséaux en imposant \enquote{un élargissement des compétences pour les professionnels de musée qui assurent le traitement documentaire des documents numériques\footcite{rizzaDocumentAuCoeur2014}.}. Cette tension entre l’expertise approfondie que requiert la gestion de vocabulaires spécialisés, et la dispersion des compétences qu’impose la multiplication des systèmes, se retrouve aujourd’hui au \mae. Les agents du \ac{drd} doivent maîtriser simultanément \gls{koha} et l’\gls{emediatheque}, et se familiariser avec le nouvel outil de gestion des collections muséales \gls{archange}. La gestion de \gls{thesaurus} — et surtout le chantier de nettoyage, mise aux normes, et migration vers un logiciel pour faciliter son exploitation — se surajoute aux missions des agents, en leur demandant d’acquérir de nouvelles compétences et de dégager du temps pour s’y consacrer. Il en est de même pour la gestionnaire de base de données côté musée, pour qui ce travail s’ajoute à ses missions d’origine.

L’organisation de l’information au musée, que ce soit au travers de la gestion d’un thésaurus ou autre, représente des défis métiers majeurs :
\begin{itemize}
	\item Cette gestion revient à manipuler des masses considérables de données, ce qui demande un temps de travail que n’ont pas nécessairement les agents à qui reviendrait plus naturellement ce rôle,
	\item elle dépend très étroitement de la qualité des outils mis à leur disposition,
	\item et demande de se former continuellement à des pratiques toujours en évolution.
\end{itemize}