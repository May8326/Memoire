\section{\label{II-A-2}Des conséquences importantes : quand la prolifération devient paralysie documentaire}

L'exercice de diagnostic mené sur les thésaurus du \mae révèle une réalité paradoxale : plus l'institution enrichit ses descriptions, plus elle complexifie l'accès à ses propres collections. Cette prolifération de vocabulaires produit aujourd'hui des effets pervers, qui conduisent chargés de collections et documentalistes à s'interroger sur les manières de faire actuelles. L'observation de terrain révèle trois manifestations principales de cette dégradation : l'invisibilisation progressive des collections, la saturation dans l'organisation des équipes, et la difficulté croissante de rationaliser l'accumulation d'informations.

\subsection{L'invisibilisation documentaire : effet direct de la fragmentation des vocabulaires}

L’un des défis relevé à plusieurs reprises dans les échanges avec les agents du \mae est la difficulté croissante à retrouver certains documents ou objets, malgré l’enrichissement constant des vocabulaires. Cette situation ne tient pas seulement à la prolifération des termes, mais à leur organisation disjointe : tout d'abord, la coexistence de thésaurus parallèles et non coordonnés, enferme les informations dans des silos que rien ne relie. Cette fragmentation est notamment étudiée par Richard Gartner et Raphaëlle Mouren dans un article sur la méthodologie mise en place pour éviter ces écueils à la Warburg Institute Library\footcite{gartnerArchivesMuseumsLibraries2019}. Cet article, bien que le contexte technologique ne soit pas exactement le même, se rapproche de la situation rencontrée au \mae : les différences de vocabulaires y sont surtout liées aux différences de conception des métadonnées en général entre les métiers des archives, des musées et des bibliothèques. Bien que le musée n'utilise pas les différents standards utilisés à la Warburg Institute Library, et que ce facteur n'y ait pas autant d'importance, cette analyse s'y applique également : \enquote{There is \textelp{} little interoperability between these three approaches, and consequently between their respective communities, owing to their differing underlying architectures which in turn owe their origins to the very different approaches to metadata that have applied for centuries within each sector. Without this interoperability it is difficult to enable cross-sector or cross-community discoverability, to allow the valuable heritage materials held within, for instance, the library sector to be accessible to users in archives or museums\footnote{\textit{Il existe très peu d’interopérabilité entre ces trois approches, et par conséquent entre leurs communautés respectives, en raison de leurs architectures sous-jacentes distinctes, elles-mêmes issues de conceptions très différentes des métadonnées qui se sont développées depuis des siècles au sein de chaque secteur. En l’absence de cette interopérabilité, il devient difficile de favoriser la découvrabilité entre différents secteurs ou intercommunautaire, et donc de permettre aux utilisateurs des archives ou des musées d’accéder aux précieuses ressources patrimoniales conservées, par exemple, dans le secteur des bibliothèques.}\cite{gartnerArchivesMuseumsLibraries2019}}.} Au \mae, cette situation se concrétise par le fait qu'une information (par exemple, un modèle d'avion rattaché à son constructeur) se retrouvera dans le thésaurus des Aéronefs de l'\gls{emediatheque} et non dans la table des constructeurs de \gls{micromusee}, et le chercheur voulant accéder à la totalité des connaissances détenues par l'institution sur le sujet doit penser aux différents moyens d'accès disponible dans chaque cas.

Or, la manière d'organiser l'information est tout aussi importante que son contenu : dans le cas du \mae où, au fil des années, de nombreuses imprécisions ou erreurs sont restées sans correction, il devient difficile d'accéder à certains documents en utilisant des mots-clés qui seraient pourtant intuitifs. Par exemple, une recherche géographique dans l’e-médiathèque : une photographie indexée sous un nom de ville devrait pouvoir apparaître lors d'une recherche par région. Or, dans la branche \enquote{lieux} des mots-clés de l'\gls{emediatheque}, le terme générique direct est un pays et non une région ; il en résulte qu’une photographie indexée sous « Pont-l’Évêque » rattachée directement à « France » ne sera pas retrouvée par un chercheur interrogeant le fonds sous « Normandie » ou « Calvados ». De même, de nombreux objets techniques, catalogués avec des dénominations précises propres au monde de l’aéronautique, échappent aux requêtes du grand public, qui n’en connaît pas le vocabulaire.

Cette problématique rejoint une exigence ancienne du métier de documentaliste, brillamment formulée par Magdeleine Moureau dès 1968 : 
\begin{quote}
	\og En outre chaque document doit pouvoir satisfaire aux deux objectifs documentaires : diffusion systématique et recherche sur question, et pouvoir restituer le même document lors d'une question générique ou lors d'une question spécifique. Cette possibilité de répondre à plusieurs niveaux pourra s'obtenir de deux façons : soit par l'indexateur humain qui rajoutera pour chaque document particulier le thème général dont il procède, soit par la machine qui associera automatiquement certaines notions génériques à certaines notions spécifiques, par exemple Europe à France ou aromatique à benzène\footcite{moureauProblemesPosesPar1968}.\fg
\end{quote}
On voit ici combien la structuration des vocabulaires, la hiérarchie des concepts et la gestion des synonymies ne sont pas de simples détails techniques : ils conditionnent l’accessibilité même de l’information. L’absence d’harmonisation des thésaurus au sein du musée, l’absence de passerelles entre les corpus, et la dépendance au « bon mot » pour chaque recherche, sont autant de facteurs d’invisibilisation de pans entiers des collections.

Ce cloisonnement n’est pas qu’un problème technique : il traduit une logique institutionnelle, héritée de l’histoire des outils et des métiers du musée. Comme le rappelle la genèse des thésaurus du \mae\footnote{Voir infra, \ref{II-A-1}}, la bibliothèque, le musée et l’e-médiathèque ont chacun développé leur propre vocabulaire selon des besoins spécifiques, sans réflexion globale sur leur articulation. Les documentalistes consultent parfois les thésaurus des collections pour enrichir le leur, mais la démarche inverse reste exceptionnelle. Petit à petit, des fonds documentaires deviennent invisibles pour certains métiers ou pour le public, alors même qu’ils sont rigoureusement indexés.

Cette invisibilisation n’est pas une fatalité : elle invite à repenser la coordination des vocabulaires et à développer des outils de recherche capables de traverser ces silos, en mobilisant les méthodes de normalisation et d’interopérabilité proposées par les dernières normes ISO\footcite{chichereauNormesConceptionGestion2007, hudonISO25964Pour2012a}.