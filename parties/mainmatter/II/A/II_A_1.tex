\section{\label{II-A-1}Une construction séparée : 25 ans d'évolution en silo}


Gérer un musée, gérer une bibliothèque ou un projet de recherche, c'est se confronter au savoir, et devoir choisir entre les multitudes de **EXPRESSION** subjectives qui peuvent l'exprimer.
 
Comme de nombreux musées, le MAE a ainsi éprouvé le besoin de contrôler la manière dont il nommerait les objets, fournissant ainsi à ses agents un cadre strict pour décrire et analyser ses collections. Au musée, à la bibliothèque, se sont donc développés, conjointement d'abord puis séparément, des outils de contrôles de ce vocabulaire : des thésaurus qui definiraient à la fois les choix faits par le musée pour écrire les noms d'avions ou de constructeurs d'avions, et les mots qui pourraient qualifier des concepts liés ou non à l'aérospatiale, spécialités du musée. 

\subsection{Histoire des thésaurus} 

• Historique des thésaurus = historique de l’indexation et de ses différents 
logiciels, histoire du rapport du musée au numérique
o Musée d’abord, et 
o Alexandrie (bibliothèque)
o e-médiathèque (récent, dérivé du musée)
• Le défi de l’unification des vocabulaires contrôlés en institutions patrimoniales 
(bib + musée). Créer un thésaurus commun à partir de plusieurs déjà existants et 
construits différemment => pour une cohérence intellectuelle du musée et de sa 
mission de recherche dans l’aéronautique.
• Panorama des logiciels utilisés au musée, des utilisateurs, de la manière 
d’indexer, des relations/ou non-relations entre eux. Axiell, Koha, Skinsoft
• Chiffres et statistiques des thésaurus traités, présentation de leur articulation, 
etc.