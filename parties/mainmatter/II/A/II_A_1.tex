\section{\label{II-A-1}Une construction séparée : 25 ans d'évolution en silo}

L’histoire des thésaurus au sein de l’établissement n’obéit pas à un plan concerté, mais à une sédimentation de pratiques, de logiciels et de métiers. Trois ensembles se partagent aujourd'hui la connaissance du musée : les thésaurus de la bibliothèque gérés dans le logiciel Alexandrie, ceux des collections muséales gérés dans le logiciel Micromusée, et ceux de l’e-médiathèque, gérés dans le logiciel de gestion dédié aux documents iconographiques et audiovisuels. Ces trois corpus de termes, bien que partageant une ambition commune – ordonner, nommer, rendre trouvable – ne sont pas nés du même mouvement ni selon les mêmes logiques. Chacun de ses trois ensembles sont en réalité constitués de plusieurs thésaurus ou listes d'autorités distincts, dont les plus importants sont ceux des mots-clés et des constructeurs d'aéronefs, qui se retrouvent dans les champs d'indexation des collections.


L'ensemble des informations recueillies au musée pour recréer une chronologie des thésaurus du musée et la méthodologie qui a été appliquée résultent pour la plupart de groupes de travail anciens ou de réflexions ponctuelles liées à des difficultés de description d'un objet en particulier. Ils sont très rarement renseignés, ou du moins l'information est difficilement récupérable dans les archives du musée, et ce bref historique provient tout autant de la mémoire des agents que des documents contemporains qui ont été retrouvés.

\subsection{Les prémices : Alexandrie et Micromusée, une coexistence sans concertation (1996 – années 2010)}

Le premier thésaurus à voir le jour est le thésaurus de la bibliothèque, mis en place dès 1996. Conçu pour accompagner la structuration du catalogue et définir précisément les termes à utiliser pour les \glspl{autorite}, il répond aux exigences classiques du monde documentaire : classification rigoureuse, maîtrise du vocabulaire, liens hiérarchiques. Celui-ci s’inscrit dans une tradition de bibliothéconomie maîtrisée par les professionnels de la documentation.

Parallèlement en 2000, le logiciel Micromusée devient l’outil principal de gestion des collections muséales. Il s’appuie sur une base propre, structurée différemment, dont la logique s’articule davantage autour des objets matériels que de concepts abstraits. 

Des comités de pilotage \footnote{Voir l'interview de Vincent Dhorne en Annexe\ref{Ax-D}}  ont été organisés en 1998 entre les chargés de collections et les documentalistes autour du thésaurus, notamment lors de l'import sur Micromusée des photos conservées par la bibliothèque. Cette instance a réuni des membres de la documentation, des chercheurs, ainsi que des chargés de collections invités à contribuer sur une base volontaire. Son objectif : poser les bases d’une politique de vocabulaire raisonnée, en définissant les différents thésaurus existants, les références à utiliser, et la nomenclature à adopter. Selon les documents retrouvés, ce comité se réunissait le premier lundi de chaque mois. S’il est difficile de dire aujourd’hui combien de temps il a perduré, l'existence de ce comité témoigne d’une volonté initiale de coordination des vocabulaires à l’échelle du musée. Cette collaboration des métiers du musée autour de la formation d'un thésaurus ne s'est cependant pas concrétisée par des actions pour unifier les thésaurus existants et ce dialogue officiel n'a pas perduré.

La coexistence de ces thésaurus reflète une division des rôles au sein du musée. Les documentalistes, forts de leur expérience des vocabulaires contrôlés, assurent la cohérence du thésaurus d'Alexandrie et travaillent régulièrement pour l'améliorer et le faire évoluer comme un outil à part entière. Les chargés de collections, alors souvent issus du monde militaire, se montrent plus réservés sur le travail à consacrer à ce type d'outils d'autant plus que Micromusée reste jusqu'au passage à un nouveau logiciel un outil de gestion plus qu’un outil de diffusion, et ne fait pas face aux mêmes enjeux d'accessibilité au public que la bibliothèque.

Ce double système, bien que fonctionnel dans chaque silo, révèle les difficultés à penser un langage documentaire commun au sein d’un même établissement. La question de la cohérence intellectuelle du musée, de la bibliothèque aux collections muséales, commence à se poser, sans qu'une stratégie unifiée ne soit pour autant esquissée.

\subsection{Un tournant documentaire : la création de l’e-médiathèque (2016 – 2020)}

Autour de 2016, un basculement discret s’opère. Le chargement des photographies dans Micromusée est délaissé au profit d’un nouveau dispositif, développé par et pour les documentalistes : l’e-médiathèque. Cette plateforme dédiée aux documents iconographiques et audiovisuels permet un travail plus fin sur l’indexation, entièrement conçu pour ces collections de nature particulière. Depuis 2020, date de mise en ligne de l'e-médiathèque, celle-ci est la référence pour l’indexation des images.

Ce nouveau thésaurus ne repart pas de zéro. Il hérite ses termes de Micromusée, puis les enrichit de nouvelles entrées liées à ses propres collections. Cependant, le dialogue avec la base Micromusée entraîné lors de la migration ne perdure pas, et les deux thésaurus entament dès lors deux évolutions séparées. Depuis au moins la période du Covid, aucun enrichissement réciproque n’a été mis en place, et chacun des thésaurus du musée s’appuie sur sa propre dynamique sans enrichissement volontaire commun.

Ainsi s'installe une coexistence entre trois vocabulaires parallèles qui s'ignorent plus ou moins. Les thésaurus de la documentation (e-médiathèque et bibliothèque) étant utilisés par les mêmes personnes, sont enrichis suite à des processus de recherche similaires et les termes utilisés se ressemblent, cependant rien n'est mise en place pour les unifier ou définir des règles générales au département. Le travail des documentalistes est reconnu et leurs thésaurus sont consultés en cas de doute lors des évolutions sur les thésaurus de description des collections muséales, mais l'inverse est rare. Progressivement, bien que les agents aient conscience de l'existence de ces thésaurus et qu'ils les consultent ponctuellement pour retrouver un terme particulier, aucune réflexion générale n'est menée pour rationaliser leur progression qui devient dépendante des pratiques individuelles et de l'indexation progressive de nouveaux objets.

\subsection{Des architectures hétérogènes}

Sur le plan technique, les divergences entre logiciels rendent toute interopérabilité complexe. Chaque base repose sur une architecture distincte : Alexandrie, en usage à la bibliothèque depuis 1996, a cédé la place en juillet 2025 à Koha, logiciel libre structuré en MySQL, couplé à Clade pour la gestion documentaire. Le musée, de son côté, utilisait Micromusée (v6) depuis 2000, version qui a peu évolué depuis et dont les difficultés d'utilisation et le manque d'ergonomie ont certainement ralenti la réflexion sur le thésaurus qui était intégré. Ce logiciel a été remplacé en juillet 2025 également par Archange, une déclinaison du logiciel S-Museum développée pour les établissements du ministère des Armées. L’e-médiathèque, développée pour les besoins du musée, reste quant à elle inchangée.

\bigskip

Ces outils ont donc été mis en place au fil du temps, dans une logique de réponse aux besoins métiers ou de politique ministérielle. Si des pratiques de structuration communes émergent de leur usage, aucune norme internationale n’a jusqu’à présent été officiellement adoptée pour garantir la cohérence entre les thésaurus. Ceux-ci respectent l'organisation générale recommandée en choisissant des termes descripteurs, leur attribuant des synonymes, les reliant à un ou plusieurs termes génériques et leur attribuant une définition, mais sans plus approfondir les possibilités décrites notamment dans les dernières normes ISO relatives à la gestion de thésaurus. Celles-ci proposent en effet des méthodes pour unifier des thésaurus existants, garantir leur interopérabilité indépendamment des systèmes et langages qui les hébergent et établir des ponts pour leur permettre de communiquer\footnote{\cite{chichereauNormesConceptionGestion2007}}, qui pourraient répondre aux exigences du musée.