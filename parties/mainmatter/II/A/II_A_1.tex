\section{\label{II-A-1}Une construction séparée : 25 ans d'évolution en silo}

L’histoire des thésaurus au sein de l’établissement n’obéit pas à un plan concerté, mais à une sédimentation de pratiques, de logiciels, de métiers. Trois ensembles se partagent aujourd’hui la mémoire documentaire : le thésaurus de la bibliothèque géré dans le logiciel Alexandrie, celui des collections muséales géré dans le logiciel Micromusée, et celui de l’e-médiathèque, géré dans le logiciel de gestion du musée, qui est dédié aux documents iconographiques et audiovisuels. Ces trois corpus de termes, bien que partageant une ambition commune – ordonner, nommer, rendre trouvable – ne sont pas nés du même mouvement ni selon les mêmes logiques.

Gérer un musée, une bibliothèque ou un projet de recherche, c’est toujours se confronter au savoir : à sa dispersion, à sa multiplicité, à son épaisseur. Et cette confrontation impose un choix – celui des termes, de leur agencement, de la structure qui en découle. Ces choix ne sont jamais neutres : ils fondent la manière dont l’institution comprend ses collections, les articule, les rend lisibles. Le MAE, comme d’autres musées, a ressenti très tôt le besoin de maîtriser son langage descriptif, en construisant des vocabulaires contrôlés, d’abord localement, puis de manière plus ambitieuse, mais toujours sans réelle coordination d’ensemble.

\subsection{Les prémices : Alexandrie et Micromusée, une coexistence sans concertation (1996 – années 2010)}

Le premier thésaurus à voir le jour est le thésaurus de la bibliothèque, mis en place dès 1996. Conçu pour accompagner la structuration du catalogue et définir précisément les termes à utiliser pour les autorités\footnote{Forme normalisée et contrôlée des points d’accès relatifs à une ressource dans un catalogue. Dans les catalogues de bibliothèques, les autorités sont principalement relatives aux “auteurs”, aux “sujets” et aux “titres”.}[TODO: note de bas de page], il répond aux exigences classiques du monde documentaire : classification rigoureuse, maîtrise du vocabulaire, liens hiérarchiques. Celui-ci s’inscrit dans une tradition de bibliothéconomie maîtrisée par les professionnels de la documentation.

Parallèlement – [TODO: date exacte à insérer], mais probablement à la fin des années 1990 ou au début des années 2000 – le logiciel Micromusée devient l’outil principal de gestion des collections muséales. Il s’appuie sur une base propre, structurée différemment, dont la logique taxonomique [TODO: à ajouter en lexique ou note de bas de page] s’articule davantage autour des objets matériels que des concepts abstraits. Si des échanges ont eu lieu entre documentalistes et chargés de collection lors du choix des termes, ils semblent avoir été ponctuels et informels. De plus, la différence de structure entre les deux bases rend difficile toute interopérabilité.

Il faut noter tout de même que des comités de pilotage \footnote{[TODO: interview Vincent annexe et fichiers]}  ont été organisés en 1998 entre les chargés de collections et les documentalistes autour du thésaurus, notamment pour l'import sur Micromusée des photos conservées par la bibliothèque. Cette instance a réuni des membres de la documentation, des chercheurs, ainsi que des chargés de collections invités à contribuer sur une base volontaire. Son objectif : poser les bases d’une politique de vocabulaire raisonnée, en définissant les différents thésaurus existants, les références à utiliser, et la nomenclature à adopter. Selon les documents retrouvés, ce comité se réunissait le premier lundi de chaque mois. S’il est difficile de dire aujourd’hui combien de temps il a perduré, son existence témoigne d’une volonté initiale de coordination sémantique à l’échelle du musée. Ceci a entamé un dialogue sur le vocabulaire à utiliser pour la description de ces collections particulières, et une reconnaissance des compétences et de l'expérience des documentalistes en matière de contrôle du vocabulaire.Cette collaboration des métiers du musée autour de la formation d'un thésaurus, ne s'est cependant pas concrétisée par des actions pour d'unification des thésaurus existants et ce dialogue officiel n'a pas perduré.

La coexistence de ces thésaurus reflète une division des rôles au sein du musée. Les documentalistes, forts de leur expérience des vocabulaires contrôlés, assurent la cohérence du thésaurus Alexandrie et travaillent régulièrement pour l'améliorer et le faire évoluer comme un outil à part entière. Les chargés de collections, alors souvent issus du monde militaire, se montrent plus réservés sur le travail à consacrer à ce type d'outils, notamment dans Micromusée, qui reste pour eux un outil de gestion plus qu’un outil de diffusion.

Ce double système, bien que fonctionnel dans chaque silo, révèle les difficultés à penser un langage documentaire commun au sein d’un même établissement. La question de la cohérence intellectuelle du musée, de la bibliothèque à la collection, commence à se poser, sans que autant une stratégie unifiée ne soit esquissée.

\subsection{Un tournant documentaire : la création de l’e-médiathèque (2016 – 2020)}

Autour de 2016, un basculement discret s’opère. Le chargement des photographies dans Micromusée est progressivement délaissé au profit d’un nouveau dispositif, développé par et pour les documentalistes : l’e-médiathèque. Cette plateforme dédiée aux documents iconographiques et audiovisuels permet un travail plus fin sur l’indexation, à l’écart des contraintes propres à Micromusée, et entièrement conçu pour ces collections de nature particulière. En 2020, le thésaurus associé à l’e-médiathèque est publié. Il devient dès lors la référence pour l’indexation des images.

Ce nouveau thésaurus ne repart pas de zéro. Il hérite de Micromusée une partie de ses termes, mais les enrichit de nouvelles entrées liées aux événements, aux personnes, aux notions transversales, avec une structuration plus souple. Cependant, cette migration ne s’accompagne pas d’un dialogue réel avec la base Micromusée, et les deux thésaurus entament dès lors deux évolutions séparées. Depuis au moins la période du Covid, aucun enrichissement réciproque n’a été mis en place. C’est là un point crucial : l’e-médiathèque ne s’appuie plus sur un enrichissement commun, mais sur sa propre dynamique, considérée comme plus fiable par les documentalistes eux-mêmes.

Ainsi s'installe une coexistence entre trois vocabulaires parallèles qui s'ignorent plus ou moins. Les thésaurus de la documentation (e-médiathèque et bibliothèque) étant utilisés par les mêmes personnes, sont enrichis suite à des processus de recherche similaires et les termes utilisés se ressemblent, cependant il n'existe pas de projet pour les unifier ou définir des règles (nomenclature, orthographe, référentiels) générales au département. Le travail des documentalistes est reconnu et leurs deux thésaurus peuvent être consultés en cas de doute lors des évolutions faites sur les thésaurus de description des collections, mais l'inverse est rare. Ainsi, même s'il existe une conscience de l'existence de ces thésaurus et qu'ils sont consultés ponctuellement pour retrouver un terme particulier, aucune réflexion générale n'est menée pour les unifier et leur évolution est dépendante des pratiques individuelles et de l'indexation progressive de nouveaux objets.

\subsection{Une gouvernance éclatée et des architectures hétérogènes}

La diversité des thésaurus s’explique aussi par celle des acteurs impliqués. À la documentation revient la maîtrise du langage, de l'histoire, du lien, de la précision sémantique. Aux chargés de collections, la connaissance de l’objet, de sa matérialité, de son contexte. Aux administrateurs de bases et de logiciels, enfin, l’intégration des structures dans des outils concrets.

Sur le plan technique, les divergences entre logiciels rendent toute interopérabilité complexe. Chaque base repose sur une architecture distincte : Alexandrie, en usage à la bibliothèque depuis 1996, a cédé la place en juillet 2025 à Koha, logiciel libre structuré en MySQL, couplé à Clade pour la gestion documentaire. Le musée, de son côté, utilisait Micromusée (v6) depuis 2000, remplacé à la même date par Archange, déclinaison du logiciel S-Museum développé pour les établissements du ministère des Armées. L’e-médiathèque, fondée en 2016 sur une infrastructure XML, reste quant à elle inchangée.

Ces outils ont été mis en place au fil du temps, dans une logique de réponse aux besoins métiers plus que dans une volonté d’unification. Si des pratiques de structuration communes émergent de leur usage, aucune norme internationale n’a jusqu’à présent été adoptée pour garantir la cohérence entre les thésaurus, bien que les dernières normes ISO relatives à la gestion de thésaurus proposent des méthodes pour unifier des thésaurus existants, garantir leur interopérabilité indépendamment des systèmes et langages qui les hébergent et établir des ponts pour leur permettre de communiquer.\footnote{\cite{chichereauNormesConceptionGestion2007}}.

\bigskip

Tous ces travaux résultant de groupes de travail anciens ou de réflexions ponctuelles suite à des difficultés de description d'un objet en particulier, ils sont très rarement renseignés et ce bref historique provient davantage de la mémoire des agents que de documents renseignant les actions concrètes qui ont été menées. Ces éléments historiques et techniques,  fondamentaux si l’on souhaite comprendre les obstacles et les possibilités d’un rapprochement des vocabulaires, appellent à une réflexion plus vaste sur ce qu’un thésaurus commun signifierait pour le musée : non pas une fusion technique, mais un projet intellectuel de cohérence sémantique entre les différents pans de l’institution.