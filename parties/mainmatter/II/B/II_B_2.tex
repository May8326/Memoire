\section{\label{II-B-2}Des différences d’appréhension qui traduisent un rapport très différent à l’information}


\subsection{Pluralité des priorités dans la description des œuvres}

On l’observe d’emblée dans la manière dont chaque métier, chaque institution, modèle la description de ses collections : rien n’y est jamais uniforme, tout y respire la spécificité. Les musées, soucieux de la matérialité des objets, s’attachent à la singularité, à la trace, à l’histoire, et le vocabulaire qu’ils forgent vise l’exhaustivité, la restitution du détail, la fidélité à la chose vue ou conservée. Les bibliothèques, elles, cherchent la cohérence sémantique, la traçabilité des concepts, la logique du langage documentaire : leur finalité est l’accès, la circulation, la recherche. Les archives, enfin, sont soucieuse de la traçabilité de la chaîne de production, de la preuve, de la continuité administrative — le document y vaut par ce qu’il certifie et non par ce qu’il expose. À travers ces priorités concurrentes, c’est tout l’agencement des thésaurus et des systèmes d’indexation qui se trouve orienté, parfois au point de rendre tout dialogue extrêmement difficile.

\subsection{Enjeux d’interopérabilité et logiques métiers}

Une institution comme le \mae regroupe ces trois ensembles de métiers, qui doivent donc, à partir d'une même base de connaissances et de collections, remplir des missions différentes. L’interopérabilité des vocabulaires et de l'information est la clé d'une cohabitation harmonieuse entre chacun : ce concept fondamental, qui s'est très répandu depuis les années 2000 et dont se sont emparées de nombreuses institutions sans toujours l'embrasser sous toutes ses facettes, est clairement défini par Michèle Hudon dans \citetitle{hudonISO25964Pour2012a}. Selon elle, 
\begin{quote}
	l’interopérabilité est la capacité qu’ont certains agents, services, systèmes et applications d’échanger des données, de l’information et des connaissances en préservant l’intégrité et la pleine signification de celles-ci. L’interopérabilité sémantique concerne plus spécifiquement le langage et le vocabulaire utilisés […] ; elle facilite pour l’usager le repérage et le partage d’information, peu importe le langage et le vocabulaire utilisés. […] Elle passe par le contrôle plus ou moins strict des significations, contrôle associé aux systèmes d’organisation des connaissances (SOC) de types verbal ou symbolique\footcite{hudonISO25964Pour2012a}.
\end{quote}

Or, cette interopérabilité ne se partage pas également entre métiers : elle est vitale pour les documentalistes qui sont directement confrontés aux questions de visibilité de leurs collections sur le web, et dont les pratiques d'indexation incluent par exemple des dérivations de notices de la \ac{bnf} grâce à l'utilisation de formats internationaux comme l'UNIMARC ou l'INTERMARC. Elle se fait contrainte pour les gestionnaires techniques, soumis aux migrations, aux recommandations du ministère de la Culture, et au besoin d’intégrer des réseaux de mise en ligne des collections pour améliorer la visibilité du musée par le public. Pour les chargés de collections davantage confrontés aux objets et moins aux transferts de connaissances et de données avec des institutions externes, elle peut être vécue comme une menace à la singularité des pratiques, ou reléguée au second plan, mobilisée lors d’expositions ou de projets transversaux, mais rarement pensée comme une priorité quotidienne. Le programme C-ADER matérialise cette tension : le thésaurus qu’il produit vient combler un besoin précis, et n’a pas vocation à s’intégrer au vocabulaire institutionnel. Pourtant, ses acquis pourraient nourrir la mémoire commune si la logique documentaire l’emportait sur la logique métier.

\subsection{La persistance des silos institutionnels : musées, bibliothèques, archives}

Bien que l'unité intellectuelle des institutions reste un idéal promu par la littérature scientifique et professionnelle, les écrits sur la coopération entre métiers et institutions sont sans illusion\footcite{gartnerArchivesMuseumsLibraries2019,rossini-paquetBibliothequesMuseesQuellesa,yarrowBibliothequesPubliquesArchives2008a} : musées, bibliothèques, archives demeurent séparés par leurs logiques, leurs normes, leurs architectures. La séparation est physique, intellectuelle, fonctionnelle : locaux distincts, référentiels propres, missions différenciées, publics parfois imperméables les uns aux autres. Les tentatives d’unification se brisent sur la résistance des métiers, sur la diversité des systèmes, sur la difficulté à penser un vocabulaire commun qui ne soit pas synonyme d’appauvrissement. Au \mae, coexister sous un même toit n’a pas suffi à faire mémoire commune ; Gartner et Mouren\footcite{gartnerArchivesMuseumsLibraries2019} parlent de « silos de métadonnées », de la peine à « favoriser la découvrabilité intercommunautaire ». La proximité géographique ne crée pas, en soi, la circulation des savoirs.

\subsection{État de la recherche et pistes théoriques}

La recherche, lucide, propose des pistes mais ne promet pas de solution miracle. Bowker et Star\footcite{bowkerArrangerChosesConsequences2023} invitent à scruter les conséquences de la classification : tout agencement documentaire est le produit de choix politiques, organisationnels, et laisse sa marque dans la mémoire institutionnelle. Bermès\footcite{bermesVersNouveauxCatalogues2016} et Hudon\footcite{hudonISO25964Pour2012a} plaident pour des classifications flexibles, capables d’absorber la pluralité des usages, la diversité des acteurs. Nouvel\footcite{nouvelOutilsDindexationBibliothecaires2022} et Chichereau\footcite{chichereauNormesConceptionGestion2007} rappellent que l’interopérabilité ne peut se penser hors des rapports de pouvoir, de la reconnaissance des expertises métiers. La question de l’unité intellectuelle reste ouverte : les solutions techniques — modélisation, alignement de vocabulaire, normes SKOS, ISO — n’abolissent pas les différences d’appréhension, elles supposent, pour porter, une politique documentaire ambitieuse et une réflexion institutionnelle profonde.

\subsection{Conclusion : reconnaître la diversité avant d'unifier}

Ainsi se dessine une évidence : la fragmentation documentaire au sein des institutions patrimoniales n’est pas un effet secondaire, ni seulement le symptôme d’un problème technique. Elle procède de la pluralité des regards, des valeurs, des missions. Toute solution envisageable exige d’abord la reconnaissance de cette diversité, avant tout projet d’unification documentaire.