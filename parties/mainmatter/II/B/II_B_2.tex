\section{\label{II-B-2}Des différences d’appréhension qui traduisent un rapport très différent à l’information}

\subsection*{Problématisation générale}

La fragmentation des pratiques documentaires au sein des institutions patrimoniales — musées, bibliothèques, archives — est un objet de réflexion récurrent dans la littérature contemporaine sur la gouvernance de l'information. Elle ne relève pas d’un simple défaut de coordination, mais traduit des rapports fondamentalement divergents à la nature de l’information, à sa matérialité, à sa finalité et à ses usages. Cette diversité des appréhensions, loin d’être accidentelle, est le produit d’histoires institutionnelles, de cultures professionnelles et de logiques métiers distinctes\footcite{bowkerArrangerChosesConsequences2023,gartnerArchivesMuseumsLibraries2019,rizzaDocumentAuCoeur2014}.

\subsection*{1. Pluralité des priorités dans la description des œuvres}

Dans la littérature sur les vocabulaires contrôlés et la gouvernance documentaire\footcite{hudonISO25964Pour2012a,nouvelOutilsDindexationBibliothecaires2022,chichereauNormesConceptionGestion2007}, il est souvent rappelé que la description des œuvres ne répond pas à une logique unique :
\begin{itemize}
	\item Les musées privilégient la matérialité, l’histoire singulière de l’objet, son contexte d’usage et sa dimension patrimoniale. Le vocabulaire est alors pensé comme un outil au service de la spécificité et de l’exhaustivité.
	\item Les bibliothèques mettent l’accent sur la structuration sémantique, la traçabilité des concepts, la cohérence du langage documentaire. La finalité est l’accès, la recherche, la circulation de l’information.
	\item Les archives accordent une importance primordiale à la chaîne de production, à la preuve, à la traçabilité du document, et à la continuité institutionnelle.
\end{itemize}
Cette diversité engendre des priorités concurrentes, qui se manifestent dans la conception même des thésaurus et des systèmes d’indexation. Comme le rappelle Maryse Rizza\footcite{rizzaDocumentAuCoeur2014}, le document prend sens différemment selon qu’il est considéré comme œuvre, publication ou preuve administrative.

\subsection*{2. Enjeux d’interopérabilité et logiques métiers}

L’interopérabilité, souvent invoquée comme horizon commun des systèmes d’information patrimoniaux\footcite{hudonISO25964Pour2012a,bermesVersNouveauxCatalogues2016}, apparaît en réalité comme un besoin inégalement partagé :
\begin{itemize}
	\item Pour les documentalistes, elle est vitale : elle garantit la visibilité du fonds, la circulation des savoirs, la légitimité scientifique de l’institution.
	\item Pour les gestionnaires techniques, elle est une contrainte opérationnelle, imposée par les migrations logicielles ou l’intégration à des réseaux ministériels.
	\item Pour les chargés de collections, l’interopérabilité peut être perçue comme une menace pour la singularité des pratiques, ou comme un enjeu secondaire, mobilisé surtout lors d’expositions ou de projets transversaux.
\end{itemize}
L’exemple du programme C-ADER illustre cette tension : le thésaurus développé pour la recherche sur les dégradations des matériaux répond à des besoins spécifiques, sans exigence d’intégration documentaire avec le musée, même si ses résultats pourraient enrichir le vocabulaire institutionnel.

\subsection*{3. La persistance des silos institutionnels : musées, bibliothèques, archives}

La littérature sur la coopération interinstitutionnelle\footcite{gartnerArchivesMuseumsLibraries2019,rossini-paquetBibliothequesMuseesQuellesa,yarrowBibliothequesPubliquesArchives2008a} souligne que, malgré leur proximité intellectuelle et leur mission commune de transmission du patrimoine, musées, bibliothèques et archives demeurent séparés par des logiques professionnelles et des architectures techniques distinctes.
\begin{itemize}
	\item Cette séparation est à la fois physique (locaux séparés), intellectuelle (référentiels, normes et standards propres) et fonctionnelle (missions spécifiques, publics différents).
	\item Les tentatives d’unification documentaire se heurtent à la résistance des métiers, à la diversité des systèmes et à la difficulté de construire des vocabulaires communs.
\end{itemize}
Ce constat est d’autant plus frappant que de nombreuses institutions, comme le \mae, rassemblent sous un même toit musée, bibliothèque et archives, mais peinent à articuler une mémoire intellectuelle unifiée. Gartner \& Mouren\footcite{gartnerArchivesMuseumsLibraries2019} parlent de « silos de métadonnées » et de la difficulté à « favoriser la découvrabilité intercommunautaire » des ressources patrimoniales.

\subsection*{4. État de la recherche et pistes théoriques}

La littérature récente propose plusieurs pistes pour penser cette fragmentation :
\begin{itemize}
	\item Bowker \& Star\footcite{bowkerArrangerChosesConsequences2023} invitent à « explorer les conséquences de la classification », soulignant que toute structuration documentaire est le produit de choix politiques et organisationnels, qui laissent des traces dans la mémoire institutionnelle.
	\item Bermès\footcite{bermesVersNouveauxCatalogues2016} et Hudon\footcite{hudonISO25964Pour2012a} insistent sur la nécessité de concevoir des classifications flexibles, capables d’intégrer la diversité des usages et la pluralité des acteurs.
	\item Nouvel\footcite{nouvelOutilsDindexationBibliothecaires2022} et Chichereau\footcite{chichereauNormesConceptionGestion2007} rappellent que l’interopérabilité ne peut être pensée sans une réflexion sur les rapports de pouvoir et la reconnaissance des expertises métiers.
\end{itemize}

Enfin, la question de l’unité intellectuelle des institutions patrimoniales reste ouverte : les solutions techniques (modélisation, alignement de vocabulaire, normes SKOS/ISO) ne suffisent pas à abolir les différences d’appréhension — elles doivent s’accompagner d’une réflexion institutionnelle et d’une politique documentaire ambitieuse.

\subsection*{Transition}

Il apparaît ainsi que la fragmentation des pratiques métiers dans la gestion de l’information patrimoniale ne relève pas seulement d’un problème technique, mais d’une pluralité de rapports à l’information, à la valeur patrimoniale, et à la mission de transmission. Les solutions envisagées dans la littérature appellent à une prise de conscience de cette diversité, préalable à toute tentative d’unification documentaire.