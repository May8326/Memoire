\section{\label{II-B-2}Des différences d’appréhension qui traduisent un rapport très différent à l’information}

Toujours pour décrire des œuvres, mais d’une manière différente : priorité différente accordée au contexte de l’œuvre, matérialité, usage, etc... 

Des besoins et des réalités différentes en matière d’interopérabilité (ex. De C-ADER, le thésaurus d’un programme de recherche, intéressant pour les missions du musée mais qui a une vie totalement à part et zéro besoin ni exigence de ce côté-là). 

Dans l’état actuel de la recherche, peu de solutions proposées / réflexions sur cet aspect propre du problème, même si chaque branche explore bien de son côté la question des vocabulaires et de l’interopérabilité. Pourtant, il n’est pas rare d’avoir musée et bibliothèque au même endroit, unité intellectuelle dans les missions de l’institution. 

Relation musées - bibliothèques - archives : des mondes séparés physiquement, intellectuellement, malgré similitude et intégration dans une même mission.