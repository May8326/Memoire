\section{\label{II-B-1}Une montée du besoin de réflexion sur le vocabulaire du musée différenciée selon les métiers }


Prise de conscience d'abord par les documentalistes, qui ont déjà des habitudes de construction de thésaurus et pour qui la question de l'interopérabilité se pose plus au quotidien (projets d'intégration au sudoc, SIGB commun, recherche de visibilité pour attirer les lecteurs/chercheurs, volonté de mettre en avant un fonds particulier) 

Gestionnaire de base de données : de par son métier, prise de conscience directe de la nécessité d'harmoniser le contenu des notices, d'avoir un référentiel pour indexer. Chargée d'aller elle-même modifier dans la BDD quand changement de terminologie 

Un sujet qui a été mis sur le tapis grâce aux migrations (changement de SIGB pour la bibliothèque, changement de logiciel de gestion des collections pour le musée) 

Chargés de collection, une relation ambigue : Une sensibilité qui dépend principalement de l’usage quotidien qui en est fait et des besoins ponctuels. 

Une sensibilité qui dépend aussi d’une conception de son métier : différents choix qui sont faits pour le thésaurus des domaines en particulier. 