\section{\label{II-B-1}Cartographie des métiers et émergence d’une conscience documentaire différenciée au \mae}

Avant d’analyser chaque rôle, il importe de rappeler que la gouvernance documentaire au \mae est le produit d’une histoire institutionnelle complexe : la structuration progressive des métiers, la professionnalisation récente des équipes, et la cohabitation d’expertises variées (documentation, collections, informatique) ont forgé un modèle où les sensibilités et pratiques divergent, sans être hermétiques.

\subsection{Les documentalistes : pionniers, vigies et promoteurs de l’interopérabilité}

La question des vocabulaires contrôlés au sein du \mae ne s’est pas imposée d’emblée à l’ensemble de l’institution : elle est née tout d'abord dans le département \enquote{documentation} du musée, chargé de la gestion de la bibliothèque, de l'audiovisuel et des archives. Historiquement, ce sont donc les documentalistes qui furent les premiers artisans d'un vocabulaire contrôlé au \mae, et les premiers sensibilisés à cette question.

En effet, dès la fin des années 1990, la création du thésaurus de la bibliothèque, alors géré dans le logiciel \gls{alexandrie}, s’inscrit dans une tradition solidement ancrée de bibliothéconomie. Les documentalistes, formés à la rigueur des normes AFNOR et ISO, et nourris de l’expérience des grands réseaux nationaux tels que le SUDOC ou la \ac{bnf}, ont d’emblée perçu l’enjeu stratégique de l’interopérabilité : il s’agissait, au-delà du seul catalogage, d’assurer la visibilité du fonds documentaire, de construire des listes d’autorité fiables, de garantir la pérennité et la communicabilité des savoirs, comme l'a présenté Blandine Nouvel dans un atelier \enquote{Thésaurus appliqués} pour le projet Bibracte, ville ouverte\footcite{nouvelOutilsDindexationBibliothecaires2022}. Les documentalistes en France ont en effet toujours été des moteurs des projets de normalisation de la gestion de l'information -- dans les thésaurus en particulier -- et ont réclamé et applaudi les travaux internationaux menés sur le sujet\footnote{Il convient en effet de citer par exemple l'article \citetitle{chichereauNormesConceptionGestion2007} publié dans Documentaliste-Sciences de l'Information, qui insiste sur l'intérêt de la nouvelle norme ISO en cours de travail et fait un panorama des existantes, avec leurs qualités et défauts.}.

Au \mae, ce rôle éminemment technique et intellectuel s’est affirmé lors des premiers comités de pilotage destinés à harmoniser les vocabulaires entre la documentation et les collections muséales — instances où le dialogue entre les métiers permit d’esquisser les premiers jalons d’une gouvernance documentaire véritable\footnote{Voir historique des thésaurus du musée, section \textit{\hyperref[II-A-1]{\nameref{II-A-1}}}}, bien que ce dialogue n'ait pas perduré. La pratique quotidienne du métier, l’attention portée aux métadonnées, la maîtrise des principes de normalisation et d’interopérabilité ont en effet fait des documentalistes les membres naturels et incontournables de ces groupes de travail, contrairement par exemple aux chargés de collections dont la présence n'était que facultative.

L’année 2025 marque un tournant : la migration vers \gls{koha} et la plateforme \gls{clade}, orchestrée par le \minarm, met en lumière la fonction d’interface que les documentalistes exercent entre les exigences institutionnelles et la réalité du terrain. Ils négocient l’intégration du thésaurus, soulignent les erreurs d’import et défendent la nécessité d’une structuration fine des termes pour la recherche — combat quelquefois solitaire face aux exigences ministérielles et aux impératifs de la technique\footnote{Voir les difficultés rencontrées lors de la migration vers la plateforme Koha, mentionnées en section \textit{\hyperref[I-B-2]{\nameref{I-B-2}}}}. Cette période voit également l’organisation d’ateliers de sensibilisation à la gouvernance de l'information, menés sous l’impulsion de la responsable du \ac{drd} : celle-ci rédige des notes et promeut l’acculturation du \ac{dsc} à la notion de thésaurus partagé, illustrant ainsi la dimension pédagogique et fédératrice du métier.

Cette prééminence des documentalistes s’explique donc par la spécificité de leur formation, leur acculturation à l’interopérabilité et la nécessité institutionnelle de rendre visible le fonds auprès d’un public hétérogène. Mais si leur action a posé les fondements d’une gouvernance documentaire digne de ce nom, elle est longtemps demeurée cantonnée au périmètre de la documentation, faute de pouvoir imposer une politique véritablement transversale à l’ensemble du musée. Ainsi, les documentalistes demeurent — encore aujourd’hui — les pionniers et les gardiens vigilants d’un chantier intellectuel qui attend d’être pleinement partagé.

\subsection{La gestionnaire de \gls{bdd}, pivot documentaire et révélateur des limites institutionnelles}

La figure du gestionnaire de \gls{bdd} s’est imposée au \mae comme l’un des pivots discrets mais essentiels de la cohérence documentaire, et ce bien avant que la notion de gouvernance de l’information ne s’invite dans les débats institutionnels. L’avènement du logiciel \gls{micromusee} au tournant des années 2000, puis son remplacement par \gls{archange} en 2025, ont doté le musée d’outils de gestion qui lui ont permis de gérer le vocabulaire de manière unifiée, et qui sont devenus le terrain d’un travail quotidien pour la construction et la transmission des référentiels au sein du \ac{dsc}.

En ce qui concerne la gouvernance de l'information, le gestionnaire de \gls{bdd} exerce une mission de veille et de correction, que la migration vers \gls{archange} a rendu visible aux yeux des autres métiers. Il ou elle intervient en modifiant les terminologies, traquant les incohérences issues du logiciel, et fédérant au niveau du musée les initiatives de chacun. Détenteur des droits de modification et de création, ce technicien de l’information devient en effet l’interlocuteur privilégié des chargés de collections, l'arbitre des enrichissements et des corrections, notamment lors de l’introduction de nouveaux matériaux ou domaines.

La grande migration de \gls{micromusee} vers \gls{archange} pour les collections muséales a joué un rôle révélateur pour le gestionnaire de \gls{bdd}. Ce chantier, en effet, a mis en lumière à la fois l’expertise technique requise et les limites de l’outillage institutionnel : \gls{micromusee} v6, dont il a été fait mention plus haut, illustre à lui seul le fossé qui sépare les besoins intellectuels du musée et les solutions techniques qui lui ont été offertes jusqu'ici. Le travail laborieux réalisé lors de la migration -- analyses des exports, détection des doublons et correction manuelle des incohérences -- a rendu manifeste la nécessité de revisiter, de rationaliser et d'harmoniser les thésaurus existants au sein du musée.

La responsable \gls{bdd}, aidée d'un expert en informatique, a ainsi dû s'atteler à l’examen minutieux des exports issus de \gls{micromusee} : plus de 16\,000 termes, dont l’analyse a révélé la présence de doublons persistants dans certaines tables, de termes privés de rattachement hiérarchique ou mal positionnés\footnote{Cela est le cas par exemple de la liste d'autorités des noms propres de la base \gls{micromusee}/\gls{archange}, qui a révélé après examen que de nombreuses personnes physiques avaient été positionnées sous personne morale, et inversement. Pour corriger ces incohérences, la gestionnaire a entamé un travail sur le long terme de corrections par petits lots des termes rencontrés au fil des recherches.}. La gestionnaire de \gls{bdd} est ainsi devenue le premier acteur du nettoyage et de la réorganisation des termes, dans un travail laborieux, long, mais nécessaire, par petites séances de correction.

Son rôle consiste également à gérer d'éventuels conflits ou besoins dans les terminologies : par exemple par l'ajout ou l'organisation des noms de matériaux utilisés pour décrire les collections. Ici, le gestionnaire de \gls{bdd} a organisé des réunions en collaboration avec des chargés de collection, pour arbitrer entre synonymes concurrents, intégrer les termes retenus, les documenter, dans un tableur permettant de visualiser l'ensemble du thésaurus. Ce travail de révision n’a pas seulement consisté à corriger des erreurs : il a nécessité une restructuration des branches du thésaurus, la suppression de doublons ou de termes inutiles, et un travail considérable de recherche pour bien positionner les termes et mettre le vocabulaire du \mae en accord avec les recommandations nationales\footnote{Il s'agit notamment des thésaurus de Joconde mis à disposition par le ministère de la Culture, qui seront développés plus bas.}.

Les projets de migration et les outils implémentés au musée jouent un rôle primordial dans la sensibilisation du métier à la question du contrôle du vocabulaire : par exemple, l’inadéquation de \gls{micromusee} a contraint à recourir à des opérations de migration fastidieuses -- conversions et traitement en CSV, visualisations dans Gephi, adaptation manuelle des formats -- et à inspecter plus en détail la composition des thésaurus présents. Le passage vers \gls{archange}, quant à lui, et les nombreuses fonctionnalités qu'offre la plateforme pour gérer les vocabulaires, est à l'origine de nouveaux projets d'organisation de celui-ci\footnote{La plateforme offre par exemple la possibilité de renseigner les liens familiaux entre différentes personnes de la base : cette fonctionnalité a été relevée comme pouvant être très intéressante pour mettre en valeur les connaissances historiques du musée sur l'histoire et les généalogies d'aviateurs ou de constructeurs d'avions. L'entrée de ces nouvelles informations -- qui se ferait donc uniquement dans la base de gestion des collections, sur le référentiel des personnes, l'éloignerait d'une structure de thésaurus ou de liste de vocabulaire pour le rapprocher d'une ontologie, solution lourde à implémenter mais qui s'est révélée tout au long du stage comme une solution plus complète pour mettre en valeur l'ensemble des connaissances scientifiques du \mae.}.

L’analyse révèle que la prise de conscience du gestionnaire de \gls{bdd} est directe, pragmatique, et fondée sur l’expérience du terrain : la stabilité des référentiels et la cohérence des notices ne sont pas des enjeux théoriques, mais la condition d’une gestion quotidienne efficace. Pourtant, ce rôle fondamental reste le plus souvent cantonné à la résolution de problèmes ponctuels, sans pouvoir s’étendre à une réflexion globale sur la gouvernance documentaire, laquelle ne saurait être menée sans l’implication de l’ensemble des métiers concernés. Les grandes migrations informatiques quant à elles, loin d’être de simples opérations techniques, jouent le rôle de révélateur des failles du système et amorcent des dynamiques de rationalisation, en exposant à tous -- techniciens comme conservateurs -- la nécessité d’une réforme des pratiques et d’une harmonisation des vocabulaires.

\subsection{Les chargés de collections : diversité des pratiques, entre nomenclature fine et catégories souples}

La question du vocabulaire contrôlé demeure périphérique dans le quotidien des chargés de collections. Loin d’être anecdotique cette situation interroge : pourquoi, alors que l’exigence de normalisation s’impose de toutes parts, la structuration lexicale peine-t-elle à s’ériger en préoccupation centrale chez ceux-là mêmes qui sont les dépositaires de la mémoire matérielle ?

\medskip

Tout d'abord, il faut reconnaître que la sensibilité des chargés de collections à l’égard du \gls{thesaurus} contrôlé procède avant tout d'un usage ponctuel : elle s’éveille à l’occasion d’une exposition, de l'enrichissement d'un fonds, ou lorsque surgit une difficulté d’indexation qui résiste aux outils habituels. Formés dans des cursus techniques ou muséaux, ces professionnels cultivent un rapport particulier à l’objet : ils privilégient l’expérience de la matérialité, la restitution de l’histoire singulière, le dialogue avec le contexte d’origine. Leur premier instinct sera, dans toute description, de préserver l'épaisseur du réel pour transmettre toute la richesse scientifique, historique et sociale de l'objet. Ces particularités du rapport à l'information selon les métiers n'ont rien de nouveau, et elles ont déjà été relevées par des chercheurs comme Maryse Rizza : par exemple, l'endroit premier où se retrouve le \gls{thesaurus} pour un chargé de collection est l'inventaire -- et comme la chercheuse l'a exprimé dans \citetitle{rizzaDocumentAuCoeur2014}, \enquote{cet inventaire, base de la production documentaire, sera utilisé de manière différente selon le rôle et la place de l’acteur dans l’organisation muséale. Le conservateur, par exemple, privilégiera la fonction scientifique du document d’inventaire pour produire son analyse et rédiger des notices et/ou des commentaires qui ont pour vocation d’enrichir la connaissance historique du patrimoine muséographique}, tandis que pour le régisseur des collections, c’est la \enquote{fonction technique du document d’inventaire qui primera sur ses usages\footcite{rizzaDocumentAuCoeur2014}}.

\medskip

On ne saurait cependant généraliser cette tendance à l'ensemble des chargés de collection : au sein du corps de métier, différentes approches existent selon les personnalités et les champs d'études : au musée par exemple, les héritiers d’une tradition anthropologique ou technique, privilégieront une nomenclature d’une extrême finesse, où chaque objet, chaque matériau appelle une granularité lexicale adaptée à son contexte intellectuel et historique. C'est dans leur branche que se fait l'essentiel de l’enrichissement du \gls{thesaurus} des domaines — utilisé pour la description des collections du musée — devient le lieu d’une quête de précision, d’une volonté de restituer la complexité du monde technique sans céder aux simplifications. À l’inverse, d’autres optent pour une approche plus souple, mobilisant des catégories génériques et des mots-clés fonctionnels, jouant sur la plasticité des termes pour affiner le classement sans enfermer l’objet dans une taxonomie rigide : ceci est visible dans les branches plus techniques du même \gls{thesaurus}\footnote{cf. \textit{\hyperref[fig:model_domaines]{\nameref{fig:model_domaines}}}}. Ces évolutions de la structuration du \gls{thesaurus} des domaines, qui oscille entre logique technique et logique anthropologique, illustrent ces arbitrages permanents : faut-il privilégier la granularité ou l’opérabilité ? La fidélité au terrain ou l’harmonisation documentaire ?

La prolifération des vocabulaires au sein du \acf{mae}, loin d’être le simple effet d’une accumulation documentaire non maîtrisée, révèle en profondeur la difficulté à penser collectivement la mémoire d’une institution technique. Fragmentation des outils, silos de métiers, absence de coordination : autant de symptômes d’un malaise documentaire qui invisibilise les collections et entrave la transmission du savoir. Une volonté d'organisation existe tout de même derrière cette apparente dispersion.

\bigskip

Ce constat invite à dépasser le seul diagnostic technique pour interroger la part humaine des pratiques d’indexation : car la structuration de l’information, au musée, ne se joue pas uniquement dans les logiciels ou les schémas de \gls{thesaurus}, mais dans la diversité des usages, des sensibilités et des compétences qui traversent les métiers. Dès lors, il est nécessaire de comprendre comment chaque acteur, selon sa formation, son rapport à l’objet ou à la documentation, s’approprie — ou délaisse — les vocabulaires contrôlés : c’est cette cartographie des rôles, des résistances et des prises de conscience différenciées que la prochaine section se propose d’explorer, pour mieux saisir les conditions d’une gouvernance informationnelle renouvelée.