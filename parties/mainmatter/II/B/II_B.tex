\chapter[Diversité des métiers et fragmentation documentaire]{\label{II-B}Logiques professionnelles et gouvernance documentaire : la fragmentation des vocabulaires au prisme des métiers du \mae}

\lettrine{L}a fragmentation des vocabulaires contrôlés au sein du \maelong ne s'explique pas seulement par la diversité des outils ou des collections. Elle traduit une histoire institutionnelle où chaque métier — documentaliste, gestionnaire de base de données, chargé de collections — a développé sa propre sensibilité à la question du vocabulaire. Cette pluralité, loin d'être accidentelle, témoigne d'une répartition différenciée des rôles, des responsabilités et des rapports à l'information, qui conditionnent aujourd'hui la possibilité d'une gouvernance documentaire commune.

\section{\label{II-B-1}Une montée du besoin de réflexion sur le vocabulaire du musée différenciée selon les métiers }


Prise de conscience d'abord par les documentalistes, qui ont déjà des habitudes de construction de thésaurus et pour qui la question de l'interopérabilité se pose plus au quotidien (projets d'intégration au sudoc, SIGB commun, recherche de visibilité pour attirer les lecteurs/chercheurs, volonté de mettre en avant un fonds particulier) 

Gestionnaire de base de données : de par son métier, prise de conscience directe de la nécessité d'harmoniser le contenu des notices, d'avoir un référentiel pour indexer. Chargée d'aller elle-même modifier dans la BDD quand changement de terminologie 

Un sujet qui a été mis sur le tapis grâce aux migrations (changement de SIGB pour la bibliothèque, changement de logiciel de gestion des collections pour le musée) 

Chargés de collection, une relation ambigue : Une sensibilité qui dépend principalement de l’usage quotidien qui en est fait et des besoins ponctuels. 

Une sensibilité qui dépend aussi d’une conception de son métier : différents choix qui sont faits pour le thésaurus des domaines en particulier. 
\section{\label{II-B-2}Des différences d’appréhension qui traduisent un rapport très différent à l’information}

\subsection*{Problématisation générale}

La fragmentation des pratiques documentaires au sein des institutions patrimoniales — musées, bibliothèques, archives — est un objet de réflexion récurrent dans la littérature contemporaine sur la gouvernance de l'information. Elle ne relève pas d’un simple défaut de coordination, mais traduit des rapports fondamentalement divergents à la nature de l’information, à sa matérialité, à sa finalité et à ses usages. Cette diversité des appréhensions, loin d’être accidentelle, est le produit d’histoires institutionnelles, de cultures professionnelles et de logiques métiers distinctes\footcite{bowkerArrangerChosesConsequences2023,gartnerArchivesMuseumsLibraries2019,rizzaDocumentAuCoeur2014}.

\subsection*{1. Pluralité des priorités dans la description des œuvres}

Dans la littérature sur les vocabulaires contrôlés et la gouvernance documentaire\footcite{hudonISO25964Pour2012a,nouvelOutilsDindexationBibliothecaires2022,chichereauNormesConceptionGestion2007}, il est souvent rappelé que la description des œuvres ne répond pas à une logique unique :
\begin{itemize}
	\item Les musées privilégient la matérialité, l’histoire singulière de l’objet, son contexte d’usage et sa dimension patrimoniale. Le vocabulaire est alors pensé comme un outil au service de la spécificité et de l’exhaustivité.
	\item Les bibliothèques mettent l’accent sur la structuration sémantique, la traçabilité des concepts, la cohérence du langage documentaire. La finalité est l’accès, la recherche, la circulation de l’information.
	\item Les archives accordent une importance primordiale à la chaîne de production, à la preuve, à la traçabilité du document, et à la continuité institutionnelle.
\end{itemize}
Cette diversité engendre des priorités concurrentes, qui se manifestent dans la conception même des thésaurus et des systèmes d’indexation. Comme le rappelle Maryse Rizza\footcite{rizzaDocumentAuCoeur2014}, le document prend sens différemment selon qu’il est considéré comme œuvre, publication ou preuve administrative.

\subsection*{2. Enjeux d’interopérabilité et logiques métiers}

L’interopérabilité, souvent invoquée comme horizon commun des systèmes d’information patrimoniaux\footcite{hudonISO25964Pour2012a,bermesVersNouveauxCatalogues2016}, apparaît en réalité comme un besoin inégalement partagé :
\begin{itemize}
	\item Pour les documentalistes, elle est vitale : elle garantit la visibilité du fonds, la circulation des savoirs, la légitimité scientifique de l’institution.
	\item Pour les gestionnaires techniques, elle est une contrainte opérationnelle, imposée par les migrations logicielles ou l’intégration à des réseaux ministériels.
	\item Pour les chargés de collections, l’interopérabilité peut être perçue comme une menace pour la singularité des pratiques, ou comme un enjeu secondaire, mobilisé surtout lors d’expositions ou de projets transversaux.
\end{itemize}
L’exemple du programme C-ADER illustre cette tension : le thésaurus développé pour la recherche sur les dégradations des matériaux répond à des besoins spécifiques, sans exigence d’intégration documentaire avec le musée, même si ses résultats pourraient enrichir le vocabulaire institutionnel.

\subsection*{3. La persistance des silos institutionnels : musées, bibliothèques, archives}

La littérature sur la coopération interinstitutionnelle\footcite{gartnerArchivesMuseumsLibraries2019,rossini-paquetBibliothequesMuseesQuellesa,yarrowBibliothequesPubliquesArchives2008a} souligne que, malgré leur proximité intellectuelle et leur mission commune de transmission du patrimoine, musées, bibliothèques et archives demeurent séparés par des logiques professionnelles et des architectures techniques distinctes.
\begin{itemize}
	\item Cette séparation est à la fois physique (locaux séparés), intellectuelle (référentiels, normes et standards propres) et fonctionnelle (missions spécifiques, publics différents).
	\item Les tentatives d’unification documentaire se heurtent à la résistance des métiers, à la diversité des systèmes et à la difficulté de construire des vocabulaires communs.
\end{itemize}
Ce constat est d’autant plus frappant que de nombreuses institutions, comme le \mae, rassemblent sous un même toit musée, bibliothèque et archives, mais peinent à articuler une mémoire intellectuelle unifiée. Gartner \& Mouren\footcite{gartnerArchivesMuseumsLibraries2019} parlent de « silos de métadonnées » et de la difficulté à « favoriser la découvrabilité intercommunautaire » des ressources patrimoniales.

\subsection*{4. État de la recherche et pistes théoriques}

La littérature récente propose plusieurs pistes pour penser cette fragmentation :
\begin{itemize}
	\item Bowker \& Star\footcite{bowkerArrangerChosesConsequences2023} invitent à « explorer les conséquences de la classification », soulignant que toute structuration documentaire est le produit de choix politiques et organisationnels, qui laissent des traces dans la mémoire institutionnelle.
	\item Bermès\footcite{bermesVersNouveauxCatalogues2016} et Hudon\footcite{hudonISO25964Pour2012a} insistent sur la nécessité de concevoir des classifications flexibles, capables d’intégrer la diversité des usages et la pluralité des acteurs.
	\item Nouvel\footcite{nouvelOutilsDindexationBibliothecaires2022} et Chichereau\footcite{chichereauNormesConceptionGestion2007} rappellent que l’interopérabilité ne peut être pensée sans une réflexion sur les rapports de pouvoir et la reconnaissance des expertises métiers.
\end{itemize}

Enfin, la question de l’unité intellectuelle des institutions patrimoniales reste ouverte : les solutions techniques (modélisation, alignement de vocabulaire, normes SKOS/ISO) ne suffisent pas à abolir les différences d’appréhension — elles doivent s’accompagner d’une réflexion institutionnelle et d’une politique documentaire ambitieuse.

\subsection*{Transition}

Il apparaît ainsi que la fragmentation des pratiques métiers dans la gestion de l’information patrimoniale ne relève pas seulement d’un problème technique, mais d’une pluralité de rapports à l’information, à la valeur patrimoniale, et à la mission de transmission. Les solutions envisagées dans la littérature appellent à une prise de conscience de cette diversité, préalable à toute tentative d’unification documentaire.

\bigskip
\bigskip
\bigskip

\lettrine{L}{a} diversité des métiers et des rapports à l'information au \mae engendre une fragmentation des vocabulaires et des pratiques documentaires, qui complique la mise en œuvre d'une gouvernance commune. Comprendre ces clivages professionnels, articuler les priorités divergentes et mobiliser les apports de la littérature académique constitue un préalable incontournable à toute politique d'unification documentaire. La réflexion sur l'interopérabilité et la mémoire de l'organisation peut s'appuyer sur de nombreux projets et recherches déjà établis dans cette direction, bien qu'aucune solution ne semble faire aujourd'hui l'unanimité dans le monde professionnel et académique. Pourtant, ces tensions ne sauraient se résoudre par la seule technique : elles s’enracinent dans la manière dont le musée conçoit et gouverne son information, et s’étendent jusqu’à la frontière mouvante des archives numériques du musée. C’est là, dans la gestion de ces fonds hybrides, que se cristallisent les difficultés : absence de solutions pérennes, légitimité encore fragile du métier d’archiviste, et nécessité d’articuler mémoire, technique et pratiques professionnelles dans une institution où chaque métier impose sa propre vision de la pérennité documentaire. La partie suivante se propose d’explorer ce territoire où le rapport du \mae à l’information, ses choix de gestion et ses logiques métiers rencontrent de plein fouet les défis contemporains de l’archivage numérique.

[TODO : ajouter des exemples au II/B/2 de systèmes mis en place en institution patrimoniale dans ce sens]