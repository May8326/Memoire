\section{\label{II-C-2}Vers une gouvernance des archives numériques au \mae : expérimentations, structuration et montée en compétences}


\begin{myquote}
	{Ne pas envisager l’archivage de documents numériques dès leur création, c’est prendre le risque de les perdre définitivement.}{ministeredelacultureArchivesElectroniquesDans2020}
\end{myquote}

\subsection{Quelques leviers}


Loin de se réduire à une succession d’obstacles, la situation des archives numériques au musées est aussi un terrain privilégié pour tenter une réconciliation des dynamiques muséales et archivistiques. Les recommandations nationales – garantie de l'intégrité du fichier et de ses métadonnées, distinction des états d’archives, traçabilité, bordereaux d’élimination – sont peu à peu reconnues, d'abord par le \ac{drd}, sensibilisé par ses missions aux enjeux de gestion de l'information, puis par les agents du \ac{dsc} qui y sont confrontés au quotidien.



Certes, la question du versement définitif demeure ouverte : aucun projet d'intégration immédiate à un \gls{sae} ne paraît envisageable, mais la problématique est reconnue, et il faut reconnaître que la priorité est donnée pour l'instant aux chantiers logiciels menés par le \minarm (\gls{koha}-\gls{clade} et \gls{archange}), reléguant la question de l’archivage à une temporalité indéfinie. La sensibilisation des agents se heurte à la dispersion des responsabilités et la fonction archivistique demeure fragile, mais des quelques avancées pour poser les bases de projets ultérieurs plus conséquents ont été proposées au \ac{dsc}, qui pourrait alors, après une première expérimentation, argumenter pour un chantier commun à l'ensemble du musée.


La première, et l'une des missions secondaires de ce stage, a consisté en la production de moyens de sensibilisation des agents du \ac{dsc} aux enjeux de gouvernance de leurs archives numériques : ceux-ci auront consisté principalement en 
\begin{itemize}
	\item la réalisation de formations de sensibilisation à l’adresse des agents,
	\item la rédaction d’une Foire aux Questions sur la gestion quotidienne et les grands principes d'archivistique[TODO : ajout annexe faq caviardée],
	\item la production de fiches de bonnes pratiques.
\end{itemize}

La seconde, liée à la précédente, est le recueillement des ressentis et des pratiques déjà en vigueur dans le service, afin de proposer des solutions réalistes et adaptées aux exigences métier : cette démarche a abouti à la proposition de normes de nommages de fichiers et de dossiers, et de structuration de l'arborescence générales au service et à la création d'un document partagé de mise en commun du vocabulaire et des abréviations utilisées.


Il s’agit ici d’insuffler une culture commune de la gestion documentaire : nommage, tri, organisation, bordereaux d’élimination sont pensés comme autant de jalons vers une maîtrise accrue du cycle de vie des fichiers. Face à l’impossibilité actuelle d’un archivage définitif, l'engagement d'un professionnel dédié au tri du serveur, à la validation des éliminations et à la mise en conformité des procédures est souhaitable ; dans l'attente d'une solution institutionnelle, une réflexion s’est engagée sur la création d’un dossier réservé à l'archivage des documents arrivés à leur fin d'utilité administrative.


Pour préparer le terrain d'une gestion robuste de ses archives, ces initiatives, portées par le \ac{dsc} illustrent la possibilité d’entamer une dynamique de modernisation fondée sur la formation, l’accompagnement et l’engagement des agents, même en l’absence de dispositifs techniques pleinement aboutis.

\bigskip
\bigskip
\bigskip

La crise de l’archivage numérique en musée ne se résume pas à une accumulation de défauts : elle révèle également la capacité de l’institution à inventer des solutions pour organiser sa gouvernance en conformité avec les recommandations en vigueur et former ses agents. Là encore, ce sont les métiers de la documentation, plus sensibilisés aux enjeux de la gouvernance de l'information en général qui sont à l'initiative de ces projets.