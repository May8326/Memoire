\section{II. L’archiviste en musée : une légitimité difficile à établir, entre injonctions réglementaires et impossibilité pratique}

\subsection{DUA, versement, archivage raisonné : obstacles structurels à la gestion des archives numériques}

\begin{itemize}
	\item \textbf{Problématique} : Comment articuler les exigences réglementaires (DUA, versement, tri, élimination) avec la réalité muséale ? Quels sont les principaux verrous à la mise en œuvre d’un archivage raisonné au MAE ?
	\item \textbf{Pistes d’analyse} :
	\begin{itemize}
		\item Absence de solution de versement définitif (pas de SAE, Vitam inaccessible).
		\item Non-distinction entre archives courantes, intermédiaires, définitives : risque de perte, confusion, surcharge (cf. FICHE\_Information\_Archivistique.md).
		\item Pas de perspective d’évolution à court terme, priorité à d’autres chantiers logiciels (Koha, Archange).
		\item Conséquences : état perpétuellement intermédiaire des archives, gestion par agents producteurs, enjeux de sensibilisation, risque juridique.
	\end{itemize}
	\item \textbf{Exemples} : Absence de tableaux de gestion, serveurs saturés, suppression de fichiers sans traçabilité.
	\item \textbf{Citation} : \og L’absence d’archivage dès la création d’un document numérique peut entraîner sa perte définitive. \fg (FICHE\_Information\_Archivistique.md)
\end{itemize}

\subsection{Agents, archivistes, informaticiens : une gouvernance éclatée de l’archive numérique}

\begin{itemize}
	\item \textbf{Problématique} : Pourquoi la gestion des archives numériques relève-t-elle d’une gouvernance éclatée ? Comment les rapports entre agents, archivistes et informaticiens conditionnent-ils la gestion de l’information ?
	\item \textbf{Pistes d’analyse} :
	\begin{itemize}
		\item Relations documentation, archiviste et informatique : collaboration essentielle mais communication déficiente.
		\item Problème de sensibilisation à l’enjeu juridique, absence de solution technique pour garantir l’immuabilité (cf. FICHE\_Information\_Archivistique.md).
		\item Limites des métadonnées Windows, absence de GED performante, optimisme technique dissimulant la saturation des serveurs.
		\item Décalage entre besoins des agents et capacité décisionnelle de l’institution.
	\end{itemize}
	\item \textbf{Exemples} : Transmission de fichiers lors départ agent (FICHE\_Depart\_Agent\_VF.md), collaboration informatique/documentation.
	\item \textbf{Citation} : \og Il est essentiel de bien évaluer les charges dans les différents projets d’archivage électronique, [...] les contraintes budgétaires constituent un élément fondamental pour construire une politique d’archivage efficace. \fg \footcite{bechardArchivesElectroniques2020a}
\end{itemize}

\subsection{Perspectives et défis pour l’archivage numérique au MAE}

\begin{itemize}
	\item \textbf{Problématique} : Quelles perspectives d’évolution pour une gestion raisonnée des archives numériques ? Quels chantiers, outils, méthodes pour surmonter les obstacles ?
	\item \textbf{Pistes d’analyse} :
	\begin{itemize}
		\item Solutions explorées : SAE, GED, plans de classement, archivage raisonné, externalisation.
		\item Adaptabilité au contexte MAE : contraintes institutionnelles, dépendance au ministère, faiblesse des moyens humains.
		\item Rôle de la formation, sensibilisation et communication interservices.
	\end{itemize}
	\item \textbf{Exemples} : Ateliers collaboratifs, audit des pratiques, intégration de l’archiviste dans les projets numériques.
	\item \textbf{Citation} : \og Le principal défi de l’archivage électronique est dans l’organisation et la structuration de l’information en amont, dans la gestion du cycle de vie et dans l’identification des responsabilités. \fg \footcite{ministeredelacultureArchivesElectroniquesDans2020}
\end{itemize}

\section{Conclusion de la sous-partie II/C}

\begin{itemize}
	\item Synthèse de la crise de l’archivage numérique en musée : fragmentation des espaces, gouvernance éclatée, sous-dimensionnement humain, absence de solutions pérennes.
	\item Spécificité du cas muséal : entre documentation et archives, le musée se trouve à la croisée des chemins, sommé d’inventer une mémoire numérique fidèle à son histoire et adaptée à ses contraintes.
	\item Nécessité d’une politique globale de gouvernance de l’information : structuration, concertation, formation, pour que la mémoire numérique du MAE ne se dissolve pas dans le flux des fichiers et des tâches.
\end{itemize}

\begin{quote}
	\og La mémoire numérique d’un musée ne se construit pas dans l’addition des fichiers, mais dans la cohérence des pratiques et la reconnaissance institutionnelle de la fonction archivistique. \fg
\end{quote}

