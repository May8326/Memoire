\section{La frontière floue entre archives et documentation : une problématique propre aux musées techniques}

\subsection{Définir l’archive numérique en contexte muséal : enjeux et ambiguïtés}

%Les musées en France, institutions dédiées à la conservation et à la diffusion du patrimoine, sont intimement liés à la question des archives : eux-même producteurs d'archives, ils peuvent être également exceptionnellement appelés à en être les dépositaires lorsqu'elles sont indissociables d'une oeuvre. Malgré la prégnance des problématiques liées à la gestion des archives dans les musées dans le paysage français, la littérature scientifique et professionnelle sur ce sujet semble peu prolifique : l'article de référence que nous retiendrons pour ce développement est celui de Véronique Sassetti-Aguilera, \citetitle{sassetti-aguileraArchivesMuseesDiversites2020a}. L'autrice identifie trois problématiques principales dans les musées actuellement : \enquote{La première difficulté réside dans le statut même de l'établissement (public ou privé) qui caractérise le plus souvent la nature des fonds d'archives et des collections [...] La deuxième, nettement plus complexe, relève de l'identification des archives par le personnel des musées comme trésors nationaux, imprescriptibles et inaliénables [...] La troisième enfin, et non la moindre, demeure dans l'enrichissement continuel de la documentation scientifique des collections de musées, situation ne permettant jamais [...] de procédures d'archivage classiques par bordereaux de versement.}. Ces problématiques se retrouvent au \mae. Celui-ci est en effet une institution publique, l'ensemble de ses archives sont donc est celle de la frontière : où finit la documentation, où commence l'archive ? Cette interrogation, classique en archivistique, se complique singulièrement dans l'environnement numérique muséal, où la nature hybride des fonds — documents de gestion, dossiers d'œuvre, photographies, fichiers bureautiques, bases de données — brouille irrémédiablement les distinctions traditionnelles.

\begin{itemize}
	\item \textbf{Problématique} : Qu’est-ce qu’une archive numérique en musée ? En quoi la nature hybride des fonds (documents de gestion, dossiers d’œuvre, photographies, fichiers bureautiques, bases de données) brouille-t-elle la distinction entre documentation courante et archives ?
	\item \textbf{Pistes d’analyse} : 
	\begin{itemize}
		\item Définitions normatives (\textit{archive courante, intermédiaire, définitive}, cf. Code du patrimoine, FICHE\_Information\_Archivistique.md).
		\item Spécificité muséale : la documentation devient archive dès lors qu’elle sert à la mémoire du musée ou à la justification de ses droits.
		\item Les dossiers d’œuvre : archives ou documentation ? \footcite{barbelinDossierDoeuvreDossier2016a}
		\item Données numériques et patrimonialisation de l’information technique \footcite{bechardArchivesElectroniques2020a}
	\end{itemize}
	\item \textbf{Exemples concrets} : Dossiers d’œuvre (archives-administration, rapports, photographies, courriels, cf. FICHE\_Information\_Archivistique.md), photographies de la médiathèque, rapports d’activité.
	\item \textbf{Citation} : \og Les dossiers d’œuvre sont des archives publiques : ils sont librement communicables à tout citoyen. \fg \footcite{barbelinDossierDoeuvreDossier2016a}
\end{itemize}

\subsection{Panorama des pratiques de gestion des archives numériques au MAE}

\begin{itemize}
	\item \textbf{Problématique} : Où et comment sont conservés les fichiers numériques ? Quelles logiques d’organisation ? Quelles difficultés concrètes ?
	\item \textbf{Pistes d’analyse} :
	\begin{itemize}
		\item Cartographie des espaces de stockage : Serveur S (archives courantes, arborescence par service, gestion Windows Explorer), SharePoint, OneDrive.
		\item Hétérogénéité des usages et compétences, conséquences sur la pérennité.
		\item Problèmes récurrents : arborescence trop ou trop peu granulaire (cf. FICHE\_RAPPORT\_ServeurS.md), doublons, conventions de nommage non pérennes (nuage de mots, FICHE\_Bonnes\_Pratiques.md), noms au nom des agents (cf. FICHE\_Photos.md), absence de plan de classement partagé.
		\item Conséquences : invisibilisation, perte de repères, saturation.
	\end{itemize}
	\item \textbf{Exemples concrets} : Ratio fichiers/dossiers, arborescence profonde (16 niveaux, cf. FICHE\_RAPPORT\_ServeurS.md), nommage des photos et vidéos (cf. FICHE\_Photos.md), dossiers « vrac » (FICHE\_Bonnes\_Pratiques.md).
	\item \textbf{Citation} : \og Un bon classement = une information trouvable, fiable et partageable durablement. \fg (FICHE\_Bonnes\_Pratiques.md)
\end{itemize}

\subsection{Un personnel sous-dimensionné : la difficile reconnaissance de la fonction archivistique}

\begin{itemize}
	\item \textbf{Problématique} : Pourquoi la fonction archivistique reste-t-elle marginale dans les musées techniques ? Quelles conséquences sur la gestion des archives numériques ?
	\item \textbf{Pistes d’analyse} :
	\begin{itemize}
		\item Une seule archiviste au MAE depuis 2024, principalement dédiée aux archives privées papier.
		\item Rattachement à la documentation : dilution des missions, surcharge, arbitrages constants.
		\item Absence de politique de formation et de sensibilisation des personnels non archivistes (cf. FICHE\_Information\_Archivistique.md).
		\item Difficulté à établir plan de classement, tableau de gestion, chaîne archivistique.
	\end{itemize}
	\item \textbf{Exemples} : Missions d’archives ponctuelles, gestion dossiers d’œuvres, documentation.
	\item \textbf{Citation} : \og La distinction entre archives et documentation est remarquablement posée, [...] la gestion purement administrative des dossiers d’acquisition permet une uniformisation des pratiques. Peut-on à partir de cet exemple réfléchir à l’élaboration de procédures spécifiques en matière d’archives publiques de musées en France ? \fg \footcite{sassetti-aguileraArchivesMuseesDiversites2020a}
\end{itemize}

