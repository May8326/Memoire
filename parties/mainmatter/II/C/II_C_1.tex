\section{\label{II-C-1}Une problématique spécifique aux musées : définir la frontière entre archives et documentation}

C'est au cœur de cette organisation complexe, que se trouvent les archives du musée, traditionnellement négligées mais désormais reconnues comme des composantes fondamentales de la mémoire aéronautique. Or, la gestion de ces fonds pâtit encore d’un déficit de personnel spécialisé : depuis 2024, une seule archiviste se consacre principalement aux archives privées, tandis que les archives courantes sont en grande partie gérées via des serveurs informatiques et des missions ponctuelles. TODO: Comment pallier ce manque de ressources humaines et mettre en valeur ce patrimoine/garantir la pérennité de l'institution
%Ce constat soulève une autre question majeure : comment garantir la pérennité et la valorisation d’un patrimoine documentaire aussi riche avec des ressources humaines aussi restreintes ?