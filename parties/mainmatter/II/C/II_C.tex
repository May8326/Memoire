\chapter[Les archives numériques]{\label{II-C}Les archives numériques : une autre manifestation des difficultés de gestion de l’information en musée }

\begin{quote}
	\enquote{\textbf{ARCHIVES INFORMATIQUES} \textit{en : electronic records}
		Documents produits ou reçus par un organisme dans l'exercice de
		ses activités et conservés sous forme d'enregistrements
		électroniques sur des supports tels que les bandes magnétiques,
		les disques magnétiques, les disques optiques etc., et qui ne
		peuvent être lus que par l’intermédiaire d’une machine.}
\end{quote}

\lettrine{D}{ans} le musée, la question des archives numériques est à la croisée des chemins : entre mémoire documentaire et gestion administrative, elle met en lumière l’ambiguïté qui sépare – ou confond – archives et documentation. Cette frontière est souvent difficile à définir dans le milieu des musées et, loin d’être purement technique ou juridique, elle recouvre des enjeux de gouvernance, de responsabilité et de pérennité du savoir dans un contexte où la masse des données numériques croît plus vite que ne se structurent les pratiques. L’expérience du \mae, qui se révèle dans son organigramme comme dans sa pratique quotidienne – où l’unique archiviste ne peut déjà embrasser la diversité des fonds papier – illustre les dilemmes contemporains de l’archivage en musée. Comment garantir la pérennité et la communicabilité des archives publiques numériques d'un musée lorsque la distinction entre archive et document est floue, quand l’outil et les référents professionnels manquent, et quand la réflexion collective reste à construire de zéro ? 

Dans une première partie, nous tâcherons d'examiner les ressorts de cette frontière incertaine entre archives et documentation qui existe en contexte muséal, en interrogeant les pratiques, les outils et les propositions institutionnelles pour tenter d'y pallier. Puis, dans une seconde partie, nous porterons l’analyse sur la figure de l’archiviste en musée, dont la légitimité s’avère difficile à établir : nous mettrons en lumière les enjeux techniques, juridiques et professionnels qui entravent la mise en place d’une véritable politique d’archivage raisonné, et les difficultés spécifiques que soulève la gestion quotidienne des archives numériques par des agents souvent peu formés et insuffisamment sensibilisés à la gouvernance documentaire.

%\lettrine{D}{ans} le paysage contemporain des institutions patrimoniales, la question des archives numériques se dresse avec une acuité nouvelle, révélant en creux les tensions qui traversent le quotidien des musées techniques. Au Musée de l’Air et de l’Espace, où la mémoire documentaire s’entrelace étroitement avec la gestion courante, la technicité des outils informatiques se heurte à la faiblesse des moyens humains, et la pratique quotidienne à la rigueur réglementaire. La frontière — déjà ténue — entre archives et documentation s’y révèle d’autant plus poreuse que l’institution ne dispose que depuis peu d’une archiviste dédiée. Il importe alors, pour comprendre les logiques qui entravent la mise en œuvre d’une gouvernance archivistique digne de ce nom, d’interroger la dialectique entre impératifs institutionnels, contraintes techniques, et réalités humaines, sous le prisme de la littérature professionnelle et du vécu des agents.
\section{La frontière floue entre archives et documentation : une problématique propre aux musées techniques}

\subsection{Définir l’archive numérique en contexte muséal : enjeux et ambiguïtés}

%Les musées en France, institutions dédiées à la conservation et à la diffusion du patrimoine, sont intimement liés à la question des archives : eux-même producteurs d'archives, ils peuvent être également exceptionnellement appelés à en être les dépositaires lorsqu'elles sont indissociables d'une oeuvre. Malgré la prégnance des problématiques liées à la gestion des archives dans les musées dans le paysage français, la littérature scientifique et professionnelle sur ce sujet semble peu prolifique : l'article de référence que nous retiendrons pour ce développement est celui de Véronique Sassetti-Aguilera, \citetitle{sassetti-aguileraArchivesMuseesDiversites2020a}. L'autrice identifie trois problématiques principales dans les musées actuellement : \enquote{La première difficulté réside dans le statut même de l'établissement (public ou privé) qui caractérise le plus souvent la nature des fonds d'archives et des collections [...] La deuxième, nettement plus complexe, relève de l'identification des archives par le personnel des musées comme trésors nationaux, imprescriptibles et inaliénables [...] La troisième enfin, et non la moindre, demeure dans l'enrichissement continuel de la documentation scientifique des collections de musées, situation ne permettant jamais [...] de procédures d'archivage classiques par bordereaux de versement.}. Ces problématiques se retrouvent au \mae. Celui-ci est en effet une institution publique, l'ensemble de ses archives sont donc est celle de la frontière : où finit la documentation, où commence l'archive ? Cette interrogation, classique en archivistique, se complique singulièrement dans l'environnement numérique muséal, où la nature hybride des fonds — documents de gestion, dossiers d'œuvre, photographies, fichiers bureautiques, bases de données — brouille irrémédiablement les distinctions traditionnelles.

\begin{itemize}
	\item \textbf{Problématique} : Qu’est-ce qu’une archive numérique en musée ? En quoi la nature hybride des fonds (documents de gestion, dossiers d’œuvre, photographies, fichiers bureautiques, bases de données) brouille-t-elle la distinction entre documentation courante et archives ?
	\item \textbf{Pistes d’analyse} : 
	\begin{itemize}
		\item Définitions normatives (\textit{archive courante, intermédiaire, définitive}, cf. Code du patrimoine, FICHE\_Information\_Archivistique.md).
		\item Spécificité muséale : la documentation devient archive dès lors qu’elle sert à la mémoire du musée ou à la justification de ses droits.
		\item Les dossiers d’œuvre : archives ou documentation ? \footcite{barbelinDossierDoeuvreDossier2016a}
		\item Données numériques et patrimonialisation de l’information technique \footcite{bechardArchivesElectroniques2020a}
	\end{itemize}
	\item \textbf{Exemples concrets} : Dossiers d’œuvre (archives-administration, rapports, photographies, courriels, cf. FICHE\_Information\_Archivistique.md), photographies de la médiathèque, rapports d’activité.
	\item \textbf{Citation} : \og Les dossiers d’œuvre sont des archives publiques : ils sont librement communicables à tout citoyen. \fg \footcite{barbelinDossierDoeuvreDossier2016a}
\end{itemize}

\subsection{Panorama des pratiques de gestion des archives numériques au MAE}

\begin{itemize}
	\item \textbf{Problématique} : Où et comment sont conservés les fichiers numériques ? Quelles logiques d’organisation ? Quelles difficultés concrètes ?
	\item \textbf{Pistes d’analyse} :
	\begin{itemize}
		\item Cartographie des espaces de stockage : Serveur S (archives courantes, arborescence par service, gestion Windows Explorer), SharePoint, OneDrive.
		\item Hétérogénéité des usages et compétences, conséquences sur la pérennité.
		\item Problèmes récurrents : arborescence trop ou trop peu granulaire (cf. FICHE\_RAPPORT\_ServeurS.md), doublons, conventions de nommage non pérennes (nuage de mots, FICHE\_Bonnes\_Pratiques.md), noms au nom des agents (cf. FICHE\_Photos.md), absence de plan de classement partagé.
		\item Conséquences : invisibilisation, perte de repères, saturation.
	\end{itemize}
	\item \textbf{Exemples concrets} : Ratio fichiers/dossiers, arborescence profonde (16 niveaux, cf. FICHE\_RAPPORT\_ServeurS.md), nommage des photos et vidéos (cf. FICHE\_Photos.md), dossiers « vrac » (FICHE\_Bonnes\_Pratiques.md).
	\item \textbf{Citation} : \og Un bon classement = une information trouvable, fiable et partageable durablement. \fg (FICHE\_Bonnes\_Pratiques.md)
\end{itemize}

\subsection{Un personnel sous-dimensionné : la difficile reconnaissance de la fonction archivistique}

\begin{itemize}
	\item \textbf{Problématique} : Pourquoi la fonction archivistique reste-t-elle marginale dans les musées techniques ? Quelles conséquences sur la gestion des archives numériques ?
	\item \textbf{Pistes d’analyse} :
	\begin{itemize}
		\item Une seule archiviste au MAE depuis 2024, principalement dédiée aux archives privées papier.
		\item Rattachement à la documentation : dilution des missions, surcharge, arbitrages constants.
		\item Absence de politique de formation et de sensibilisation des personnels non archivistes (cf. FICHE\_Information\_Archivistique.md).
		\item Difficulté à établir plan de classement, tableau de gestion, chaîne archivistique.
	\end{itemize}
	\item \textbf{Exemples} : Missions d’archives ponctuelles, gestion dossiers d’œuvres, documentation.
	\item \textbf{Citation} : \og La distinction entre archives et documentation est remarquablement posée, [...] la gestion purement administrative des dossiers d’acquisition permet une uniformisation des pratiques. Peut-on à partir de cet exemple réfléchir à l’élaboration de procédures spécifiques en matière d’archives publiques de musées en France ? \fg \footcite{sassetti-aguileraArchivesMuseesDiversites2020a}
\end{itemize}


\section{\label{II-C-2}Vers une gouvernance des archives numériques au \mae : expérimentations, structuration et montée en compétences}


\begin{myquote}
	{Ne pas envisager l’archivage de documents numériques dès leur création, c’est prendre le risque de les perdre définitivement.}{ministeredelacultureArchivesElectroniquesDans2020}
\end{myquote}

\subsection{Quelques leviers}


Loin de se réduire à une succession d’obstacles, la situation des archives numériques au musées est aussi un terrain privilégié pour tenter une réconciliation des dynamiques muséales et archivistiques. Les recommandations nationales – garantie de l'intégrité du fichier et de ses métadonnées, distinction des états d’archives, traçabilité, bordereaux d’élimination – sont peu à peu reconnues, d'abord par le \ac{drd}, sensibilisé par ses missions aux enjeux de gestion de l'information, puis par les agents du \ac{dsc} qui y sont confrontés au quotidien.



Certes, la question du versement définitif demeure ouverte : aucun projet d'intégration immédiate à un \gls{sae} ne paraît envisageable, mais la problématique est reconnue, et il faut reconnaître que la priorité est donnée pour l'instant aux chantiers logiciels menés par le \minarm (\gls{koha}-\gls{clade} et \gls{archange}), reléguant la question de l’archivage à une temporalité indéfinie. La sensibilisation des agents se heurte à la dispersion des responsabilités et la fonction archivistique demeure fragile, mais des quelques avancées pour poser les bases de projets ultérieurs plus conséquents ont été proposées au \ac{dsc}, qui pourrait alors, après une première expérimentation, argumenter pour un chantier commun à l'ensemble du musée.


La première, et l'une des missions secondaires de ce stage, a consisté en la production de moyens de sensibilisation des agents du \ac{dsc} aux enjeux de gouvernance de leurs archives numériques : ceux-ci auront consisté principalement en 
\begin{itemize}
	\item la réalisation de formations de sensibilisation à l’adresse des agents,
	\item la rédaction d’une Foire aux Questions sur la gestion quotidienne et les grands principes d'archivistique[TODO : ajout annexe faq caviardée],
	\item la production de fiches de bonnes pratiques.
\end{itemize}

La seconde, liée à la précédente, est le recueillement des ressentis et des pratiques déjà en vigueur dans le service, afin de proposer des solutions réalistes et adaptées aux exigences métier : cette démarche a abouti à la proposition de normes de nommages de fichiers et de dossiers, et de structuration de l'arborescence générales au service et à la création d'un document partagé de mise en commun du vocabulaire et des abréviations utilisées.


Il s’agit ici d’insuffler une culture commune de la gestion documentaire : nommage, tri, organisation, bordereaux d’élimination sont pensés comme autant de jalons vers une maîtrise accrue du cycle de vie des fichiers. Face à l’impossibilité actuelle d’un archivage définitif, l'engagement d'un professionnel dédié au tri du serveur, à la validation des éliminations et à la mise en conformité des procédures est souhaitable ; dans l'attente d'une solution institutionnelle, une réflexion s’est engagée sur la création d’un dossier réservé à l'archivage des documents arrivés à leur fin d'utilité administrative.


Pour préparer le terrain d'une gestion robuste de ses archives, ces initiatives, portées par le \ac{dsc} illustrent la possibilité d’entamer une dynamique de modernisation fondée sur la formation, l’accompagnement et l’engagement des agents, même en l’absence de dispositifs techniques pleinement aboutis.

\bigskip
\bigskip
\bigskip

La crise de l’archivage numérique en musée ne se résume pas à une accumulation de défauts : elle révèle également la capacité de l’institution à inventer des solutions pour organiser sa gouvernance en conformité avec les recommandations en vigueur et former ses agents. Là encore, ce sont les métiers de la documentation, plus sensibilisés aux enjeux de la gouvernance de l'information en général qui sont à l'initiative de ces projets.

\bigskip
\bigskip
\bigskip

\lettrine{S}{ynthétiser} la crise de l’archivage numérique en musée, c’est reconnaître la fragmentation des espaces, la gouvernance éclatée, le sous-dimensionnement humain, et l’absence de solutions pérennes. Le cas muséal, singulier, se situe à la croisée des chemins : entre documentation et archives, le musée est sommé d’inventer une mémoire numérique qui soit fidèle à son histoire et adaptée à ses contraintes.

La nécessité d’une politique globale de gouvernance de l’information s’impose, pour que la mémoire numérique du MAE ne se dissolve pas dans le flux des fichiers et des tâches. Structuration, concertation, formation, sont les jalons à poser pour qu’à l’avenir, la mémoire du musée ne devienne pas l’otage de l’urgence, mais le reflet d’une institution soucieuse de son passé autant que de son devenir.