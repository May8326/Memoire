\section{\label{III-C-1}Automatiser les tâches : promesses et réalités de l’IA en institution patrimoniale}



Face à cette surcharge, elle semble en effet promettre un grand allègement : comme d'autres projets l'ont montré, celui ci n'est efficace que dans la mesure où l'équilibre avec l'expertise humaine est respecté. \enquote{L’\ac{ia} est \textelp{} un outil complémentaire qui, bien encadré, permettrait d’assister les équipes sans remettre en cause l’expertise humaine essentielle à la validation des informations\footcite{bermesRepenserCollectionsPatrimoniales2025}.} Les cas d’utilisation se multiplient dans les institutions patrimoniales, principalement pour l’indexation automatique : on peut citer par exemple le projet d'indexation automatique RAMEAU à la \ac{bnf} \footcite{filabesLindexationRAMEAUAssistee2025}, ou TORNE-H\footcite{bermesRepenserCollectionsPatrimoniales2025} au Musée des arts décoratifs. Les modèles de langage y assistent la reconnaissance des entités, l’extraction et la normalisation des termes. Les méthodes d’apprentissage supervisé permettent désormais de nettoyer les vocabulaires, de regrouper les variantes et d’automatiser la détection d’incohérences (que ce soit avec l'assistance de \textit{LLM} ou de bibliothèques Python comme \textit{NLTK} ou \textit{spaCy}, distances de chaînes comme \textit{Levenshtein}, modèles linguistiques type \textit{CamemBERT} ou \textit{sentence-transformers}) :

\inserttable{img/TABL_outils_ia.tex}

Ces outils efficaces semblent pour l'instant rester cantonnés à des projets pilotes ou à des institutions prestigieuses dont les ressources sont plus élevées que celles du \mae. Le passage de l’expérimentation à la routine semble, pour l’heure, demeurer l’exception dans la majorité des institutions patrimoniales.