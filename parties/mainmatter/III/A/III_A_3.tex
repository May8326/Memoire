\section{\label{III-A-3}L’intégration dans l’écosystème institutionnel : vers une gouvernance collaborative des outils}

L’intégration du logiciel Opentheso au \mae~semblerait être la solution qui réponde le mieux à ses aspirations de gouvernance du vocabulaire et d'interopérabilité avec le web. Cet outil libre recommandé par le ministère de la Culture, s’impose aujourd'hui en France comme le socle d’un vocabulaire  capable de transcender les cloisonnements logiciels et de fédérer les métiers. Sa conformité aux normes ISO 25964 et au standard \gls{skos}, sa capacité à exporter dans des formats interopérables \gls{rdf}, comme \textit{Turtle} ou \textit{JSON-LD}, à visualiser l’arborescence en graphe, à documenter les relations synonymiques ou hiérarchiques, en font un instrument rigoureux qui peut être au service à la fois de la recherche et de la médiation. Opentheso permet de structurer la connaissance, mais aussi de relier les bases métiers – un plugin d'intégration dans \gls{koha} existe déjà, et il serait envisageable de demander une intégration à \textit{Skinsoft} pour Archange qui permettrait au logiciel d'interagir avec l'\ac{api} d'OpenTheso.

La mise en place, en complément, d'un réseau de logiciels dédiés à une gestion rationalisée de l'information au \mae~-- comme des versements réguliers sur le \gls{sae} Vitam pour l’archivage et la mise en place d'une \gls{ged} – serait d'une grande aide au musée pour parvenir à ses ambitions de gouvernance.

\subsection{Accompagnement au changement et formation des agents}

Toute réforme documentaire ne pourra cependant aboutir sans accompagnement au changement : ateliers de sensibilisation, guides d’usage, fresques de la connaissance, tutoriels, autant de dispositifs qui harmoniseraient les pratiques et préparent les agents à la prise en main des nouveaux outils. La formation s'imposera comme la condition de la réussite : elle permettra de comprendre la logique des outils, de s’approprier les méthodes de structuration et d’assurer la pérennité des acquis.

\subsection{Interopérabilité et dialogue intermétiers}

L’interopérabilité enfin, même si la mise en place de solutions dédiées l'aidera grandement, ne se limitera pas à la technique : elle suppose un dialogue continu entre les métiers du musée et une construction de passerelles entre les silos professionnels et plateformes. Les groupes de travail menés au musée ont permis de réfléchir à des solutions communes pour l’archivage numérique et l'unification des thésaurus, ils devront continuer pour les mettre en place et assurer leur bon fonctionnement. 
