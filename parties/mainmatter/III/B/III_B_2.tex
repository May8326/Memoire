\section{\label{III-B-2}Créer un thésaurus commun à partir des vocabulaires existants : un travail collaboratif}

L’unification procède d’une méthodologie précise : la démarche adoptée au musée s’est articulée autour d’une formation initiale sur ce qu’est un thésaurus, suivie par des groupes de travail généraux pour explorer l'ensemble des thésaurus et recueillir les usages de chacun pour dégrossir les besoins, puis par des groupes thématiques qui ont examiné chaque type de branche (mots-clés, constructeurs, événements, périodes, matériaux...). Selon le corpus, deux approches se sont dégagées : 
\begin{itemize}
	\item soit une analyse terme à terme lorsque le corpus était limité (définition, synonymes, organisation hiérarchique à partir du niveau haut),
	\item soit une recherche de niveaux hauts communs, puis le rangement progressif des termes spécifiques par la suite.
\end{itemize}
À chaque étape, il a fallu observer les règles de normalisations qui avaient pu être suivies dans le passé et proposer des pistes pour en définir de nouvelles qui puissent être générales au musée.

L’harmonisation des branches et des hiérarchies se fait progressivement, en s’appuyant sur les recommandations institutionnelles (Ministère de la Culture, Joconde\footnote{Voir \cite{ministeredelacultureVocabulairesScientifiquesService2014}} pour le musée). Elle est loin d'être achevée aujourd'hui, mais les premières bases ont été posées pour continuer ce travail d'unification et migrer --- à une date qui n'est pas encore définie et qui sera à négocier avec la tutelle --- vers une plateforme de gestion comme OpenTheso\footcite{OpenThesoa} qui permette de faciliter la maintenance de ce travail.

Le processus de fusion, enfin, s’est construit sur la base des vocabulaires existants : il a fallu croiser les listes de termes, identifier des synonymes et des concepts communs, avant de constituer un référentiel central validé collectivement et partagé sur SharePoint.

\subsection{La réflexion en groupes de travail : avantages et limites}

Le travail collectif a présenté des avantages incontestables : il a notamment permis de mutualiser l'expertise de chacun, de confronter les usages et les logiques métiers et de discuter de solutions pragmatiques qui conviennent à l'ensemble des usagers --- c'est par exemple lors de ces réunions qu'il a été choisi de garder au singulier les mots-clés du thésaurus, de les écrire en minuscules contrairement à ce qui était en usage à la bibliothèque, ou ce qui a permis de relever les difficultés inhérentes à la dénomination et à la hiérarchisation des constructeurs d'avions, dont l'histoire mouvementée rend toute classification difficile.

Mais ce mode de travail comporte aussi de réelles limites : le thésaurus étant un outil de structuration du vocabulaire, il reste nécessaire de se pencher sur l'ensemble des mots qu'il contient pour les organiser et les définir. Or, le volume de données à traiter est considérable et il s'est avéré que ce mode de travail ne serait pas suffisant pour l'unification des thésaurus, et qu'une personne seule, ou un groupe de trois personnes maximum peut être plus efficace --- en s'attelant à un travail de tri au fur et à mesure, par petites séances --- qu'un groupe dont l'objectif sera plutôt de discuter des modifications que de les appliquer. Dans l'un ou l'autre cas cependant, la clé d'une harmonisation efficace reste la communication : toute modification doit être explicitée et confrontée au point de vue des autres métiers avant d'entrer en vigueur.

\inserttable{img/TABL_cr_gt_thematique}

\subsection{Où se situent les principaux besoins ?}

Le constat est paradoxal : ce n’est pas dans les segments les plus techniques ou spécialisés que le besoin d’unification se fait le plus sentir --- ces vocabulaires sont déjà travaillés, structurés, et adossés à la physicalité des objets. Les difficultés majeures surgissent dans les intersections avec les autres institutions patrimoniales, là où le choix entre le respect des normes (ISO 25964, Joconde, SUDOC) et la conservation des spécificités du musée devient le plus épineux.  [TODO : exemple à trouver]