\section{\label{III-B-1}Convaincre de la nécessité du processus}

La structuration du savoir, loin d’être un acte isolé, engage l’institution dans une négociation permanente entre ses acteurs, ses usages et ses impératifs de transmission. La mise en œuvre d’une unification des thésaurus au sein du \mae ne procède donc pas d’une simple injonction technique. Comme nous l'avons vu précédemment, bibliothèque, e-médiathèque et musée disposent chacun de leurs référentiels propres, hérités d’usages, de logiques métiers et de logiciels distincts. Le \mae étant intégré au réseau des musées et bibliothèques du \minarm, toute décision d’unification --- à titre d’exemple, l’intégration d’OpenTheso dans \gls{koha}\footnote{Cette intégration est possible grâce à un plugin développé par la société \textit{Tamil}, voir \cite{PluginTamilOpentheso}} --- peut posséder un impact qui dépasse le seul cadre du \mae pour concerner l'ensemble de ces institutions. Il est donc illusoire de prétendre harmoniser localement une pratique documentaire sans se heurter à la doctrine technique et documentaire imposée à l’échelle ministérielle : ceci rend la négociation indispensable.

Or, nous avons vu que l’interopérabilité s’impose comme une condition de la communicabilité et de la valeur scientifique du fonds documentaire\footcite{hudonISO25964Pour2012a, maroyeISO25964Distinction2015}. Convaincre de la nécessité du processus, c’est inscrire la démarche dans une logique de mutualisation, de visibilité accrue des collections et de conformité aux standards nationaux et internationaux (ISO 25964, \ac{skos}).

Mais ce chantier excède de loin le quotidien des usagers du thésaurus. L’unification s’ajoute aux missions ordinaires des agents, requérant du temps, des compétences techniques et une acculturation documentaire spécifique. Il s’agit donc d’un travail de conviction : chaque professionnel doit être convaincu du bénéfice collectif que représente la démarche. En effet, la réussite du processus dépend de l’implication des agents, des groupes de travail transversaux, du partage de la documentation et de la prise en compte des besoins spécifiques de chaque métier. Insister sur la dimension collaborative est essentiel : seule la concertation régulière entre les différents usagers des thésaurus permet de dépasser les clivages métier.
