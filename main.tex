%Document vide aux normes de l'École nationale des Chartes
%crée par J.B. Camps
%Dernières modifications E. Rouquette (03/2025)

%Personnalisation: Maëlys Gioan (2025)
% - bibliographie thématique
% - acronymes


%%%%%%%%%%%%%%%%%%%%%%%%%%%%%%%%%%%%%%%%%%%%%%%%%%%%%%%%%%%%%%%%%%
%%%%%%%%%%%%%%%%%%%%%%%%%%% PRÉAMBULE %%%%%%%%%%%%%%%%%%%%%%%%%%%%
%%%%%%%%%%%%%%%%%%%%%%%%%%%%%%%%%%%%%%%%%%%%%%%%%%%%%%%%%%%%%%%%%%


%%%%%%%%%%%%%%%%%%%%%% PARTIE OBLIGATOIRE%%%%%%%%%%%%%%%%%%%%%%%%%
%%%%%%%%%%%%%%%%%%%%%%%%%%%%%%%%%%%%%%%%%%%%%%%%%%%%%%%%%%%%%%%%%%

\documentclass[12pt,twoside]{book}
\usepackage{fontspec}
\usepackage{xunicode}
\usepackage[french]{babel}
%\setotherlanguage{} %indiquer les autres langues utilisée


%%%%%%%%%%%%%%%%%%%%%% PARTIE OBLIGATOIRE%%%%%%%%%%%%%%%%%%%%%%%%%
%%%%%%%%%%%%%%%%%%%%%%%%%%%%%%%%%%%%%%%%%%%%%%%%%%%%%%%%%%%%%%%%%%

\usepackage{csquotes} % guillemets français
\usepackage{lettrine}

%%%%%%%%%%%%%%%%%%%%%%%%% BIBLIOGRAPHIE %%%%%%%%%%%%%%%%%%%%%%%%%%

\usepackage[style=enc,sorting=nyt,maxbibnames=10]{biblatex}% style de l'EnC (https://ctan.org/pkg/biblatex-enc)

%\defbibheading{}{\subsection*{Biblio}} % Pour changer le titre de la/les bibliographie(s)
\addbibresource{./bibliographie/archives_numeriques.bib}
\addbibresource{./bibliographie/cooperation.bib}
\addbibresource{./bibliographie/histoire_musee.bib}
\addbibresource{./bibliographie/numerique_patrimoine.bib}
\addbibresource{./bibliographie/thesaurus.bib}
\addbibresource{./bibliographie/reflexions.bib}

%%%% Mots clés pour gérer la bibliographie thématique
\DeclareSourcemap{
 \maps[datatype=bibtex]{
  \map[overwrite]{
    \perdatasource{./bibliographie/archives_numeriques.bib}
    \step[fieldset=keywords, fieldvalue={,texArchivesNumeriques}, append]
  }
  \map[overwrite]{
    \perdatasource{./bibliographie/cooperation.bib}
    \step[fieldset=keywords, fieldvalue={,texCollab}, append]
  }
  \map[overwrite]{
    \perdatasource{./bibliographie/histoire_musee.bib}
    \step[fieldset=keywords, fieldvalue={,texHistoire}, append]
  }
  \map[overwrite]{
    \perdatasource{./bibliographie/numerique_patrimoine.bib}
    \step[fieldset=keywords, fieldvalue={,texNumeriquePatrimoine}, append]
  }
  \map[overwrite]{
    \perdatasource{./bibliographie/thesaurus.bib}
    \step[fieldset=keywords, fieldvalue={,texThesau}, append]
  }
  \map[overwrite]{
  	\perdatasource{./bibliographie/reflexions.bib}
  	\step[fieldset=keywords, fieldvalue={,reflexions}, append]
  }
 }
}


%%%%%%%%%%%%%%%%%%%%%%%%%%%%% INDEX %%%%%%%%%%%%%%%%%%%%%%%%%%%%%%

\usepackage{imakeidx} %pour faire un ou plusieurs index
\makeindex %commande pour générer l'index


%%%%%%%%%%%%%%%%%%%%%%%%% ABREVIATIONS %%%%%%%%%%%%%%%%%%%%%%%%%%

\usepackage{acro}

\DeclareAcronym{mae}{short= MAE, long = Musée de l'air et de l'espace}
\DeclareAcronym{api}{short = API, long = Application Programming Interface}
\DeclareAcronym{dsc}{short = DSC, long = Département scientifique des collections}
\DeclareAcronym{drd}{short = DRD, long = Département recherche et documentation}
\DeclareAcronym{skos}{short = SKOS, long = Simple Knowledge Organization System}
\DeclareAcronym{sparql}{short = SPARQL, long = SPARQL Protocol and RDF Query Language}
\DeclareAcronym{json}{short = JSON, long = JavaScript Object Notation}
\DeclareAcronym{bnf}{short = BnF, long = Bibliothèque nationale de France}
\DeclareAcronym{siae}{short = SIAE, long = Salon international de l'aéronautique et de l'espace}


%%%%%%%%%%%%%%%%%%%%%%%%%%%%%%%%%%%%%%%%%%%%%%%%%%%%%%%%%%%%%%%%%%
%%%%%%%%%%%%%%%%%%%%%%%%% MISE EN PAGE %%%%%%%%%%%%%%%%%%%%%%%%%%%
%%%%%%%%%%%%%%%%%%%%%%%%%%%%%%%%%%%%%%%%%%%%%%%%%%%%%%%%%%%%%%%%%%


%%%%%%%%%%%%%%%%%%%%%%%%%% COMPTEURS %%%%%%%%%%%%%%%%%%%%%%%%%%%%%
%%%%%%%%%%%%%%%%%%%%%%%%%%%%%%%%%%%%%%%%%%%%%%%%%%%%%%%%%%%%%%%%%%
%%%% Sections, subsections, subsubsections

\renewcommand{\thesection}{\Roman{section}.}%On ne fait apparaître que le numéro de la section
\renewcommand{\thesubsection}{\arabic{subsection}.}%subsection en chiffres arabes
\renewcommand{\thesubsubsection}{\alph{subsubsection}.}%subsubsection en lettres minuscules
%Si l'on veut faire apparaître les subsubsection dans le table des matières (à commenter sinon)
\setcounter{tocdepth}{3}
\setcounter{secnumdepth}{3}  % La subsubsection (profondeur=3 dans la table des matières) apparait numérotée dans la TdM



%%%%%%%%%%%%%%%%%%%%%% NORMES DE L'ECOLE %%%%%%%%%%%%%%%%%%%%%%%%%
%%%%%%%%%%%%%%%%%%%%%%%%%%%%%%%%%%%%%%%%%%%%%%%%%%%%%%%%%%%%%%%%%%

\usepackage[margin=2.5cm]{geometry} %marges
\usepackage{setspace} % espacement qui permet ensuite de définir un interligne
\onehalfspacing % interligne de 1.5
\setlength\parindent{1cm} % indentation des paragraphes à 1 cm


%%%%%%%%%%%%%%%%%%%%%%%%%%% HEADERS %%%%%%%%%%%%%%%%%%%%%%%%%%%%%

\usepackage{fancyhdr} %package utilisé pour modifier les headers
\pagestyle{fancy} %utiliser ses propres choix de mise en page et non ceux par défaut du package

\setlength\headheight{16pt}%la hauteur des headers

%%%%%%%%%%%%%%%% SECTIONS dans les en-têtes %%%%%%%%%%%%%%%%%%%%%%

%%\renewcommand{\sectionmark}[1]{\markright{\small\textit{\thesection~\  #1}}}%Faire apparaître dans les headers les sections en  petit et en italiques
\renewcommand{\sectionmark}[1]{}%Commenter la lign précédetne et mettre celle-ci pour ne pas avoir le titre des sections dans le header


%%%%%%%%%%%%%%%%%%%%%%%%% FRONTMATTER %%%%%%%%%%%%%%%%%%%%%%%%%%%%
%%%%%%%%%%%%%%%%%%%%%%%%%%%%%%%%%%%%%%%%%%%%%%%%%%%%%%%%%%%%%%%%%%

\appto\frontmatter{\pagestyle{fancy}%
	\renewcommand{\chaptermark}[1]{\markboth{\small\textit{#1}}{}}% ne pas faire apparaître de <<numéro>> de chapitre dans les chapitres non numérotés (front: l'introduction, els remerciement, etc)
}

%%%%%%%%%%%%%%%%%%%%%%%%% MAINMATTER %%%%%%%%%%%%%%%%%%%%%%%%%%%%
%%%%%%%%%%%%%%%%%%%%%%%%%%%%%%%%%%%%%%%%%%%%%%%%%%%%%%%%%%%%%%%%%%

\appto\mainmatter{
\renewcommand{\chaptermark}[1]{\markboth{\small\chaptername~\thechapter~--\ \textit{#1}}{}}%faire apparaître dans les headers les sections en  petit et en italiques
%\renewcommand{\chaptermark}[1]{}%Commenter la ligne précédente et mettre celle-ci pour ne pas avoir le titre des chapitres  dans le header
}

%%%%%%%%%%%%%%%%%%%%%%%%%%% ANNEXES %%%%%%%%%%%%%%%%%%%%%%%%%%%%%%
%%%%%%%%%%%%%%%%%%%%%%%%%%%%%%%%%%%%%%%%%%%%%%%%%%%%%%%%%%%%%%%%%%

\appto\appendix{
	\renewcommand{\chaptermark}[1]{\markboth{\small~Annexe \thechapter~--\ \textit{#1}}{}}%faire apparaître dans les headers le nom des annexes
	%\renewcommand{\chaptermark}[1]{}%Commenter la ligne précédente et mettre celle-ci pour ne pas avoir le titre des chapitres  dans le header
}

%%%%%%%%%%%%%%%%%%%%%%%%%%%%%%%%%%%%%%%%%%%%%%%%%%%%%%%%%%%%%%%%%%
%%%%%%%%%%%%%%%%%%%%%%% AUTRES PACKAGES %%%%%%%%%%%%%%%%%%%%%%%%%%
%%%%%%%%%%%%%%%%%%%%%%%%%%%%%%%%%%%%%%%%%%%%%%%%%%%%%%%%%%%%%%%%%%

%%%%%%%%%%%%%%%%%%%%%%%%%% TABLEAUX %%%%%%%%%%%%%%%%%%%%%%%%%%%%%%
%%%%%%%%%%%%%%%%%%%%%%%%%%%%%%%%%%%%%%%%%%%%%%%%%%%%%%%%%%%%%%%%%%
\usepackage{multirow}
\usepackage{ltxtable}
\usepackage{tabularx}
\usepackage{longtable}
\usepackage{booktabs}
\usepackage[table]{xcolor}
\usepackage{longtable}
\usepackage{array}
\usepackage{multirow}
\definecolor{lightgray}{gray}{0.95}

\newcolumntype{B}{>{\bfseries\raggedleft\arraybackslash}p{0.25\textwidth}}
\newcolumntype{C}{>{\raggedright\arraybackslash}p{0.75\textwidth}}

%%%%%%%%%%%%%%%%%%%%%%%%% HYPHENATION %%%%%%%%%%%%%%%%%%%%%%%%%%%%
%%%%%%%%%%%%%%%%%%%%%%%%%%%%%%%%%%%%%%%%%%%%%%%%%%%%%%%%%%%%%%%%%%

%indiquer des règles d'hyphénation pour des mots précis si besoin
%\begin{hyphenrules}{french}
%	\hyphenation{}
%\end{hyphenrules}


%%%%%%%%%%%%%%%%%%%%%%%%%% HYPERREF %%%%%%%%%%%%%%%%%%%%%%%%%%%%%%
%%%%%%%%%%%%%%%%%%%%%%%%%%%%%%%%%%%%%%%%%%%%%%%%%%%%%%%%%%%%%%%%%%

% A mettre après les autres appels de packages car redéfinit certaines commandes.

\usepackage[colorlinks=false, breaklinks=true, pdfusetitle, pdfsubject ={Mémoire TNAH}, pdfkeywords={les mots-clés}]{hyperref} %
\usepackage[numbered]{bookmark}%va avec hyperref; marche mieux pour les signets. l'option numbered: les signets dans le pdf sont numérotés

% Compléter pdfsubjet et pdfkeywords
%Explication des options de hyperref (modifiables)
% hyperindex=false
% colorlinks=false: pour que le cadre des liens n'apparaisse pas à l'impression
% breaklinks permet d'avoir des liens allant sur pusieurs lignes
%pdfusetitle: utiliser \author et \title pour produire le nom et le titre du pdf


%avec overleaf, utiliser :
%\usepackage[xetex]{hyperref}
%\hypersetup{
	%	pdfauthor = {Prénom Nom},
	%	pdftitle = {titre},
	%	pdfsubject = {sujet},
	%	pdfkeywords = {premier mot-clé} {deuxième mot-clé} {troisième mot-clé} {etc}
	%}



%%%%%%%%%%%%%%%%%%%%%%%%%% GLOSSAIRE %%%%%%%%%%%%%%%%%%%%%%%%%%%%%
%%%%%%%%%%%%%%%%%%%%%%%%%%%%%%%%%%%%%%%%%%%%%%%%%%%%%%%%%%%%%%%%%%

%Exception: il faut le charger APRÈS hyperref
\usepackage[toc=true]{glossaries}
\setglossarystyle{altlistgroup}
\makeglossaries
%avec TexStudio: F9 pour compiler le glossaire (s'il y a aussi un index)

\loadglsentries{./parties/backmatter/glossaire.tex}



%%%%%%%%%%%%%%%%%%%%%%%%%%%%%%%%%%%%%%%%%%%%%%%%%%%%%%%%%%%%%%%%%%
%%%%%%%%%%% COMMANDES EN ENVIRONNEMENTS PERSONNALISES %%%%%%%%%%%%
%%%%%%%%%%%%%%%%%%%%%%%%%%%%%%%%%%%%%%%%%%%%%%%%%%%%%%%%%%%%%%%%%%


%%%%%%%%%%%%%%%%%%%%%%%%%% COMMANDES %%%%%%%%%%%%%%%%%%%%%%%%%%%%%
%%%%%%%%%%%%%%%%%%%%%%%%%%%%%%%%%%%%%%%%%%%%%%%%%%%%%%%%%%%%%%%%%%


%%%%%%%%%%%%%%%%%%%%%%%% ENVIRONNEMENTS %%%%%%%%%%%%%%%%%%%%%%%%%%
%%%%%%%%%%%%%%%%%%%%%%%%%%%%%%%%%%%%%%%%%%%%%%%%%%%%%%%%%%%%%%%%%%


%%%%%%%%%%%%%%%%%%%%%%%%%%%%%%%%%%%%%%%%%%%%%%%%%%%%%%%%%%%%%%%%%%
%%%%%%%%%%%%%%% INFORMATIONS DE LA PAGE DE TITRE %%%%%%%%%%%%%%%%%
%%%%%%%%%%%%%%%%%%%%%%%%%%%%%%%%%%%%%%%%%%%%%%%%%%%%%%%%%%%%%%%%%%
 
\author{Maëlys \textsc{Gioan} - M2 TNAH}
\title{Gestion de l’information à l’ère du numérique : entre héritage et innovation}


%%%%%%%%%%%%%%%%%%%%%%%%%%%%%%%%%%%%%%%%%%%%%%%%%%%%%%%%%%%%%%%%%%
%%%%%%%%%%%%%%%%%%%%%%%%%%%%%%%%%%%%%%%%%%%%%%%%%%%%%%%%%%%%%%%%%%
%%%%%%%%%%%%%%%%%%%%%%%%%% DOCUMENT %%%%%%%%%%%%%%%%%%%%%%%%%%%%%%
%%%%%%%%%%%%%%%%%%%%%%%%%%%%%%%%%%%%%%%%%%%%%%%%%%%%%%%%%%%%%%%%%%
%%%%%%%%%%%%%%%%%%%%%%%%%%%%%%%%%%%%%%%%%%%%%%%%%%%%%%%%%%%%%%%%%%



\begin{document}

%%%%%%%%%%%%%%%%%%%%%%%%%%%%%%%%%%%%%%%%%%%%%%%%%%%%%%%%%%%%%%%%%%
%%%%%%%%%%%%%%%%%%%%%%%%% FRONTMATTER %%%%%%%%%%%%%%%%%%%%%%%%%%%%
%%%%%%%%%%%%%%%%%%%%%%%%%%%%%%%%%%%%%%%%%%%%%%%%%%%%%%%%%%%%%%%%%%

%%%%%%%%%%%%%%%%%%%%%%%%% PAGE DE TITRE %%%%%%%%%%%%%%%%%%%%%%%%%%
%%%%%%%%%%%%%%%%%%%%%%%%%%%%%%%%%%%%%%%%%%%%%%%%%%%%%%%%%%%%%%%%%%

	\begin{titlepage}
		\begin{center}
			
			\bigskip
			
			\begin{large}				
				ÉCOLE NATIONALE DES CHARTES\\
				UNIVERSITÉ PARIS, SCIENCES \& LETTRES
			\end{large}
			\begin{center}\rule{2cm}{0.02cm}\end{center}
			
			\bigskip
			\bigskip
			\bigskip
			\begin{Large}
				\textbf{Maëlys Gioan}\\
			\end{Large}
		%selon le cas
			\begin{normalsize} \textit{licenciée ès histoire}\\
				%\textit{diplômé.e de master}
			\end{normalsize}
			
			\bigskip
			\bigskip
			\bigskip
			
			\begin{Huge}
				\MakeUppercase{\textbf{La gestion de l’information à l’ère du numérique.\\ \textit{Entre héritage et innovation}}}\\
			\end{Huge}
			\bigskip
			\bigskip
			\begin{LARGE}
				\textbf{Le cas du Musée de l’Air et de l’Espace}\\
			\end{LARGE}
			
			\bigskip
			\bigskip
			\bigskip
			\begin{large}
			\end{large}
			\vfill
			
			\begin{large}
				Mémoire pour le diplôme de master \\
				\enquote{Technologies numériques appliquées à l'histoire} \\
				\bigskip
				2025
			\end{large}
			
		\end{center}
	\end{titlepage}

	\thispagestyle{empty}	
	\cleardoublepage
	
\frontmatter


%%%%%%%%%%%%%%%%%%%%%%%%%%%% RESUME %%%%%%%%%%%%%%%%%%%%%%%%%%%%%%
%%%%%%%%%%%%%%%%%%%%%%%%%%%%%%%%%%%%%%%%%%%%%%%%%%%%%%%%%%%%%%%%%%


	\chapter{Résumé}
\medskip
	Résumé du mémoire en français. Cette page ne doit pas dépasser une page.\\
	
	\textbf{Mots-clés:} une liste de mots-clés~; séparés par des points-virgules.
	
	\textbf{Informations bibliographiques:} GIOAN Maëlys, \textit{Gestion de l’information à l’ère du numérique : entre héritage et innovation. Le cas du Musée de l’Air et de l’Espace}, mémoire de master \enquote{Technologies numériques appliquées à l'histoire}, dir. Emmanuelle Bermès, Valérie Joyaux, École nationale des chartes, 2025.
	
		\newpage{\pagestyle{empty}\cleardoublepage}

%%%%%%%%%%%%%%%%%%%%%%%%% REMERCIEMENTS %%%%%%%%%%%%%%%%%%%%%%%%%%
%%%%%%%%%%%%%%%%%%%%%%%%%%%%%%%%%%%%%%%%%%%%%%%%%%%%%%%%%%%%%%%%%%

	\chapter{Remerciements}
	
\lettrine{M}es remerciements vont tout d'abord à\dots
	\newpage{\pagestyle{empty}\cleardoublepage}
	
\chapter{Liste des abréviations}
	\printacronyms[heading=none]

%%%%%%%%%%%%%%%%%%%%%%%%% BIBLIOGRAPHIE %%%%%%%%%%%%%%%%%%%%%%%%%%
%%%%%%%%%%%%%%%%%%%%%%%%%%%%%%%%%%%%%%%%%%%%%%%%%%%%%%%%%%%%%%%%%%

\chapter{Bibliographie}
%\addcontentsline{toc}{chapter}{Bibliographie}
\printbibliography[keyword={texHistoire}, title={Histoire du musée de l'air et de l'espace}]
\printbibliography[keyword={texCollab}, title={Collaboration entre institutions patrimoniales}]
\printbibliography[keyword={texNumeriquePatrimoine}, title={Numérique en institution patrimoniale}]
\printbibliography[keyword={texArchivesNumeriques}, title={Archives Numériques}]
\printbibliography[keyword={texThesau}, title={Gestion de thésaurus}]
\printbibliography[keyword={reflexions}, title={Références littéraires}]
	
 

%%%%%%%%%%%%%%%%%%%%%%%%% INTRODUCTION %%%%%%%%%%%%%%%%%%%%%%%%%%%
%%%%%%%%%%%%%%%%%%%%%%%%%%%%%%%%%%%%%%%%%%%%%%%%%%%%%%%%%%%%%%%%%%

\chapter{Introduction}	

\begin{quote}
    \og La Bibliothèque comporte toutes les structures verbales, toutes les variations que permettent les vingt-cinq symboles orthographiques, mais point un seul non-sens absolu [...] Je ne puis combiner une série quelconque de caractères, par exemple	\textit{Dhcmrlchtdj} que la divine Bibliothèque n’ait déjà prévue, et qui dans quelqu'une de ses langues secrètes ne renferme
	une signification terrible\footcite{borges_bibliotheque_2011}.\fg
\end{quote} %Input: importer un fichier

\newpage{\pagestyle{empty}\cleardoublepage}

%%%%%%%%%%%%%%%%%%%%%%%%%%%%%%%%%%%%%%%%%%%%%%%%%%%%%%%%%%%%%%%%%%
%%%%%%%%%%%%%%%%%%%%%%%%% MAINMATTER %%%%%%%%%%%%%%%%%%%%%%%%%%%%%
%%%%%%%%%%%%%%%%%%%%%%%%%%%%%%%%%%%%%%%%%%%%%%%%%%%%%%%%%%%%%%%%%%

	\mainmatter

%%%%%%%%%%%%%%%%%%%%%%%%%%% PARTIE I %%%%%%%%%%%%%%%%%%%%%%%%%%%%%
%%%%%%%%%%%%%%%%%%%%%%%%%%%%%%%%%%%%%%%%%%%%%%%%%%%%%%%%%%%%%%%%%%

	\part{Le contexte institutionnel particulier du Musée de l'air et de l'espace}

%%%%%%%%%%%%%%%%%%%%%%%%%% INTRODUCTION %%%%%%%%%%%%%%%%%%%%%%%%%%

Ici, je pourrai mettre une introduction de ma première partie.

%%%%%%%%%%%%%%%%%%%%%%%%%%%%% TEXTE %%%%%%%%%%%%%%%%%%%%%%%%%%%%%

\chapter[Référence nationale]{\label{I-A}Une référence nationale pour les collections aéronautiques}

\lettrine{L}e \mae du Bourget occupe une position singulière dans le paysage muséographique français. Institution technique aux collections exceptionnelles, il incarne les défis contemporains de la conservation patrimoniale appliquée aux objets technologiques. Son histoire mouvementée témoigne des difficultés rencontrées par les institutions dédiées au patrimoine technique pour trouver leur légitimité, et a façonné un musée unique qui dépasse la simple fonction conservatoire pour s'affirmer comme un acteur central de la recherche aéronautique.

\section{\label{I-A-1}Un enjeu de représentation nationale et une autorité auprès des musées similaires}

Le Musée de l’Air et de l’Espace ne s’est pas imposé d’emblée comme une institution majeure dans le paysage culturel français. Son histoire est récente, et c’est seulement à partir du milieu du XXᵉ siècle qu’il s’est progressivement affirmé, témoignant d’une lente mais ferme professionnalisation. À l’origine, ce sont principalement des militaires ou des passionnés d’aéronautique qui en assuraient la direction et la gestion, ce qui, bien que légitime, limitait les perspectives muséales et patrimoniales à une vision technique, voire partielle, du patrimoine aéronautique. Le musée apparaissait davantage comme un lieu de mémoire militaire que comme un établissement culturel capable d’embrasser toutes les dimensions de l’aéronautique.

C’est donc à partir des années 2000 que le musée s’est véritablement transformé, sous l’effet conjoint d’une reconnaissance accrue de l’importance culturelle du secteur aéronautique et d’une volonté institutionnelle d’inscrire cette entité dans le réseau national des musées. Le déménagement du musée du Grand Palais au Bourget, en 1975, est révélateur de cette double fonction qu’il occupe aujourd’hui : à la fois conservatoire historique d’un patrimoine technique unique et vitrine nationale d’une industrie stratégique. L’aéroport du Bourget, premier aérodrome civil français, constitue un lieu hautement symbolique, qui confère au musée une légitimité forte. Par ailleurs, le lien étroit avec le Salon international de l’aéronautique et de l’espace confère à l’institution une dimension promotionnelle, où l’histoire se mêle à la modernité, et la culture au dynamisme industriel.

Cette proximité soulève néanmoins une question essentielle : dans quelle mesure le Musée de l’Air et de l’Espace, en étant si étroitement associé à une manifestation commerciale, peut-il conserver sa posture de conservateur impartial et de référence scientifique ? Cette interrogation traverse les pratiques et les choix stratégiques du musée, notamment dans ses efforts pour se professionnaliser et renforcer son expertise muséale. La question de la tutelle et des modes de gestion n’est pas moins cruciale : longtemps administré par des instances militaires, le musée a dû repenser ses organigrammes, en répartissant clairement les responsabilités entre Recherche, Documentation et Conservation, regroupées aujourd’hui dans un département unique, le DSC (Département des Collections). Ce regroupement vise à favoriser les synergies, mais il pose aussi des défis, notamment en termes de gestion des archives, où les moyens restent limités.

Au cœur de cette organisation complexe, se trouve la bibliothèque et les archives, traditionnellement négligées mais désormais reconnues comme des composantes fondamentales de la mémoire aéronautique. Or, la gestion de ces fonds pâtit encore d’un déficit de personnel spécialisé : depuis 2024, une seule archiviste se consacre principalement aux archives privées, tandis que les archives courantes sont en grande partie gérées via des serveurs informatiques et des missions ponctuelles de stagiaires. Ce constat soulève une autre question majeure : comment garantir la pérennité et la valorisation d’un patrimoine documentaire aussi riche avec des ressources humaines aussi restreintes ?

Par-delà son histoire et sa structure, ce qui distingue avant tout le Musée de l’Air et de l’Espace, c’est la richesse et la diversité de ses collections, qui en font une référence nationale sans équivalent. On y trouve, bien sûr, des avions historiques, des moteurs, des équipements techniques — objets dont la conservation requiert des conditions très spécifiques et une expertise rare. Cette particularité technique, sans précédent dans les musées français, impose une gestion adaptée et des vocabulaires spécialisés. Mais la collection ne se limite pas à ces objets spectaculaires. S’y ajoutent des maquettes, des estampes, des objets d’art, et des ensembles communs aux musées militaires tels que uniformes ou vestiaires. La prise en compte, plus récente, des collections civiles — vêtements d’aviateurs civils, objets du quotidien — témoigne d’une évolution de la politique muséale vers une approche plus anthropologique, qui valorise l’histoire sociale et humaine de l’aéronautique.

À côté de cette richesse matérielle, la documentation constitue un pilier essentiel : la base exhaustive de périodiques aéronautiques, les publications techniques, les archives photographiques et audiovisuelles illustrent la volonté du musée d’être aussi un centre de recherche et de diffusion du savoir. L’organisation interne, qui regroupe collections, documentation et recherche sous une même direction, traduit une conception intégrée du patrimoine aéronautique, mais elle fait également apparaître les différences fondamentales entre les métiers concernés — différence qui, si elle est source de richesse, génère aussi des tensions et complexifie le fonctionnement quotidien.

Enfin, cette singularité du musée reflète une réalité plus large, celle des musées techniques, qui occupent une place particulière dans le paysage muséal français. Souvent mis à l’écart au profit des musées beaux-arts, ces établissements rencontrent des difficultés spécifiques. Leurs chargés de collections, formés à la fois aux savoirs techniques et aux pratiques muséales, subissent parfois une reconnaissance moindre dans le secteur culturel. Ces musées doivent sans cesse composer avec la nature même de leurs collections — objets souvent volumineux, complexes à conserver et à exposer — ce qui impose des méthodes innovantes et une adaptation constante.

Ainsi, le Musée de l’Air et de l’Espace s’inscrit dans cette catégorie d’institutions où l’expertise technique se mêle à l’exigence muséale, conférant à l’établissement un statut d’autorité et de référence dans son domaine. Cette position, fragile et exigeante, le place au carrefour des enjeux de représentation nationale, de conservation patrimoniale, et d’innovation culturelle.


\section{\label{I-A-2}Rôle déterminant dans la recherche}

Texte ici


\bigskip
\bigskip
\bigskip

Le \mae révèle ainsi les enjeux complexes qui traversent les institutions patrimoniales techniques contemporaines. Entre conservation d'objets monumentaux, médiation auprès de publics diversifiés et participation aux réseaux de recherche, il illustre les mutations du métier muséal. Sa capacité à articuler expertise technique et exigence culturelle témoigne d'une professionnalisation réussie mais fragile. Les recherches actuelles sur son organisation soulignent également la nécessité d'un équilibre constant entre efficacité et reconnaissance des spécificités professionnelles. 
\chapter[Acteurs et dépendances]{\label{I-B}De nombreux acteurs et dépendances ministérielles }


\lettrine{I}ci, je pourrai mettre une intro pour mon chapitre

\section{\label{I-B-1}A musée d’exception, contraintes d’exception : un musée étroitement dépendant du ministère de la Défense. }

Le Musée de l’Air et de l’Espace, en tant que musée d’exception, ne jouit paradoxalement d’aucune autonomie véritable. Placé sous l’autorité du ministère des Armées, il est pris dans un faisceau de décisions ministérielles qui conditionnent ses évolutions, qu’elles soient techniques, documentaires ou institutionnelles. Cela vaut tant pour les choix informatiques (comme la solution d’archivage SAE) que pour les outils de catalogage ou de diffusion des collections. L’intégration d’un outil tel qu’Opentheso dans Koha, par exemple, ne saurait être envisagée sans considérer son impact sur l’ensemble des musées et bibliothèques relevant de la même tutelle. Le musée partage son infrastructure avec d'autres entités défendant un patrimoine militaire et technique, et c’est donc à léchelle ministérielle que les décisions sont prises, parfois au prix d’une adaptation imparfaite aux besoins spécifiques du site du Bourget.\footnote{Par exemple, la cartographie des flux de données de thésaurus au \mae en Annexe \ref{Ax-C} montre bien le rôle central du ministère des armées comme validateur et gestionnaire des bases de données publiques du musée (Archange et Clade).}

Ces décisions techniques sont également influencées par les priorités politiques du ministère, notamment en ce qui concerne la présence du musée dans les grands événements comme le salon du Bourget. L’ANSI (Agence du Numérique de la Sécurité Informatique) joue par ailleurs un rôle déterminant dans la validation des solutions techniques, imposant des contraintes de sécurité parfois difficilement conciliables avec les outils de la recherche ou du patrimoine. La gestion documentaire, l’interopérabilité des thésaurus, ou encore la diffusion des collections ne peuvent donc être pensées uniquement à léchelle du musée lui-même. À cela s’ajoute un panorama complexe de tutelles multiples : entre la Direction de la Mémoire, de la Culture et des Archives (DMCA), la Délégation à l’Information et à la Communication de la Défense (DICoD), et les directions propres à chaque arme, les marges de manœuvre apparaissent étroites.



\section{\label{I-B-2}Un musée qui s'inclut dans un ensemble de choix politiques qui lui sont indépendants}

Ici, mon texte



\bigskip
\bigskip
\bigskip

Et ici, une conclusion.



%%%%%%%%%%%%%%%%%%%%%%%%%% CONCLUSION %%%%%%%%%%%%%%%%%%%%%%%%%%

Ici, je pourrai mettre la conclusion de cette partie

%%%%%%%%%%%%%%%%%%%%%%%%%%% PARTIE II %%%%%%%%%%%%%%%%%%%%%%%%%%%%
%%%%%%%%%%%%%%%%%%%%%%%%%%%%%%%%%%%%%%%%%%%%%%%%%%%%%%%%%%%%%%%%%%
	
	\part{La prolifération de l’information en institution culturelle, un sujet facilement mis de côté}
	
%%%%%%%%%%%%%%%%%%%%%%%%%% INTRODUCTION %%%%%%%%%%%%%%%%%%%%%%%%%%

	Ici, je pourrai mettre une introduction de ma première partie
	
%%%%%%%%%%%%%%%%%%%%%%%%%%%%% TEXTE %%%%%%%%%%%%%%%%%%%%%%%%%%%%%%
	
	\chapter[Les vocabulaires contrôlés au MAE]{\label{II-A} Multiplication et fragmentation des vocabulaires au MAE}

\lettrine{I}ci, je pourrai mettre une intro pour mon chapitre

\section{\label{II-A-1}Une construction séparée : 25 ans d'évolution en silo}


Gérer un musée, gérer une bibliothèque ou un projet de recherche, c'est se confronter au savoir, et devoir choisir entre les multitudes de **EXPRESSION** subjectives qui peuvent l'exprimer.
 
Comme de nombreux musées, le MAE a ainsi éprouvé le besoin de contrôler la manière dont il nommerait les objets, fournissant ainsi à ses agents un cadre strict pour décrire et analyser ses collections. Au musée, à la bibliothèque, se sont donc développés, conjointement d'abord puis séparément, des outils de contrôles de ce vocabulaire : des thésaurus qui definiraient à la fois les choix faits par le musée pour écrire les noms d'avions ou de constructeurs d'avions, et les mots qui pourraient qualifier des concepts liés ou non à l'aérospatiale, spécialités du musée. 

\subsection{Histoire des thésaurus} 

• Historique des thésaurus = historique de l’indexation et de ses différents 
logiciels, histoire du rapport du musée au numérique
o Musée d’abord, et 
o Alexandrie (bibliothèque)
o e-médiathèque (récent, dérivé du musée)
• Le défi de l’unification des vocabulaires contrôlés en institutions patrimoniales 
(bib + musée). Créer un thésaurus commun à partir de plusieurs déjà existants et 
construits différemment => pour une cohérence intellectuelle du musée et de sa 
mission de recherche dans l’aéronautique.
• Panorama des logiciels utilisés au musée, des utilisateurs, de la manière 
d’indexer, des relations/ou non-relations entre eux. Axiell, Koha, Skinsoft
• Chiffres et statistiques des thésaurus traités, présentation de leur articulation, 
etc.

\section{\label{II-A-2}Des conséquences importantes : quand la prolifération devient paralysie documentaire}

L'exercice de diagnostic mené sur les thésaurus du \mae révèle une réalité paradoxale : plus l'institution enrichit ses descriptions, plus elle complexifie l'accès à ses propres collections. Cette prolifération de vocabulaires produit aujourd'hui des effets pervers, qui conduisent chargés de collections et documentalistes à s'interroger sur les manières de faire actuelles. L'observation de terrain révèle trois manifestations principales de cette dégradation : l'invisibilisation progressive des collections, la saturation dans l'organisation des équipes, et la difficulté croissante de rationaliser l'accumulation d'informations.

\subsection{L'invisibilisation documentaire : effet direct de la fragmentation des vocabulaires}

L’un des défis relevé à plusieurs reprises dans les échanges avec les agents du \mae est la difficulté croissante à retrouver certains documents ou objets, malgré l’enrichissement constant des vocabulaires. Cette situation ne tient pas seulement à la prolifération des termes, mais à leur organisation disjointe : tout d'abord, la coexistence de thésaurus parallèles et non coordonnés, enferme les informations dans des silos que rien ne relie. Cette fragmentation est notamment étudiée par Richard Gartner et Raphaëlle Mouren dans un article sur la méthodologie mise en place pour éviter ces écueils à la Warburg Institute Library\footcite{gartnerArchivesMuseumsLibraries2019}. Cet article, bien que le contexte technologique ne soit pas exactement le même, se rapproche de la situation rencontrée au \mae : les différences de vocabulaires y sont surtout liées aux différences de conception des métadonnées en général entre les métiers des archives, des musées et des bibliothèques. Bien que le musée n'utilise pas les différents standards utilisés à la Warburg Institute Library, et que ce facteur n'y ait pas autant d'importance, cette analyse s'y applique également : \enquote{There is \textelp{} little interoperability between these three approaches, and consequently between their respective communities, owing to their differing underlying architectures which in turn owe their origins to the very different approaches to metadata that have applied for centuries within each sector. Without this interoperability it is difficult to enable cross-sector or cross-community discoverability, to allow the valuable heritage materials held within, for instance, the library sector to be accessible to users in archives or museums\footnote{\textit{Il existe très peu d’interopérabilité entre ces trois approches, et par conséquent entre leurs communautés respectives, en raison de leurs architectures sous-jacentes distinctes, elles-mêmes issues de conceptions très différentes des métadonnées qui se sont développées depuis des siècles au sein de chaque secteur. En l’absence de cette interopérabilité, il devient difficile de favoriser la découvrabilité entre différents secteurs ou intercommunautaire, et donc de permettre aux utilisateurs des archives ou des musées d’accéder aux précieuses ressources patrimoniales conservées, par exemple, dans le secteur des bibliothèques.}\cite{gartnerArchivesMuseumsLibraries2019}}.} Au \mae, cette situation se concrétise par le fait qu'une information (par exemple, un modèle d'avion rattaché à son constructeur) se retrouvera dans le thésaurus des Aéronefs de l'\gls{emediatheque} et non dans la table des constructeurs de \gls{micromusee}, et le chercheur voulant accéder à la totalité des connaissances détenues par l'institution sur le sujet doit penser aux différents moyens d'accès disponible dans chaque cas.

Or, la manière d'organiser l'information est tout aussi importante que son contenu : dans le cas du \mae où, au fil des années, de nombreuses imprécisions ou erreurs sont restées sans correction, il devient difficile d'accéder à certains documents en utilisant des mots-clés qui seraient pourtant intuitifs. Par exemple, une recherche géographique dans l’e-médiathèque : une photographie indexée sous un nom de ville devrait pouvoir apparaître lors d'une recherche par région. Or, dans la branche \enquote{lieux} des mots-clés de l'\gls{emediatheque}, le terme générique direct est un pays et non une région ; il en résulte qu’une photographie indexée sous « Pont-l’Évêque » rattachée directement à « France » ne sera pas retrouvée par un chercheur interrogeant le fonds sous « Normandie » ou « Calvados ». De même, de nombreux objets techniques, catalogués avec des dénominations précises propres au monde de l’aéronautique, échappent aux requêtes du grand public, qui n’en connaît pas le vocabulaire.

Cette problématique rejoint une exigence ancienne du métier de documentaliste, brillamment formulée par Magdeleine Moureau dès 1968 : 
\begin{quote}
	\og En outre chaque document doit pouvoir satisfaire aux deux objectifs documentaires : diffusion systématique et recherche sur question, et pouvoir restituer le même document lors d'une question générique ou lors d'une question spécifique. Cette possibilité de répondre à plusieurs niveaux pourra s'obtenir de deux façons : soit par l'indexateur humain qui rajoutera pour chaque document particulier le thème général dont il procède, soit par la machine qui associera automatiquement certaines notions génériques à certaines notions spécifiques, par exemple Europe à France ou aromatique à benzène\footcite{moureauProblemesPosesPar1968}.\fg
\end{quote}
On voit ici combien la structuration des vocabulaires, la hiérarchie des concepts et la gestion des synonymies ne sont pas de simples détails techniques : ils conditionnent l’accessibilité même de l’information. L’absence d’harmonisation des thésaurus au sein du musée, l’absence de passerelles entre les corpus, et la dépendance au « bon mot » pour chaque recherche, sont autant de facteurs d’invisibilisation de pans entiers des collections.

Ce cloisonnement n’est pas qu’un problème technique : il traduit une logique institutionnelle, héritée de l’histoire des outils et des métiers du musée. Comme le rappelle la genèse des thésaurus du \mae\footnote{Voir infra, \ref{II-A-1}}, la bibliothèque, le musée et l’e-médiathèque ont chacun développé leur propre vocabulaire selon des besoins spécifiques, sans réflexion globale sur leur articulation. Les documentalistes consultent parfois les thésaurus des collections pour enrichir le leur, mais la démarche inverse reste exceptionnelle. Petit à petit, des fonds documentaires deviennent invisibles pour certains métiers ou pour le public, alors même qu’ils sont rigoureusement indexés.

Cette invisibilisation n’est pas une fatalité : elle invite à repenser la coordination des vocabulaires et à développer des outils de recherche capables de traverser ces silos, en mobilisant les méthodes de normalisation et d’interopérabilité proposées par les dernières normes ISO\footcite{chichereauNormesConceptionGestion2007, hudonISO25964Pour2012a}.


\bigskip
\bigskip
\bigskip

Et ici, une conclusion.
	\chapter[PDV métier]{\label{II-B}Des rôles et une prise de conscience différenciée selon les métiers }

\lettrine{I}ci, je pourrai mettre une intro pour mon chapitre

\section{\label{II-B-1}Cartographie des métiers et émergence d’une conscience documentaire différenciée au \mae}

Avant d’analyser chaque rôle, il importe de rappeler que la gouvernance documentaire au \mae est le produit d’une histoire institutionnelle complexe : la structuration progressive des métiers, la professionnalisation récente des équipes, et la cohabitation d’expertises variées (documentation, collections, informatique) ont forgé un modèle où les sensibilités et pratiques divergent, sans être hermétiques.

\subsection{Les documentalistes : pionniers, vigies et promoteurs de l’interopérabilité}

La question des vocabulaires contrôlés au sein du \mae ne s’est pas imposée d’emblée à l’ensemble de l’institution : elle est née tout d'abord dans le département \enquote{documentation} du musée, chargé de la gestion de la bibliothèque, de l'audiovisuel et des archives. Historiquement, ce sont donc les documentalistes qui furent les premiers artisans d'un vocabulaire contrôlé au \mae, et les premiers sensibilisés à cette question.

En effet, dès la fin des années 1990, la création du thésaurus de la bibliothèque, alors géré dans le logiciel \gls{alexandrie}, s’inscrit dans une tradition solidement ancrée de bibliothéconomie. Les documentalistes, formés à la rigueur des normes AFNOR et ISO, et nourris de l’expérience des grands réseaux nationaux tels que le SUDOC ou la \ac{bnf}, ont d’emblée perçu l’enjeu stratégique de l’interopérabilité : il s’agissait, au-delà du seul catalogage, d’assurer la visibilité du fonds documentaire, de construire des listes d’autorité fiables, de garantir la pérennité et la communicabilité des savoirs, comme l'a présenté Blandine Nouvel dans un atelier \enquote{Thésaurus appliqués} pour le projet Bibracte, ville ouverte\footcite{nouvelOutilsDindexationBibliothecaires2022}. Les documentalistes en France ont en effet toujours été des moteurs des projets de normalisation de la gestion de l'information -- dans les thésaurus en particulier -- et ont réclamé et applaudi les travaux internationaux menés sur le sujet\footnote{Il convient en effet de citer par exemple l'article \citetitle{chichereauNormesConceptionGestion2007} publié dans Documentaliste-Sciences de l'Information, qui insiste sur l'intérêt de la nouvelle norme ISO en cours de travail et fait un panorama des existantes, avec leurs qualités et défauts.}.

Au \mae, ce rôle éminemment technique et intellectuel s’est affirmé lors des premiers comités de pilotage destinés à harmoniser les vocabulaires entre la documentation et les collections muséales — instances où le dialogue entre les métiers permit d’esquisser les premiers jalons d’une gouvernance documentaire véritable\footnote{Voir historique des thésaurus du musée, section \textit{\hyperref[II-A-1]{\nameref{II-A-1}}}}, bien que ce dialogue n'ait pas perduré. La pratique quotidienne du métier, l’attention portée aux métadonnées, la maîtrise des principes de normalisation et d’interopérabilité ont en effet fait des documentalistes les membres naturels et incontournables de ces groupes de travail, contrairement par exemple aux chargés de collections dont la présence n'était que facultative.

L’année 2025 marque un tournant : la migration vers \gls{koha} et la plateforme \gls{clade}, orchestrée par le \minarm, met en lumière la fonction d’interface que les documentalistes exercent entre les exigences institutionnelles et la réalité du terrain. Ils négocient l’intégration du thésaurus, soulignent les erreurs d’import et défendent la nécessité d’une structuration fine des termes pour la recherche — combat quelquefois solitaire face aux exigences ministérielles et aux impératifs de la technique\footnote{Voir les difficultés rencontrées lors de la migration vers la plateforme Koha, mentionnées en section \textit{\hyperref[I-B-2]{\nameref{I-B-2}}}}. Cette période voit également l’organisation d’ateliers de sensibilisation à la gouvernance de l'information, menés sous l’impulsion de la responsable du \ac{drd} : celle-ci rédige des notes et promeut l’acculturation du \ac{dsc} à la notion de thésaurus partagé, illustrant ainsi la dimension pédagogique et fédératrice du métier.

Cette prééminence des documentalistes s’explique donc par la spécificité de leur formation, leur acculturation à l’interopérabilité et la nécessité institutionnelle de rendre visible le fonds auprès d’un public hétérogène. Mais si leur action a posé les fondements d’une gouvernance documentaire digne de ce nom, elle est longtemps demeurée cantonnée au périmètre de la documentation, faute de pouvoir imposer une politique véritablement transversale à l’ensemble du musée. Ainsi, les documentalistes demeurent — encore aujourd’hui — les pionniers et les gardiens vigilants d’un chantier intellectuel qui attend d’être pleinement partagé.

\subsection{La gestionnaire de \gls{bdd}, pivot documentaire et révélateur des limites institutionnelles}

La figure du gestionnaire de \gls{bdd} s’est imposée au \mae comme l’un des pivots discrets mais essentiels de la cohérence documentaire, et ce bien avant que la notion de gouvernance de l’information ne s’invite dans les débats institutionnels. L’avènement du logiciel \gls{micromusee} au tournant des années 2000, puis son remplacement par \gls{archange} en 2025, ont doté le musée d’outils de gestion qui lui ont permis de gérer le vocabulaire de manière unifiée, et qui sont devenus le terrain d’un travail quotidien pour la construction et la transmission des référentiels au sein du \ac{dsc}.

En ce qui concerne la gouvernance de l'information, le gestionnaire de \gls{bdd} exerce une mission de veille et de correction, que la migration vers \gls{archange} a rendu visible aux yeux des autres métiers. Il ou elle intervient en modifiant les terminologies, traquant les incohérences issues du logiciel, et fédérant au niveau du musée les initiatives de chacun. Détenteur des droits de modification et de création, ce technicien de l’information devient en effet l’interlocuteur privilégié des chargés de collections, l'arbitre des enrichissements et des corrections, notamment lors de l’introduction de nouveaux matériaux ou domaines.

La grande migration de \gls{micromusee} vers \gls{archange} pour les collections muséales a joué un rôle révélateur pour le gestionnaire de \gls{bdd}. Ce chantier, en effet, a mis en lumière à la fois l’expertise technique requise et les limites de l’outillage institutionnel : \gls{micromusee} v6, dont il a été fait mention plus haut, illustre à lui seul le fossé qui sépare les besoins intellectuels du musée et les solutions techniques qui lui ont été offertes jusqu'ici. Le travail laborieux réalisé lors de la migration -- analyses des exports, détection des doublons et correction manuelle des incohérences -- a rendu manifeste la nécessité de revisiter, de rationaliser et d'harmoniser les thésaurus existants au sein du musée.

La responsable \gls{bdd}, aidée d'un expert en informatique, a ainsi dû s'atteler à l’examen minutieux des exports issus de \gls{micromusee} : plus de 16\,000 termes, dont l’analyse a révélé la présence de doublons persistants dans certaines tables, de termes privés de rattachement hiérarchique ou mal positionnés\footnote{Cela est le cas par exemple de la liste d'autorités des noms propres de la base \gls{micromusee}/\gls{archange}, qui a révélé après examen que de nombreuses personnes physiques avaient été positionnées sous personne morale, et inversement. Pour corriger ces incohérences, la gestionnaire a entamé un travail sur le long terme de corrections par petits lots des termes rencontrés au fil des recherches.}. La gestionnaire de \gls{bdd} est ainsi devenue le premier acteur du nettoyage et de la réorganisation des termes, dans un travail laborieux, long, mais nécessaire, par petites séances de correction.

Son rôle consiste également à gérer d'éventuels conflits ou besoins dans les terminologies : par exemple par l'ajout ou l'organisation des noms de matériaux utilisés pour décrire les collections. Ici, le gestionnaire de \gls{bdd} a organisé des réunions en collaboration avec des chargés de collection, pour arbitrer entre synonymes concurrents, intégrer les termes retenus, les documenter, dans un tableur permettant de visualiser l'ensemble du thésaurus. Ce travail de révision n’a pas seulement consisté à corriger des erreurs : il a nécessité une restructuration des branches du thésaurus, la suppression de doublons ou de termes inutiles, et un travail considérable de recherche pour bien positionner les termes et mettre le vocabulaire du \mae en accord avec les recommandations nationales\footnote{Il s'agit notamment des thésaurus de Joconde mis à disposition par le ministère de la Culture, qui seront développés plus bas.}.

Les projets de migration et les outils implémentés au musée jouent un rôle primordial dans la sensibilisation du métier à la question du contrôle du vocabulaire : par exemple, l’inadéquation de \gls{micromusee} a contraint à recourir à des opérations de migration fastidieuses -- conversions et traitement en CSV, visualisations dans Gephi, adaptation manuelle des formats -- et à inspecter plus en détail la composition des thésaurus présents. Le passage vers \gls{archange}, quant à lui, et les nombreuses fonctionnalités qu'offre la plateforme pour gérer les vocabulaires, est à l'origine de nouveaux projets d'organisation de celui-ci\footnote{La plateforme offre par exemple la possibilité de renseigner les liens familiaux entre différentes personnes de la base : cette fonctionnalité a été relevée comme pouvant être très intéressante pour mettre en valeur les connaissances historiques du musée sur l'histoire et les généalogies d'aviateurs ou de constructeurs d'avions. L'entrée de ces nouvelles informations -- qui se ferait donc uniquement dans la base de gestion des collections, sur le référentiel des personnes, l'éloignerait d'une structure de thésaurus ou de liste de vocabulaire pour le rapprocher d'une ontologie, solution lourde à implémenter mais qui s'est révélée tout au long du stage comme une solution plus complète pour mettre en valeur l'ensemble des connaissances scientifiques du \mae.}.

L’analyse révèle que la prise de conscience du gestionnaire de \gls{bdd} est directe, pragmatique, et fondée sur l’expérience du terrain : la stabilité des référentiels et la cohérence des notices ne sont pas des enjeux théoriques, mais la condition d’une gestion quotidienne efficace. Pourtant, ce rôle fondamental reste le plus souvent cantonné à la résolution de problèmes ponctuels, sans pouvoir s’étendre à une réflexion globale sur la gouvernance documentaire, laquelle ne saurait être menée sans l’implication de l’ensemble des métiers concernés. Les grandes migrations informatiques quant à elles, loin d’être de simples opérations techniques, jouent le rôle de révélateur des failles du système et amorcent des dynamiques de rationalisation, en exposant à tous -- techniciens comme conservateurs -- la nécessité d’une réforme des pratiques et d’une harmonisation des vocabulaires.

\subsection{Les chargés de collections : diversité des pratiques, entre nomenclature fine et catégories souples}

La question du vocabulaire contrôlé demeure périphérique dans le quotidien des chargés de collections. Loin d’être anecdotique cette situation interroge : pourquoi, alors que l’exigence de normalisation s’impose de toutes parts, la structuration lexicale peine-t-elle à s’ériger en préoccupation centrale chez ceux-là mêmes qui sont les dépositaires de la mémoire matérielle ?

\medskip

Tout d'abord, il faut reconnaître que la sensibilité des chargés de collections à l’égard du \gls{thesaurus} contrôlé procède avant tout d'un usage ponctuel : elle s’éveille à l’occasion d’une exposition, de l'enrichissement d'un fonds, ou lorsque surgit une difficulté d’indexation qui résiste aux outils habituels. Formés dans des cursus techniques ou muséaux, ces professionnels cultivent un rapport particulier à l’objet : ils privilégient l’expérience de la matérialité, la restitution de l’histoire singulière, le dialogue avec le contexte d’origine. Leur premier instinct sera, dans toute description, de préserver l'épaisseur du réel pour transmettre toute la richesse scientifique, historique et sociale de l'objet. Ces particularités du rapport à l'information selon les métiers n'ont rien de nouveau, et elles ont déjà été relevées par des chercheurs comme Maryse Rizza : par exemple, l'endroit premier où se retrouve le \gls{thesaurus} pour un chargé de collection est l'inventaire -- et comme la chercheuse l'a exprimé dans \citetitle{rizzaDocumentAuCoeur2014}, \enquote{cet inventaire, base de la production documentaire, sera utilisé de manière différente selon le rôle et la place de l’acteur dans l’organisation muséale. Le conservateur, par exemple, privilégiera la fonction scientifique du document d’inventaire pour produire son analyse et rédiger des notices et/ou des commentaires qui ont pour vocation d’enrichir la connaissance historique du patrimoine muséographique}, tandis que pour le régisseur des collections, c’est la \enquote{fonction technique du document d’inventaire qui primera sur ses usages\footcite{rizzaDocumentAuCoeur2014}}.

\medskip

On ne saurait cependant généraliser cette tendance à l'ensemble des chargés de collection : au sein du corps de métier, différentes approches existent selon les personnalités et les champs d'études : au musée par exemple, les héritiers d’une tradition anthropologique ou technique, privilégieront une nomenclature d’une extrême finesse, où chaque objet, chaque matériau appelle une granularité lexicale adaptée à son contexte intellectuel et historique. C'est dans leur branche que se fait l'essentiel de l’enrichissement du \gls{thesaurus} des domaines — utilisé pour la description des collections du musée — devient le lieu d’une quête de précision, d’une volonté de restituer la complexité du monde technique sans céder aux simplifications. À l’inverse, d’autres optent pour une approche plus souple, mobilisant des catégories génériques et des mots-clés fonctionnels, jouant sur la plasticité des termes pour affiner le classement sans enfermer l’objet dans une taxonomie rigide : ceci est visible dans les branches plus techniques du même \gls{thesaurus}\footnote{cf. \textit{\hyperref[fig:model_domaines]{\nameref{fig:model_domaines}}}}. Ces évolutions de la structuration du \gls{thesaurus} des domaines, qui oscille entre logique technique et logique anthropologique, illustrent ces arbitrages permanents : faut-il privilégier la granularité ou l’opérabilité ? La fidélité au terrain ou l’harmonisation documentaire ?

La prolifération des vocabulaires au sein du \acf{mae}, loin d’être le simple effet d’une accumulation documentaire non maîtrisée, révèle en profondeur la difficulté à penser collectivement la mémoire d’une institution technique. Fragmentation des outils, silos de métiers, absence de coordination : autant de symptômes d’un malaise documentaire qui invisibilise les collections et entrave la transmission du savoir. Une volonté d'organisation existe tout de même derrière cette apparente dispersion.

\bigskip

Ce constat invite à dépasser le seul diagnostic technique pour interroger la part humaine des pratiques d’indexation : car la structuration de l’information, au musée, ne se joue pas uniquement dans les logiciels ou les schémas de \gls{thesaurus}, mais dans la diversité des usages, des sensibilités et des compétences qui traversent les métiers. Dès lors, il est nécessaire de comprendre comment chaque acteur, selon sa formation, son rapport à l’objet ou à la documentation, s’approprie — ou délaisse — les vocabulaires contrôlés : c’est cette cartographie des rôles, des résistances et des prises de conscience différenciées que la prochaine section se propose d’explorer, pour mieux saisir les conditions d’une gouvernance informationnelle renouvelée.
\section{\label{II-B-2}Des différences d’appréhension qui traduisent un rapport très différent à l’information}

Toujours pour décrire des œuvres, mais d’une manière différente : priorité différente accordée au contexte de l’œuvre, matérialité, usage, etc... 

Des besoins et des réalités différentes en matière d’interopérabilité (ex. De C-ADER, le thésaurus d’un programme de recherche, intéressant pour les missions du musée mais qui a une vie totalement à part et zéro besoin ni exigence de ce côté-là). 

Dans l’état actuel de la recherche, peu de solutions proposées / réflexions sur cet aspect propre du problème, même si chaque branche explore bien de son côté la question des vocabulaires et de l’interopérabilité. Pourtant, il n’est pas rare d’avoir musée et bibliothèque au même endroit, unité intellectuelle dans les missions de l’institution. 

Relation musées - bibliothèques - archives : des mondes séparés physiquement, intellectuellement, malgré similitude et intégration dans une même mission.

\bigskip
\bigskip
\bigskip

Et ici, une conclusion.
	\chapter[Les archives numériques]{\label{II-C}Les archives numériques : une autre manifestation des difficultés de gestion de l’information en musée }

\begin{quote}
	\enquote{\textbf{ARCHIVES INFORMATIQUES} \textit{en : electronic records}
		Documents produits ou reçus par un organisme dans l'exercice de
		ses activités et conservés sous forme d'enregistrements
		électroniques sur des supports tels que les bandes magnétiques,
		les disques magnétiques, les disques optiques etc., et qui ne
		peuvent être lus que par l’intermédiaire d’une machine.}
\end{quote}

\lettrine{D}{ans} le musée, la question des archives numériques est à la croisée des chemins : entre mémoire documentaire et gestion administrative, elle met en lumière l’ambiguïté qui sépare – ou confond – archives et documentation. Cette frontière est souvent difficile à définir dans le milieu des musées et, loin d’être purement technique ou juridique, elle recouvre des enjeux de gouvernance, de responsabilité et de pérennité du savoir dans un contexte où la masse des données numériques croît plus vite que ne se structurent les pratiques. L’expérience du \mae, qui se révèle dans son organigramme comme dans sa pratique quotidienne – où l’unique archiviste ne peut déjà embrasser la diversité des fonds papier – illustre les dilemmes contemporains de l’archivage en musée. Comment garantir la pérennité et la communicabilité des archives publiques numériques d'un musée lorsque la distinction entre archive et document est floue, quand l’outil et les référents professionnels manquent, et quand la réflexion collective reste à construire de zéro ? 

Dans une première partie, nous tâcherons d'examiner les ressorts de cette frontière incertaine entre archives et documentation qui existe en contexte muséal, en interrogeant les pratiques, les outils et les propositions institutionnelles pour tenter d'y pallier. Puis, dans une seconde partie, nous porterons l’analyse sur la figure de l’archiviste en musée, dont la légitimité s’avère difficile à établir : nous mettrons en lumière les enjeux techniques, juridiques et professionnels qui entravent la mise en place d’une véritable politique d’archivage raisonné, et les difficultés spécifiques que soulève la gestion quotidienne des archives numériques par des agents souvent peu formés et insuffisamment sensibilisés à la gouvernance documentaire.

%\lettrine{D}{ans} le paysage contemporain des institutions patrimoniales, la question des archives numériques se dresse avec une acuité nouvelle, révélant en creux les tensions qui traversent le quotidien des musées techniques. Au Musée de l’Air et de l’Espace, où la mémoire documentaire s’entrelace étroitement avec la gestion courante, la technicité des outils informatiques se heurte à la faiblesse des moyens humains, et la pratique quotidienne à la rigueur réglementaire. La frontière — déjà ténue — entre archives et documentation s’y révèle d’autant plus poreuse que l’institution ne dispose que depuis peu d’une archiviste dédiée. Il importe alors, pour comprendre les logiques qui entravent la mise en œuvre d’une gouvernance archivistique digne de ce nom, d’interroger la dialectique entre impératifs institutionnels, contraintes techniques, et réalités humaines, sous le prisme de la littérature professionnelle et du vécu des agents.
\section{La frontière floue entre archives et documentation : une problématique propre aux musées techniques}

\subsection{Définir l’archive numérique en contexte muséal : enjeux et ambiguïtés}

%Les musées en France, institutions dédiées à la conservation et à la diffusion du patrimoine, sont intimement liés à la question des archives : eux-même producteurs d'archives, ils peuvent être également exceptionnellement appelés à en être les dépositaires lorsqu'elles sont indissociables d'une oeuvre. Malgré la prégnance des problématiques liées à la gestion des archives dans les musées dans le paysage français, la littérature scientifique et professionnelle sur ce sujet semble peu prolifique : l'article de référence que nous retiendrons pour ce développement est celui de Véronique Sassetti-Aguilera, \citetitle{sassetti-aguileraArchivesMuseesDiversites2020a}. L'autrice identifie trois problématiques principales dans les musées actuellement : \enquote{La première difficulté réside dans le statut même de l'établissement (public ou privé) qui caractérise le plus souvent la nature des fonds d'archives et des collections [...] La deuxième, nettement plus complexe, relève de l'identification des archives par le personnel des musées comme trésors nationaux, imprescriptibles et inaliénables [...] La troisième enfin, et non la moindre, demeure dans l'enrichissement continuel de la documentation scientifique des collections de musées, situation ne permettant jamais [...] de procédures d'archivage classiques par bordereaux de versement.}. Ces problématiques se retrouvent au \mae. Celui-ci est en effet une institution publique, l'ensemble de ses archives sont donc est celle de la frontière : où finit la documentation, où commence l'archive ? Cette interrogation, classique en archivistique, se complique singulièrement dans l'environnement numérique muséal, où la nature hybride des fonds — documents de gestion, dossiers d'œuvre, photographies, fichiers bureautiques, bases de données — brouille irrémédiablement les distinctions traditionnelles.

\begin{itemize}
	\item \textbf{Problématique} : Qu’est-ce qu’une archive numérique en musée ? En quoi la nature hybride des fonds (documents de gestion, dossiers d’œuvre, photographies, fichiers bureautiques, bases de données) brouille-t-elle la distinction entre documentation courante et archives ?
	\item \textbf{Pistes d’analyse} : 
	\begin{itemize}
		\item Définitions normatives (\textit{archive courante, intermédiaire, définitive}, cf. Code du patrimoine, FICHE\_Information\_Archivistique.md).
		\item Spécificité muséale : la documentation devient archive dès lors qu’elle sert à la mémoire du musée ou à la justification de ses droits.
		\item Les dossiers d’œuvre : archives ou documentation ? \footcite{barbelinDossierDoeuvreDossier2016a}
		\item Données numériques et patrimonialisation de l’information technique \footcite{bechardArchivesElectroniques2020a}
	\end{itemize}
	\item \textbf{Exemples concrets} : Dossiers d’œuvre (archives-administration, rapports, photographies, courriels, cf. FICHE\_Information\_Archivistique.md), photographies de la médiathèque, rapports d’activité.
	\item \textbf{Citation} : \og Les dossiers d’œuvre sont des archives publiques : ils sont librement communicables à tout citoyen. \fg \footcite{barbelinDossierDoeuvreDossier2016a}
\end{itemize}

\subsection{Panorama des pratiques de gestion des archives numériques au MAE}

\begin{itemize}
	\item \textbf{Problématique} : Où et comment sont conservés les fichiers numériques ? Quelles logiques d’organisation ? Quelles difficultés concrètes ?
	\item \textbf{Pistes d’analyse} :
	\begin{itemize}
		\item Cartographie des espaces de stockage : Serveur S (archives courantes, arborescence par service, gestion Windows Explorer), SharePoint, OneDrive.
		\item Hétérogénéité des usages et compétences, conséquences sur la pérennité.
		\item Problèmes récurrents : arborescence trop ou trop peu granulaire (cf. FICHE\_RAPPORT\_ServeurS.md), doublons, conventions de nommage non pérennes (nuage de mots, FICHE\_Bonnes\_Pratiques.md), noms au nom des agents (cf. FICHE\_Photos.md), absence de plan de classement partagé.
		\item Conséquences : invisibilisation, perte de repères, saturation.
	\end{itemize}
	\item \textbf{Exemples concrets} : Ratio fichiers/dossiers, arborescence profonde (16 niveaux, cf. FICHE\_RAPPORT\_ServeurS.md), nommage des photos et vidéos (cf. FICHE\_Photos.md), dossiers « vrac » (FICHE\_Bonnes\_Pratiques.md).
	\item \textbf{Citation} : \og Un bon classement = une information trouvable, fiable et partageable durablement. \fg (FICHE\_Bonnes\_Pratiques.md)
\end{itemize}

\subsection{Un personnel sous-dimensionné : la difficile reconnaissance de la fonction archivistique}

\begin{itemize}
	\item \textbf{Problématique} : Pourquoi la fonction archivistique reste-t-elle marginale dans les musées techniques ? Quelles conséquences sur la gestion des archives numériques ?
	\item \textbf{Pistes d’analyse} :
	\begin{itemize}
		\item Une seule archiviste au MAE depuis 2024, principalement dédiée aux archives privées papier.
		\item Rattachement à la documentation : dilution des missions, surcharge, arbitrages constants.
		\item Absence de politique de formation et de sensibilisation des personnels non archivistes (cf. FICHE\_Information\_Archivistique.md).
		\item Difficulté à établir plan de classement, tableau de gestion, chaîne archivistique.
	\end{itemize}
	\item \textbf{Exemples} : Missions d’archives ponctuelles, gestion dossiers d’œuvres, documentation.
	\item \textbf{Citation} : \og La distinction entre archives et documentation est remarquablement posée, [...] la gestion purement administrative des dossiers d’acquisition permet une uniformisation des pratiques. Peut-on à partir de cet exemple réfléchir à l’élaboration de procédures spécifiques en matière d’archives publiques de musées en France ? \fg \footcite{sassetti-aguileraArchivesMuseesDiversites2020a}
\end{itemize}


\section{\label{II-C-2}Vers une gouvernance des archives numériques au \mae : expérimentations, structuration et montée en compétences}


\begin{myquote}
	{Ne pas envisager l’archivage de documents numériques dès leur création, c’est prendre le risque de les perdre définitivement.}{ministeredelacultureArchivesElectroniquesDans2020}
\end{myquote}

\subsection{Quelques leviers}


Loin de se réduire à une succession d’obstacles, la situation des archives numériques au musées est aussi un terrain privilégié pour tenter une réconciliation des dynamiques muséales et archivistiques. Les recommandations nationales – garantie de l'intégrité du fichier et de ses métadonnées, distinction des états d’archives, traçabilité, bordereaux d’élimination – sont peu à peu reconnues, d'abord par le \ac{drd}, sensibilisé par ses missions aux enjeux de gestion de l'information, puis par les agents du \ac{dsc} qui y sont confrontés au quotidien.



Certes, la question du versement définitif demeure ouverte : aucun projet d'intégration immédiate à un \gls{sae} ne paraît envisageable, mais la problématique est reconnue, et il faut reconnaître que la priorité est donnée pour l'instant aux chantiers logiciels menés par le \minarm (\gls{koha}-\gls{clade} et \gls{archange}), reléguant la question de l’archivage à une temporalité indéfinie. La sensibilisation des agents se heurte à la dispersion des responsabilités et la fonction archivistique demeure fragile, mais des quelques avancées pour poser les bases de projets ultérieurs plus conséquents ont été proposées au \ac{dsc}, qui pourrait alors, après une première expérimentation, argumenter pour un chantier commun à l'ensemble du musée.


La première, et l'une des missions secondaires de ce stage, a consisté en la production de moyens de sensibilisation des agents du \ac{dsc} aux enjeux de gouvernance de leurs archives numériques : ceux-ci auront consisté principalement en 
\begin{itemize}
	\item la réalisation de formations de sensibilisation à l’adresse des agents,
	\item la rédaction d’une Foire aux Questions sur la gestion quotidienne et les grands principes d'archivistique[TODO : ajout annexe faq caviardée],
	\item la production de fiches de bonnes pratiques.
\end{itemize}

La seconde, liée à la précédente, est le recueillement des ressentis et des pratiques déjà en vigueur dans le service, afin de proposer des solutions réalistes et adaptées aux exigences métier : cette démarche a abouti à la proposition de normes de nommages de fichiers et de dossiers, et de structuration de l'arborescence générales au service et à la création d'un document partagé de mise en commun du vocabulaire et des abréviations utilisées.


Il s’agit ici d’insuffler une culture commune de la gestion documentaire : nommage, tri, organisation, bordereaux d’élimination sont pensés comme autant de jalons vers une maîtrise accrue du cycle de vie des fichiers. Face à l’impossibilité actuelle d’un archivage définitif, l'engagement d'un professionnel dédié au tri du serveur, à la validation des éliminations et à la mise en conformité des procédures est souhaitable ; dans l'attente d'une solution institutionnelle, une réflexion s’est engagée sur la création d’un dossier réservé à l'archivage des documents arrivés à leur fin d'utilité administrative.


Pour préparer le terrain d'une gestion robuste de ses archives, ces initiatives, portées par le \ac{dsc} illustrent la possibilité d’entamer une dynamique de modernisation fondée sur la formation, l’accompagnement et l’engagement des agents, même en l’absence de dispositifs techniques pleinement aboutis.

\bigskip
\bigskip
\bigskip

La crise de l’archivage numérique en musée ne se résume pas à une accumulation de défauts : elle révèle également la capacité de l’institution à inventer des solutions pour organiser sa gouvernance en conformité avec les recommandations en vigueur et former ses agents. Là encore, ce sont les métiers de la documentation, plus sensibilisés aux enjeux de la gouvernance de l'information en général qui sont à l'initiative de ces projets.

\bigskip
\bigskip
\bigskip

\lettrine{S}{ynthétiser} la crise de l’archivage numérique en musée, c’est reconnaître la fragmentation des espaces, la gouvernance éclatée, le sous-dimensionnement humain, et l’absence de solutions pérennes. Le cas muséal, singulier, se situe à la croisée des chemins : entre documentation et archives, le musée est sommé d’inventer une mémoire numérique qui soit fidèle à son histoire et adaptée à ses contraintes.

La nécessité d’une politique globale de gouvernance de l’information s’impose, pour que la mémoire numérique du MAE ne se dissolve pas dans le flux des fichiers et des tâches. Structuration, concertation, formation, sont les jalons à poser pour qu’à l’avenir, la mémoire du musée ne devienne pas l’otage de l’urgence, mais le reflet d’une institution soucieuse de son passé autant que de son devenir.

%%%%%%%%%%%%%%%%%%%%%%%%%%% CONCLUSION %%%%%%%%%%%%%%%%%%%%%%%%%%%
	
Ici, je pourrai mettre la conclusion  de cette partie.
	
%%%%%%%%%%%%%%%%%%%%%%%%%%% PARTIE III %%%%%%%%%%%%%%%%%%%%%%%%%%%
%%%%%%%%%%%%%%%%%%%%%%%%%%%%%%%%%%%%%%%%%%%%%%%%%%%%%%%%%%%%%%%%%%
	
	\part{Gérer la prolifération. Outils et méthodes}
	
	
%%%%%%%%%%%%%%%%%%%%%%%%%% INTRODUCTION %%%%%%%%%%%%%%%%%%%%%%%%%%
	
	Ici, je pourrai mettre une introduction de ma première partie
	
%%%%%%%%%%%%%%%%%%%%%%%%%%%%% TEXTE %%%%%%%%%%%%%%%%%%%%%%%%%%%%%%
	
	\chapter[Gérer la prolifération]{\label{III-A} Gérer la prolifération : outils et méthodes}



\lettrine{I}ntro

\section{\label{III-A-1}La représentation conceptuelle : modéliser pour maîtriser la complexité}

\subsection{Les enjeux de la modélisation conceptuelle}

\paragraph*{Visualiser pour comprendre : le dévoilement des structures cachées}
Comme démontré plus haut dans la rapide présentation des vocabulaires utilisés au \mae, la masse de données à traiter est considérable\footnote{Voir le tableau \refinterne{tab:thesaurus_synthese}}. Ces données se présentent toutes sous la forme de tableurs excel exportés des différents logiciels de gestion du vocabulaire, et analyser leur structure et en déduire une méthodologie à suivre pour les optimiser est un réel défi pour le cerveau humain. Avant tout projet d'unification, visualiser les données pour pouvoir mieux les appréhender s’impose donc comme un nécessité : en effet, en les mettant à plat les et en permettant à l'utilisateur de les visualiser et de les embrasser dans leur ensemble, il est possible de rendre plus manifeste la structure des connaissances, de faire apparaître ce qui se cache dans l’enchevêtrement des termes, des usages et des hiérarchies.

\paragraph*{La modélisation comme instrument de gouvernance intellectuelle}
Lorsque fut lancé au \mae~le projet de rationalisation des \gls{thesaurus} et vocabulaires contrôlés du musée, dont notre stage a été l'un des premiers jalons, la première étape, après avoir pris connaissance des données -- mais également pour mieux les connaître -- a immédiatement consisté en la réalisation de graphes d’arborescence pour chaque lexique, à l’aide du logiciel \textit{Gephi}, de représentations générales de la structure de ces vocabulaires en UML ou sous forme de \textit{mindmaps} afin de visualiser et comparer leurs structures générales. C'est cette visualisation qui a notamment permis de révéler l’existence de termes orphelins, de clusters thématiques isolés, ou d’incohérences de rattachement qui rendaient difficile l'accès à certains termes. [TODO : voir figure à ajouter à la fin du mémoire.] Le travail de modélisation a ainsi permis de remettre au jour ces impasses sémantiques et d’envisager une restructuration cohérente.

Gérer l'information, en musée ou ailleurs, revient à une forme de gestion des données : comme le démontrent les articles de projets en institution patrimoniales, toute manipulation de données à grande échelle sous-entend cette étape. Les choix de modélisations, les outils et les manières de faire sont cependant extrêmement diverses et peuvent varier d'un projet à l'autre : si, au musée, le choix a été fait de commencer par ces visualisations manuelles afin de déterminer une base de travail pour les projets ultérieurs, la plupart des projets en institutions patrimoniales se sont tournés vers des standards universels comme le \gls{rdf}, qui permettent notamment de générer des graphes aisément manipulables et explorables après conversion\footnote{Sur l'utilisation du \gls{rdf} en institution patrimoniale, voir notamment \cite{bermesCasLierDonnees2013a,bermesConvergenceInteroperabiliteLapport2011,filabesLindexationRAMEAUAssistee2025,reichThesaurusAuGraphe2022}}.

La modélisation en effet n'est pas seulement un outil technique, mais un véritable instrument de gouvernance intellectuelle. Clarifier les hiérarchies, identifier les termes orphelins ou doublons, faciliter le dialogue entre métiers : la modélisation conceptuelle prépare le terrain à une transmission des savoirs, et à la constitution d’un langage commun qui transcende la division des métiers et des utilisateurs.

\subsection{Outils et méthodes de modélisation}

\paragraph*{Typologie des outils : entre rigueur formelle et accessibilité}
La typologie des outils disponibles est désormais bien établie. Les diagrammes UML qui sont aujourd'hui devenus des standards dans la formalisation des systèmes d'information, s'imposent par leur rigueur et leur capacité à modéliser des relations complexes. Dans notre cas, ils permettent notamment de distinguer nettement les différents types de vocabulaires : thésaurus de mots-clés, listes d'autorités pour les événements, les personnes, ou encore les lieux. Leur formalisme offre une grande cohérence et une intéropérabilité précieuse : la norme ISO 25964 est ainsi accompagnée d'une modélisation en UML du thésaurus idéal et aux normes. En mettant en miroir cette modélisation avec l'une tirée des données du musée, il est ainsi possible de repérer les divergences d'avec les normes, de voir ce qui pourrait être ajouté à l'existant, en bref, de montrer plus concrètement ce qui pourrait être réalisé avec l'existant grâce à cette norme\footnote{Voir en l'annexe \refinterne{Ax-F}}.

Pour autant, l'expérience du terrain montre que ces modèles, s'ils sont puissants, ne sont pas toujours adaptés à la diversité des publics et des usages. Les schémas UML nécessitent en effet une compréhension des codes utilisés : jugés « obscurs » ou trop abstraits par les agents de conservation ou les usagers non spécialisés, ils se sont révélés de moindre utilité que d'autres formats. 


\begin{figure}[htbp]
	\centering
	\begin{subfigure}{0.50\textwidth}
		\centering
		\includegraphics[width=\linewidth]{img/MODEL_emediatheque_mindmap}
		\caption{Modélisation en \textit{mindmap} réalisée avec \textit{draw.io}}
		\label{model:mindmap-emediatheque}
	\end{subfigure}
	\begin{subfigure}{0.35\textwidth}
		\centering
		\includegraphics[width=\linewidth]{img/MODEL_emediatheque_arbre}
		\caption{Modélisation en arbre html interactif réalisée avec \textit{markmap}}
		\label{model:arbre-emediatheque}
	\end{subfigure}
	\hfill
	\begin{subfigure}{0.5\textwidth}
		\centering
		\includegraphics[width=\linewidth]{img/GRAPH_emediatheque}
		\caption{Modélisation en graphe réalisée avec \textit{Gephi}}
		\label{model:graph-emediatheque}
	\end{subfigure}
	\caption[Modélisation des \gls{thesaurus} du \mae]{Trois types de modélisations pour un même thésaurus offrent trois approches différentes : notamment, la première permet une vue d'ensemble rapide du type de contenu, la seconde permet d'explorer les branches les unes après les autres, la troisième permet de visualiser des \textit{clusters} de mots -- et pourrait également représenter leurs relations.}
	\label{fig:modelisationthesaurus}
\end{figure}


\paragraph*{Graphes, mindmaps, arbres : l’approche pragmatique}
La visualisation des données par le biais de graphes générés avec Gephi ou d'arbres de concepts interactifs\footnote{Ces arbres, qui auraient également pu être réalisés en Python à l'aide bibliothèques comme \textit{pandas} ou \textit{matplotlib}, ont été réalisés à des fins d'illustration en convertissant un tableau excel en format \textit{markdown} et en générant un fichier html à l'aide de l'application web markmap.} offre une alternative plus intuitive et immédiatement opératoire. Au MAE, ces graphes – qui ont été ensuite mis à disposition de tous les utilisateurs – ont permis de cartographier les clusters thématiques, de détecter les termes orphelins, d'identifier des incohérences dans la hiérarchie des vocabulaires ou des disproportions dans certaines branches du thésaurus. Ils ont servi de support à la restructuration du thésaurus, mais aussi d'outil de médiation entre métiers, lors des groupes de travail ou des ateliers de sensibilisation.

Le format le plus simple -- et donc le plus utilisé, même si moins riche -- a été la modélisation sous forme de \textit{mindmap}, pour représenter seulement les niveaux les plus hauts d'une hiérarchie et travailler sur des grandes catégories de concepts. Celui-ci,  réalisé à partir de logiciels en ligne comme \textit{draw.io}, s'utilise facilement en combinaison avec les tableaux bruts du thésaurus.

On retrouve ici une tension propre à la modélisation documentaire en institution patrimoniale : le compromis entre la formalisation technique – garante de l'interopérabilité et de la pérennité – et la souplesse nécessaire à la prise en main par des agents aux profils variés. Combiner les différents schémas permet ainsi de montrer différents aspects de l'état de l'information au musée, mais aussi de parler à différents profils, et de répondre aux biais, omissions ou transformations inévitables dans toute modélisation et représentation du savoir\footcite{bowkerArrangerChosesConsequences2023}

Ainsi, la modélisation documentaire au \mae~– et plus largement dans les institutions patrimoniales – ne saurait se réduire à un exercice technique. Elle est un instrument de dialogue, d'analyse et de gouvernance, dont la réussite dépend de la capacité à articuler exigence conceptuelle et pragmatisme métier, rigueur des normes et souplesse des usages, pour faire du vocabulaire documentaire un espace partagé, vivant et évolutif.

\begin{table}[htbp]
	\caption{Comparatif des visualisations de données utilisées au MAE pendant le stage}
	\label{tab:comparatif_visualisations}
	\centering
	\setlength{\arrayrulewidth}{0.6pt}
	\renewcommand{\arraystretch}{1.3}
	\begin{tabular}{|p{3cm}|p{4cm}|p{4cm}|p{3cm}|}
		\hline
		\rowcolor{lightgray}
		\textbf{Outil / Visualisation} & \textbf{Avantages principaux} & \textbf{Défauts / Limites} & \textbf{Usages / Destinataires} \\
		\hline
		Diagramme UML & Rigueur, structuration, conformité aux normes (ISO 25964), repérage des écarts avec les standards, interopérabilité forte & Complexité du formalisme, peu accessible pour les non spécialistes, jugé "obscur" & Métiers techniques, experts, audit documentaire \\
		\hline
		Graphe Gephi & Intuitif, interactif, cartographie des clusters, identification des orphelins, support à la restructuration, accessible à tous & Moins formel, difficulté à intégrer des métadonnées complexes, préparation nécessaire & Groupes de travail, ateliers, médiation \\
		\hline
		Arbres de concepts (draw.io, markmap, etc.) & Très accessibles, vues d'ensemble, communication institutionnelle, rapide à réaliser & Peu de granularité, perte de précision sur les liens, adapté aux grandes catégories & Sensibilisation, CA, ateliers \\
		\hline
		Tableaux de synthèse (Excel, markdown) & Utilisation universelle, tri et export rapide, support aux corrections, facile à enrichir & Peu visuel, ne cartographie pas les relations, perte du contexte relationnel & Agents, gestionnaires, formation \\
		\hline
	\end{tabular}
\end{table}

\subsection{La modélisation comme outil de sensibilisation et de pilotage}


La modélisation conceptuelle n’est pas qu’un acte technique : elle est aussi un instrument de sensibilisation et de pilotage institutionnel. Au \mae, l’organisation d’ateliers a permis d’inviter les agents à réfléchir ensemble à la structuration de l’information, à prendre conscience des enjeux de la donnée et de la transmission documentaire.

Ceux-ci ont utilisé la modélisation pour repenser la hiérarchie des termes ou la nomenclature des vocabulaires. Ce travail a permis de concrétiser aux yeux de tous les métiers l'état des connaissances de l'ensemble du musée, et de réfléchir à la manière de les unifier en un unique système.

\bigskip
\bigskip
\bigskip

Ainsi, la modélisation conceptuelle se révèle être le socle sur lequel peut s’édifier une gouvernance documentaire partagée, une culture commune de l’information, et une capacité à piloter le changement dans la durée.

\section{\label{III-A-2}Les solutions techniques existantes : entre standardisation et adaptation locale}


La maîtrise de la prolifération documentaire ne saurait être réduite à une question de modélisation conceptuelle : elle implique le choix raisonné d’outils susceptibles d’articuler la complexité du réel, tout en répondant à des contraintes de gouvernance, d’interopérabilité et de pérennité. Nous nous sommes attachés dans ce mémoire à l'analyse de deux grandes catégories de l'information qui se retrouve dans un musée : celle qui est utilisée pour décrire ses collections, en la matière des thésaurus et vocabulaires contrôlés aujourd'hui présents au musée, et celle qu'il produit quotidiennement dans le cadre de on activité en tant qu'institution publique avec ses archives numériques. Des outils, techniques comme conceptuels, existent pour aider à la structuration et à la diffusion de cette information : nous avons évoqué, pour les archives numérique, l'utilité du respect de normes nationales et d'outils numériques comme une \gls{ged} ou un \ac{sae}. Pour la gestion des vocabulaires contrôlés, c'est un autre outil de modélisation de la connaissance qui s'est imposé : le thésaurus documentaire.



\subsection{Qu’est-ce qu’un thésaurus documentaire ?}

On ne peut réduire le thésaurus documentaire à une simple liste de mots ou à un instrument technique. La norme ISO 25964, qui fait aujourd’hui autorité, le définit comme un \enquote{vocabulaire contrôlé et structuré dans lequel les concepts sont représentés par des termes, organisés de façon à ce que des relations entre les concepts soient explicitées, et dont les termes préférentiels sont accompagnés par des entrées vers leurs synonymes ou quasi-synonymes}\footcite{ISO25964120112011,maroyeISO25964Distinction2015}. Celui-ci ne se contente donc pas d’indexer, il articule une vision du monde, une manière de penser le réel à travers le langage documentaire. Il impose donc notamment :
\begin{itemize}
	\item de distinguer le concept (une idée), du terme (le mot choisi pour l'exprimer),
	\item de choisir un terme préféré qui servira à décrire le concept,
	\item de mettre les autres termes en synonymes (\textbf{relations d'équivalence})
	\item des \textbf{relations hiérarchiques} avec :
		\subitem des termes génériques,
		\subitem des termes spécifiques ;
	\item des \textbf{relations d'association} entre les termes,
	\item des définitions, notes et alignements externes associés aux termes.
\end{itemize}

Comme montré dans l'image suivante, la structuration d’un thésaurus suppose ainsi une modélisation précise : chaque concept est relié à des termes, chaque terme préféré est accompagné de notes explicatives, chaque branche de la hiérarchie répond à des logiques de genre à espèce, de tout à partie ou  d'instance à classification\footnote{Pour plus de précisions, consulter l'article suivant \cite{perrinBonnesPratiquesPour2020}}. Cette méthode permet notamment de multiplier les accès à l'information, et de faire entrer en adéquation le langage de l'utilisateur avec le langage de l'institution. 

\begin{figure}
	\centering
	\includegraphics[width=0.9\linewidth]{img/SCHEM_thesaurus}
	\caption[Relations sémantiques exprimées par un thésaurus]{L'ensemble des relations sémantiques exprimées par un thésaurus. \textit{Schéma utilisé dans la formation du 4 juin 2025, inspiré de \protect\cite{perrinBonnesPratiquesPour2020}}.}
	\label{fig:schemthesaurus}
\end{figure}


\subsection{Diversité des outils de structuration de l’information}

Simple et efficace, le \gls{thesaurus} s'est imposé au \mae comme un outil compatible avec les logiciels métiers utilisés. Au fil des réflexions sur leur réorganisation, sont cependant apparues diverses difficultés : il est en effet apparu que le \gls{thesaurus} ne permettrait pas d'englober l'ensemble des connaissances requises au musée pour gérer ses collections. En effet, un thésaurus reste un outil lexical, qui permet de contrôler la description des objets : il ne permet pas, par exemple, de symboliser des liens de famille entre des personnes enregistrées comme terme ou d'expliciter les types de relations -- auteur, constructeur, utilisé dans... -- entre deux termes associés.
Il est apparu que pour arriver à cette granularité d'information, un autre outil devrait être adopté : l'ontologie documentaire. Pour citer Thomas Francart, expert dans les systèmes de gestion des connaissances et fondateur de la société SPARNA, \blockquote{l’ontologie cherche à décrire de façon formelle un domaine de connaissance, en identifiant les types d’objets de ce domaine, leurs propriétés et leurs relations\footcite{francartOntologieThesaurusTaxonomie2013}.}

Bien plus riche que le thésaurus, ce format pourrait répondre aux besoins diversifiés du \mae en lui permettant de couvrir davantage de connaissances, et de mieux adresser la diversité de ses utilisateurs avec un système plus modulables que le thésaurus : sa mise en place demanderait cependant un travail de restructuration, de recherche et de création de liens considérable.

C'est le choix qui a été fait par d'autres institutions devant répondre à des exigences similaires : pour ne pas citer le cas bien connu de la \ac{bnf}, citons par exemple celui de la fondation SAPA -- Archives suisses des arts de la scène qui a migré en 2021 toutes les métadonnées de ses collections en une ontologie utilisant le nouveau standard de description \ac{rdf} \ac{rico}\footnote{Sur \ac{rico}, voir \cite{bruleauxRecordsContextsRIC2024}}. Ce format d'ontologie documentaire lui permet ainsi une granularité extrême et une interopérabilité native avec les standards du web sémantique\footcite{coulonDeploiementNormeRecords2024} : l'auteur insiste cependant sur sa nouveauté, l'absence d'une communauté active pour aider à la mise en place du système, et d'outils informatiques appropriés. Il insiste fortement sur \enquote{le bouleversement de nos pratiques professionnelles qui dépasse la nouvelle norme en elle-même} qu'a apporté la migration, montrant que ce type de projet, bien qu'il aboutisse à un système finalement robuste et efficace pour partager et dépasser les \enquote{silos} de données, demande à l'institution qui le met en place un investissement considérable de temps et de ressources, une familiarisation à ces nouvelles techniques, et exige enfin d'être prêt à changer la manière de travailler de l'ensemble de ses utilisateurs.

\subsection{Solutions pour l’interopérabilité et l’ouverture}

L’histoire des normes documentaires, à ce titre, est celle d’une lente montée en exigence : du premier standard Z39.19, consacré à la structuration des thésaurus pour la recherche documentaire, à la norme ISO 25964 publiée en 2011, qui impose la distinction concept/terme et formalise les relations hiérarchiques, associatives et d’équivalence, chaque étape marque un progrès vers la mise en dialogue des systèmes. Comme le souligne Dominique Chichereau, le mouvement de normalisation des thésaurus qui s'est amorcé dès les années 1970 a accompagné le développement de l’informatisation documentaire , et s’est accéléré avec l’essor du web sémantique et la nécessité d'interopérabilité entre des vocabulaires hétérogènes\footcite{chichereauNormesConceptionGestion2007}.

[TODO : schéma d'explication skos rdf ou ajout au glossaire]
C’est dans ce contexte qu'est apparu \ac{skos}, publié par le \ac{w3c} pour \enquote{proposer un système permettant d’exprimer et de gérer des modèles interprétables par les machines dans la perspective du web sémantique\footcite{lenartSKOSLangageRepresentation2007}}, en offrant un modèle fondé sur des triplets \ac{rdf}. Ce standard permet notamment de décrire les concepts, leurs labels multilingues, et les relations hiérarchiques ou associatives entre eux.  \ac{skos} s’impose dès lors comme un langage technique idéal,\enquote{défini comme « simple » par opposition à d’autres modèles, comme OWL (Ontologic Web Language)}, pour mutualiser les vocabulaires sur le web tout en respectant la complexité des relations et la richesse des annotations. 

Bien qu'il existe d'autres solutions de gestion du vocabulaire et que l'avènement de l'intelligence artificielle pousse des institutions à se tourner vers des modèles plus complexes pour implémenter des solutions IA, de la Bibliothèque nationale de Finlande, qui expose ses vocabulaires sur Skosmos\footcite{Skosmos} également utilisé pour le thésaurus de l’UNESCO\footcite{unescoThesaurusLUNESCO1977}, en passant par des institutions patrimoniales françaises telles que le ministère de la Culture avec le thésaurus de Joconde\footcite{ministeredelacultureListeDautoriteDenomination} ou le réseau \ac{frantiq}\footcite{Pactols}, l’usage du format \ac{skos} pour la structuration et la diffusion de thésaurus s’est aujourd’hui imposé aussi bien dans le monde de la recherche, de l’administration publique que dans l’écosystème documentaire des grands musées.
	
	
L’avènement de \ac{skos} et la généralisation du web sémantique n’ont pas seulement permis d’exposer les vocabulaires : ils ont ouvert la voie à une ambition nouvelle, celle de l’alignement des données, qui consacre la possibilité pour chaque institution de faire dialoguer ses savoirs avec ceux de ses pairs. Aligner en effet, c'est relier explicitement des concepts équivalents ou proches, structurer leurs correspondances à l’aide des propriétés (en \ac{skos}, \lstinline|skos:exactMatch|, \lstinline|closeMatch|, \lstinline|broadMatch| ou \lstinline|related|). Ce travail, loin d’être purement technique, engage une réflexion sur la valeur des termes, la portée des synonymes et la fidélité aux usages locaux : il impose de choisir, parmi la profusion des vocabulaires, ceux qui feront pont entre les systèmes et permettront la circulation des connaissances. Au \mae -- ou il n'est encore qu'à l'état de projet -- comme ailleurs, l’alignement avec les thésaurus de Joconde, le SUDOC ou Wikidata ne se réduit pas à un échange de fichiers : il suppose une révision minutieuse des branches, une concertation des acteurs, une vigilance sur les notes historiques et les spécificités des collections. Cette opération apparaît ainsi comme le prolongement naturel de toute politique documentaire soucieuse d’ouverture et de pérennité, pour garantir à l'institution sa capacité à se transmettre, à s’enrichir, et à dialoguer au sein d’un univers patrimonial interconnecté.

\section{\label{III-A-3}L’intégration dans l’écosystème institutionnel : vers une gouvernance collaborative des outils}

L’intégration du logiciel Opentheso au \mae~semblerait être la solution qui réponde le mieux à ses aspirations de gouvernance du vocabulaire et d'interopérabilité avec le web. Cet outil libre recommandé par le ministère de la Culture, s’impose aujourd'hui en France comme le socle d’un vocabulaire  capable de transcender les cloisonnements logiciels et de fédérer les métiers. Sa conformité aux normes ISO 25964 et au standard \gls{skos}, sa capacité à exporter dans des formats interopérables \gls{rdf}, comme \textit{Turtle} ou \textit{JSON-LD}, à visualiser l’arborescence en graphe, à documenter les relations synonymiques ou hiérarchiques, en font un instrument rigoureux qui peut être au service à la fois de la recherche et de la médiation. Opentheso permet de structurer la connaissance, mais aussi de relier les bases métiers – un plugin d'intégration dans \gls{koha} existe déjà, et il serait envisageable de demander une intégration à \textit{Skinsoft} pour Archange qui permettrait au logiciel d'interagir avec l'\ac{api} d'OpenTheso.

La mise en place, en complément, d'un réseau de logiciels dédiés à une gestion rationalisée de l'information au \mae~-- comme des versements réguliers sur le \gls{sae} Vitam pour l’archivage et la mise en place d'une \gls{ged} – serait d'une grande aide au musée pour parvenir à ses ambitions de gouvernance.

\subsection{Accompagnement au changement et formation des agents}

Toute réforme documentaire ne pourra cependant aboutir sans accompagnement au changement : ateliers de sensibilisation, guides d’usage, fresques de la connaissance, tutoriels, autant de dispositifs qui harmoniseraient les pratiques et préparent les agents à la prise en main des nouveaux outils. La formation s'imposera comme la condition de la réussite : elle permettra de comprendre la logique des outils, de s’approprier les méthodes de structuration et d’assurer la pérennité des acquis.

\subsection{Interopérabilité et dialogue intermétiers}

L’interopérabilité enfin, même si la mise en place de solutions dédiées l'aidera grandement, ne se limitera pas à la technique : elle suppose un dialogue continu entre les métiers du musée et une construction de passerelles entre les silos professionnels et plateformes. Les groupes de travail menés au musée ont permis de réfléchir à des solutions communes pour l’archivage numérique et l'unification des thésaurus, ils devront continuer pour les mettre en place et assurer leur bon fonctionnement. 


\bigskip
\bigskip
\bigskip

\lettrine{A}insi se dessine, à travers l’expérience du Musée de l’Air et de l’Espace, une voie exigeante pour la maîtrise de la prolifération documentaire. Il ne s’agit pas seulement de juxtaposer modèles conceptuels, outils techniques et dispositifs de gouvernance : il faut les articuler, les penser ensemble, dans une logique de préservation de la richesse sémantique et d’ouverture à l’interopérabilité. Les solutions expérimentées — modélisation via Gephi, migration vers Opentheso, ateliers de sensibilisation, harmonisation documentaire — illustrent la nécessité d’allier rigueur, adaptation et accompagnement au changement pour garantir la transmission du patrimoine et la vitalité de l’écosystème muséal.
	\chapter[Gérer la prolifération]{\label{III-B} L'unification des thésaurus : un processus collaboratif}



\lettrine{I}ntro

\section{\label{III-B-1}Convaincre de la nécessité du processus}

La structuration du savoir, loin d’être un acte isolé, engage l’institution dans une négociation permanente entre ses acteurs, ses usages et ses impératifs de transmission. La mise en œuvre d’une unification des thésaurus au sein du \mae ne procède donc pas d’une simple injonction technique. Comme nous l'avons vu précédemment, bibliothèque, e-médiathèque et musée disposent chacun de leurs référentiels propres, hérités d’usages, de logiques métiers et de logiciels distincts. Le \mae étant intégré au réseau des musées et bibliothèques du \minarm, toute décision d’unification --- à titre d’exemple, l’intégration d’OpenTheso dans \gls{koha}\footnote{Cette intégration est possible grâce à un plugin développé par la société \textit{Tamil}, voir \cite{PluginTamilOpentheso}} --- peut posséder un impact qui dépasse le seul cadre du \mae pour concerner l'ensemble de ces institutions. Il est donc illusoire de prétendre harmoniser localement une pratique documentaire sans se heurter à la doctrine technique et documentaire imposée à l’échelle ministérielle : ceci rend la négociation indispensable.

Or, nous avons vu que l’interopérabilité s’impose comme une condition de la communicabilité et de la valeur scientifique du fonds documentaire\footcite{hudonISO25964Pour2012a, maroyeISO25964Distinction2015}. Convaincre de la nécessité du processus, c’est inscrire la démarche dans une logique de mutualisation, de visibilité accrue des collections et de conformité aux standards nationaux et internationaux (ISO 25964, \ac{skos}).

Mais ce chantier excède de loin le quotidien des usagers du thésaurus. L’unification s’ajoute aux missions ordinaires des agents, requérant du temps, des compétences techniques et une acculturation documentaire spécifique. Il s’agit donc d’un travail de conviction : chaque professionnel doit être convaincu du bénéfice collectif que représente la démarche. En effet, la réussite du processus dépend de l’implication des agents, des groupes de travail transversaux, du partage de la documentation et de la prise en compte des besoins spécifiques de chaque métier. Insister sur la dimension collaborative est essentiel : seule la concertation régulière entre les différents usagers des thésaurus permet de dépasser les clivages métier.


\section{\label{III-B-2}Créer un thésaurus commun à partir des vocabulaires existants : un travail collaboratif}

L’unification procède d’une méthodologie précise : la démarche adoptée au musée s’est articulée autour d’une formation initiale sur ce qu’est un thésaurus, suivie par des groupes de travail généraux pour explorer l'ensemble des thésaurus et recueillir les usages de chacun pour dégrossir les besoins, puis par des groupes thématiques qui ont examiné chaque type de branche (mots-clés, constructeurs, événements, périodes, matériaux...). Selon le corpus, deux approches se sont dégagées : 
\begin{itemize}
	\item soit une analyse terme à terme lorsque le corpus était limité (définition, synonymes, organisation hiérarchique à partir du niveau haut),
	\item soit une recherche de niveaux hauts communs, puis le rangement progressif des termes spécifiques par la suite.
\end{itemize}
À chaque étape, il a fallu observer les règles de normalisations qui avaient pu être suivies dans le passé et proposer des pistes pour en définir de nouvelles qui puissent être générales au musée.

L’harmonisation des branches et des hiérarchies se fait progressivement, en s’appuyant sur les recommandations institutionnelles (Ministère de la Culture, Joconde\footnote{Voir \cite{ministeredelacultureVocabulairesScientifiquesService2014}} pour le musée). Elle est loin d'être achevée aujourd'hui, mais les premières bases ont été posées pour continuer ce travail d'unification et migrer --- à une date qui n'est pas encore définie et qui sera à négocier avec la tutelle --- vers une plateforme de gestion comme OpenTheso\footcite{OpenThesoa} qui permette de faciliter la maintenance de ce travail.

Le processus de fusion, enfin, s’est construit sur la base des vocabulaires existants : il a fallu croiser les listes de termes, identifier des synonymes et des concepts communs, avant de constituer un référentiel central validé collectivement et partagé sur SharePoint.

\subsection{La réflexion en groupes de travail : avantages et limites}

Le travail collectif a présenté des avantages incontestables : il a notamment permis de mutualiser l'expertise de chacun, de confronter les usages et les logiques métiers et de discuter de solutions pragmatiques qui conviennent à l'ensemble des usagers --- c'est par exemple lors de ces réunions qu'il a été choisi de garder au singulier les mots-clés du thésaurus, de les écrire en minuscules contrairement à ce qui était en usage à la bibliothèque, ou ce qui a permis de relever les difficultés inhérentes à la dénomination et à la hiérarchisation des constructeurs d'avions, dont l'histoire mouvementée rend toute classification difficile.

Mais ce mode de travail comporte aussi de réelles limites : le thésaurus étant un outil de structuration du vocabulaire, il reste nécessaire de se pencher sur l'ensemble des mots qu'il contient pour les organiser et les définir. Or, le volume de données à traiter est considérable et il s'est avéré que ce mode de travail ne serait pas suffisant pour l'unification des thésaurus, et qu'une personne seule, ou un groupe de trois personnes maximum peut être plus efficace --- en s'attelant à un travail de tri au fur et à mesure, par petites séances --- qu'un groupe dont l'objectif sera plutôt de discuter des modifications que de les appliquer. Dans l'un ou l'autre cas cependant, la clé d'une harmonisation efficace reste la communication : toute modification doit être explicitée et confrontée au point de vue des autres métiers avant d'entrer en vigueur.

\inserttable{img/TABL_cr_gt_thematique}

\subsection{Où se situent les principaux besoins ?}

Le constat est paradoxal : ce n’est pas dans les segments les plus techniques ou spécialisés que le besoin d’unification se fait le plus sentir --- ces vocabulaires sont déjà travaillés, structurés, et adossés à la physicalité des objets. Les difficultés majeures surgissent dans les intersections avec les autres institutions patrimoniales, là où le choix entre le respect des normes (ISO 25964, Joconde, SUDOC) et la conservation des spécificités du musée devient le plus épineux.  [TODO : exemple à trouver]

\bigskip
\bigskip
\bigskip

\lettrine{I}ci conclusion
	\chapter[Apports de l'\ac{ia}]{\label{III-C}L'IA : une aide devant la masse des données ?}

\lettrine{L}a croissance exponentielle des données numériques au sein des institutions patrimoniales -- thésaurus, notices, archives, corpus iconographiques -- fait émerger un constat d’impuissance : l’humain semble ne pas suffire pour garantir leur intégrité et leur bonne organisation. Au \mae, la réorganisation et l'harmonisation des vocabulaires, l’identification des doublons, la création de relations d'associations et l'ajout de définitions aux termes mobilisent des volumes de données qui excèdent largement la capacité de traitement manuel des agents. Ces tâches chronophages de comparaison et d'enrichissement des thésaurus et d'autres tâches de la vie quotidienne du musée comme l'indexation et la correction de notices ont amené la question de l'usage de l'intelligence artificielle comme aide pour l'agent. Lors de notre stage, nous avons pu ainsi explorer quelques pistes d'utilisation de l'\ac{ia}, sans toutefois aller plus loin dans l'implémentation de ses solutions -- en effet, ce projet ne rentrant pas dans le périmètre direct des missions du stage, il a été choisi de se concentrer davantage sur la constitution d'une base de vocabulaire aéronautique de référence à partir de \textit{Wikidata}. Avant de détailler les solutions proposées, leurs perspectives et leurs limites, nous nous pencherons brièvement sur les réalités de l'utilisation de l'\ac{ia} en institution patrimoniale.  


%%%%%%%%%%%%%


\section{\label{III-C-1}Automatiser les tâches : promesses et réalités de l’IA en institution patrimoniale}



Face à cette surcharge, elle semble en effet promettre un grand allègement : comme d'autres projets l'ont montré, celui ci n'est efficace que dans la mesure où l'équilibre avec l'expertise humaine est respecté. \enquote{L’\ac{ia} est \textelp{} un outil complémentaire qui, bien encadré, permettrait d’assister les équipes sans remettre en cause l’expertise humaine essentielle à la validation des informations\footcite{bermesRepenserCollectionsPatrimoniales2025}.} Les cas d’utilisation se multiplient dans les institutions patrimoniales, principalement pour l’indexation automatique : on peut citer par exemple le projet d'indexation automatique RAMEAU à la \ac{bnf} \footcite{filabesLindexationRAMEAUAssistee2025}, ou TORNE-H\footcite{bermesRepenserCollectionsPatrimoniales2025} au Musée des arts décoratifs. Les modèles de langage y assistent la reconnaissance des entités, l’extraction et la normalisation des termes. Les méthodes d’apprentissage supervisé permettent désormais de nettoyer les vocabulaires, de regrouper les variantes et d’automatiser la détection d’incohérences (que ce soit avec l'assistance de \textit{LLM} ou de bibliothèques Python comme \textit{NLTK} ou \textit{spaCy}, distances de chaînes comme \textit{Levenshtein}, modèles linguistiques type \textit{CamemBERT} ou \textit{sentence-transformers}) :

\inserttable{img/TABL_outils_ia.tex}

Ces outils efficaces semblent pour l'instant rester cantonnés à des projets pilotes ou à des institutions prestigieuses dont les ressources sont plus élevées que celles du \mae. Le passage de l’expérimentation à la routine semble, pour l’heure, demeurer l’exception dans la majorité des institutions patrimoniales.

\section{\label{III-C-2}Entraîner une IA sur la base d’un vocabulaire spécialisé : défis et solutions}

Cette ambition d’automatiser le traitement des thésaurus se heurte, au MAE, à des contraintes majeures. Ce musée, sous tutelle du \minarm, ne saurait exposer ses données sensibles -- notices techniques, archives, référentiels -- à des modèles externes ou à des services cloud\footnote{A ce sujet, les support informatique du musée se conforme autant que possible aux recommandations de l'\ac{anssi}, et à la suite des méfiances exprimées par l'organisme concernant la sécurité et la confidentialité des données d'entraînement des \ac{llm}, la nouvelle charte informatique de septembre 2025 demandera à tous les agents de s'engager à ne pas les utiliser.}. La confidentialité, renforcée par le \ac{rgpd} et les restrictions propres au secteur, impose une politique de sécurité stricte : tout entraînement doit s’effectuer sur des outils déployés en interne, sur serveurs sécurisés, sans échange hors du périmètre institutionnel.

La méthodologie proposée lors de ce stage pour entraîner une IA sur les vocabulaires spécialisés du musée tente de faire face à ces contraintes. L’entraînement à partir de Wikidata, par extraction du vocabulaire aéronautique via requêtes SPARQL\footnote{Cf. [TODO annexe]}, permet de sélectionner des concepts pertinents, mais génère un bruit considérable : l'exhaustivité reste partielle, et il est difficile de ne pas récupérer de termes non pertinents ce qui nécessite un nettoyage manuel important.

Une autres difficulté rencontrée est le manque de précédent sur des projets de ce type : si les projets d'implémentation de l'\ac{ia} pour des tâches d'indexation -- où celle-ci peut donc mettre chaque oeuvre dans un contexte, de contenu ou de description --, aucun précédent d'utilisation pour traiter un thésaurus semble n'avoir été réalisé, ou du moins diffusé sur le web. La tâche est en effet complexe : dans un thésaurus, chaque mot est indépendant, et il comprend peu de matière de contextualisation dont un modèle de type \textit{bert}\footnote{voir paragraphe précédent} aurait besoin. La solution semble donc être l'utilisation d'un \ac{llm} : cependant, ceux-ci peinent à saisir les nuances du vocabulaire extrêmement spécialisé utilisé par le musée. La solution qui a donc été proposée est l'entraînement d'un \ac{llm} en local\footnote{des tests encore peu concluant ont été réalisés avec le modèle \textit{llama3:2}} sur les données extraites de Wikidata -- dont les termes sont déjà définis et associés entre eux -- et un corpus de données du musée avant de lui fournir les fichiers csv des thésaurus du musée pour lui demander de reconnaître d'éventuels synonymes, associations, et hiérarchisations\footnote{[TODO : ajout annexe prompt et modèle]}. 

\begin{figure}
	\centering
	\includegraphics[width=0.7\linewidth]{img/SCHEM_processus_IA}
	\caption[Proposition de processus d'implémentation d'une IA au \mae]{Proposition de processus d'implémentation d'une IA au \mae après entraînement sur un corpus spécialisé.}
	\label{fig:schemprocessusia}
\end{figure}



\subsection{Limites et perspectives}

À ce stade cependant, demeure la limite de la spécialisation du vocabulaire du musée : même si l'\ac{ia} peut permettre d'alléger la charge de travail des agents travaillant à l'unification du thésaurus, une validation humaine reste nécessaire, pour entraîner l'\ac{ia} et valider ses données. Ceci demande donc de mettre en place une pratique hybride qui articulerait automatisation et savoir-faire documentaire, avant d'aligner le résultat sur des formats \ac{skos}/\ac{rdf}, dans le respect des contraintes propres au musée.

La réflexion sur l’apport de l’IA en institution patrimoniale ne se résume pas à une question d’outillage : elle engage en effet un débat sur la gouvernance de l’information et sur la capacité de l’institution à transmettre son patrimoine sans le dissoudre dans l’automatisme. L'\ac{ia}, ici comme ailleurs, ne reste utile que si elle demeure un auxiliaire du métier et non un substitut. 

\bigskip
\bigskip
\bigskip

%\lettrine{I}ci conclusion

%%%%%%%%%%%%%%%%%%%%%%%%%%% CONCLUSION %%%%%%%%%%%%%%%%%%%%%%%%%%%

	Ici, je pourrai mettre la conclusion de cette partie
	
%%%%%%%%%%%%%%%%%%%%%%%%%%%%%%%%%%%%%%%%%%%%%%%%%%%%%%%%%%%%%%%%%%
%%%%%%%%%%%%%%%%%%%%%%%%%%% CONCLUSION %%%%%%%%%%%%%%%%%%%%%%%%%%%
%%%%%%%%%%%%%%%%%%%%%%%%%%%%%%%%%%%%%%%%%%%%%%%%%%%%%%%%%%%%%%%%%%
	
\chapter*{Conclusion}
\addcontentsline{toc}{chapter}{Conclusion}
\newpage{\pagestyle{empty}\cleardoublepage}
	






%%%%%%%%%%%%%%%%%%%%%%%%%%%%%%%%%%%%%%%%%%%%%%%%%%%%%%%%%%%%%%%%%%
%%%%%%%%%%%%%%%%%%%%%%%%%% BACKMATTER %%%%%%%%%%%%%%%%%%%%%%%%%%%%
%%%%%%%%%%%%%%%%%%%%%%%%%%%%%%%%%%%%%%%%%%%%%%%%%%%%%%%%%%%%%%%%%%

%%%%%%%%%%%%%%%%%%%%%%%%% ANNEXES %%%%%%%%%%%%%%%%%%%%%%%%%%%%%%%%
%%%%%%%%%%%%%%%%%%%%%%%%%%%%%%%%%%%%%%%%%%%%%%%%%%%%%%%%%%%%%%%%%%

\appendix %Des appendices: tables figures, etc

%%%%%%%%%%%%%%%%%%%%%%%%%%% ANNEXE 1 %%%%%%%%%%%%%%%%%%%%%%%%%%%%%
\part*{Annexes}
\addcontentsline{toc}{part}{Annexes}

\chapter[Chronologie du MAE]{Chronologie du musée de l'air et de l'espace}


\LTXtable{\textwidth}{./parties/backmatter/annexes/A1_frise.tex}

\newpage{\pagestyle{empty}\cleardoublepage}



\backmatter % glossaire, index, table des figures, table des matières.. (la bibliographie a déjà été appelée)

%%%%%%%%%%%%%%%%%%%%%%%%%%%%%% INDEX %%%%%%%%%%%%%%%%%%%%%%%%%%%%%
%%%%%%%%%%%%%%%%%%%%%%%%%%%%%%%%%%%%%%%%%%%%%%%%%%%%%%%%%%%%%%%%%%

%\printindex

%%%%%%%%%%%%%%%%%%%%%%%%%%% GLOSSAIRE %%%%%%%%%%%%%%%%%%%%%%%%%%%%
%%%%%%%%%%%%%%%%%%%%%%%%%%%%%%%%%%%%%%%%%%%%%%%%%%%%%%%%%%%%%%%%%%

\printglossaries
\addcontentsline{toc}{part}{Glossaire}

%%%%%%%%%%%%%%%%%%%%%%%%%% TABLES DES MATIÈRES %%%%%%%%%%%%%%%%%%%
%%%%%%%%%%%%%%%%%%%%%%%%%%%%%%%%%%%%%%%%%%%%%%%%%%%%%%%%%%%%%%%%%%
\addcontentsline{toc}{part}{Tables}
%%%%%%%%%%%%%%%%%%%%% TABLE DES TABLEAUX %%%%%%%%%%%%%%%%%%%%%%%%%
%\listoftables

%%%%%%%%%%%%%%%%%%%%%%%%% TABLE DES FIGURES %%%%%%%%%%%%%%%%%%%%%%
%\listoffigures

%%%%%%%%%%%%%%%%%%%%%%%%% TABLE DES MATIÈRES %%%%%%%%%%%%%%%%%%%%%
\tableofcontents


%%%%%%%%%%%%%%%%%%%%%%%%%%%%%%%%%%%%%%%%%%%%%%%%%%%%%%%%%%%%%%%%%%%
%%%%%%%%%%%%%%%%%%%%%%%%%% FIN DU DOCUMENT %%%%%%%%%%%%%%%%%%%%%%%%
%%%%%%%%%%%%%%%%%%%%%%%%%%%%%%%%%%%%%%%%%%%%%%%%%%%%%%%%%%%%%%%%%%%
\end{document}
