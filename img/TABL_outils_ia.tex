\begin{longtable}{D E E}
	\caption{Récapitulatif des outils d'\ac{ia} envisagés pour les \gls{thesaurus} du \mae}
	\label{tab:recapIA}\\
	\hline\hline
	\textbf{Outil / Modèle} & \textbf{Fonction principale} & \textbf{Usages en contexte documentaire / thésaurus} \\
	\hline
	\endfirsthead
	
	\hline\hline
	\textbf{Outil / Modèle} & \textbf{Fonction principale} & \textbf{Usages en contexte documentaire / thésaurus} \\
	\hline
	\endhead
	
	\hline
	\endfoot
	
	\hline\hline
	\endlastfoot
	
	\ac{llm} & Modèles de traitement du langage naturel entraînés sur de vastes corpus & Analyse, génération et synthèse de texte ; proposition d’associations sémantiques ou corrections automatiques dans les vocabulaires ; efficacité dépendante du corpus d’entraînement et de l’adaptation au domaine patrimonial. \\
	\rowcolor{lightgray}
	NLTK (Natural Language Toolkit) & Bibliothèque Python dédiée au traitement linguistique & Segmentation, nettoyage, analyse morphologique, reconnaissance d’entités nommées ; automatisation de la normalisation des termes, détection des variantes orthographiques, structuration des listes d’autorité. \\
	spaCy & Bibliothèque Python spécialisée dans le traitement du langage naturel & Reconnaissance d’entités, lemmatisation, classification des termes ; identification d’entités dans les notices, regroupement de variantes lexicales, extraction de relations entre les termes. \\
	\rowcolor{lightgray}
	Distance de Levenshtein & Algorithme de calcul de distance entre chaînes de caractères & Repérage des doublons ou variantes orthographiques au sein d’un thésaurus ; facilite l’harmonisation des vocabulaires. \\
	CamemBERT, sentence-transformers & Modèles de langage pré-entraînés pour le français & Encodage de termes dans un espace vectoriel, calcul de proximité sémantique ; détection et regroupement de synonymes ou quasi-synonymes, automatisation des suggestions de corrections sémantiques. \\
\end{longtable}