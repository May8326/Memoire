\begin{table}[htbp]
	\caption{Comparatif des visualisations de données utilisées au MAE pendant le stage}
	\label{tab:comparatif_visualisations}
	\centering
	\setlength{\arrayrulewidth}{0.6pt}
	\renewcommand{\arraystretch}{1.3}
	\begin{tabular}{|p{3cm}|p{4cm}|p{4cm}|p{3cm}|}
		\hline
		\rowcolor{lightgray}
		\textbf{Outil / Visualisation} & \textbf{Avantages principaux} & \textbf{Défauts / Limites} & \textbf{Usages / Destinataires} \\
		\hline
		Diagramme UML & Rigueur, structuration, conformité aux normes (ISO 25964), repérage des écarts avec les standards, interopérabilité forte & Complexité du formalisme, peu accessible pour les non spécialistes, jugé "obscur" & Métiers techniques, experts, audit documentaire \\
		\hline
		Graphe Gephi & Intuitif, interactif, cartographie des clusters, identification des orphelins, support à la restructuration, accessible à tous & Moins formel, difficulté à intégrer des métadonnées complexes, préparation nécessaire & Groupes de travail, ateliers, médiation \\
		\hline
		Arbres de concepts (draw.io, markmap, etc.) & Très accessibles, vues d'ensemble, communication institutionnelle, rapide à réaliser & Peu de granularité, perte de précision sur les liens, adapté aux grandes catégories & Sensibilisation, CA, ateliers \\
		\hline
		Tableaux de synthèse (Excel, markdown) & Utilisation universelle, tri et export rapide, support aux corrections, facile à enrichir & Peu visuel, ne cartographie pas les relations, perte du contexte relationnel & Agents, gestionnaires, formation \\
		\hline
	\end{tabular}
\end{table}