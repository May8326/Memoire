\begin{table}[htbp]
	\centering
	\begin{tabularx}{\textwidth}{|>{\raggedright\arraybackslash}X|c|c|c|c|}
		\hline
		\textbf{Thésaurus} & \textbf{Termes} & \textbf{Profondeur} & \textbf{Relations associées} & \textbf{Synonymes} \\
		\hline
		\textbf{Alexandrie / Koha} & \textbf{21\,523} & 17 & 3\,820 & 13\,582 \\
		\hline
		\multicolumn{5}{|l|}{\textbf{Micromusée : 16\,520 termes}} \\
		\hline
		Dégradations & 147 & 2 & 0 & 13 \\
		\hline
		Événements & 534 + 21 types & 1 & 244 & 0 \\
		\hline
		Matériaux et techniques & 598 & 4 & 0 & 0 \\
		\hline
		Mots clés & 2\,024 & 6 & 8 & 79 \\
		\hline
		Constructeurs & 2\,358 & 6 & 157 & 334 \\
		\hline
		Personnes & 5\,260 + 3 types & 1 & 0 & 0 \\
		\hline
		Collectivités & 5\,571 + 4 types & 1 & 0 & 0 \\
		\hline
		Domaines & \multicolumn{4}{|l|}{18 domaines, 41 sous-domaines, 74 dénominations} \\
		\hline
		\multicolumn{5}{|l|}{\textbf{E-médiathèque : 17\,917 termes}} \\
		\hline
		Événements & 422 + 21 types & 1 & 0 & 0 \\
		\hline
		Personnes & 11\,084 + 2 types & 1 & 0 & 0 \\
		\hline
		Mots clés & 6\,383 + 5 types & 9 & 0 & 1\,229 \\
		\hline
		Valeurs & 1\,797 + 10 types & 1 & 0 & 0 \\
		\hline
	\end{tabularx}
	\caption[Synthèse des caractéristiques des thésaurus et listes d'autorités utilisés au \mae]{Trois ensembles de vocabulaires contrôlés coexistent au \mae : on remarquera la présence de thésaurus complexes (profondeur de 2 ou plus, associations, synonymes), et de simples listes d'autorités. Leur volumétrie importante rend tout traitement manuel au terme à terme long et fastidieux.\footnote{Pour plus de précision sur les concepts de gestion de thésaurus utilisés, voir \refinterne{III-A-2.1}}}
	\label{tab:thesaurus_synthese}
\end{table}