\begin{longtable}{D E E}
	\caption{Récapitulatif des groupes de travail réalisés en juin et juillet 2025 au \mae}
	\label{tab:recapGT} \\
	\hline\hline
	\textbf{Groupe} & \textbf{Objectifs / Méthodologie} & \textbf{Points clés / À poursuivre} \\
	\hline
	\endfirsthead
	
	\hline\hline
	\textbf{Groupe} & \textbf{Objectifs / Méthodologie} & \textbf{Points clés / À poursuivre} \\
	\hline
	\endhead
	
	\hline
	\endfoot
	
	\hline\hline
	\endlastfoot
	
	\multicolumn{3}{|p{\textwidth}|}{
		\textbf{Méthodologie générale : } \newline
		Approche progressive par termes génériques. \newline
		Comparaison des thésaurus existants et validation collective. \newline
		Consultation de référentiels externes. \newline
		Création d’une base cohérente et adaptée aux besoins des services.
	} \\
	\hline
	\multicolumn{3}{|p{\textwidth}|}{
		\textbf{Axes de travail : } \newline
		Hiérarchisation par niveaux généraux. \newline
		Harmonisation des règles d’écriture. \newline
		Clarification des définitions et normalisation des usages. \newline
		Discussion entre les métiers.
	} \\
	\hline
	\textbf{Événements} & 
	Structuration par termes génériques, hiérarchie logique, définitions claires. \newline
	Comparaison des thésaurus existants. \newline
	Consultation de dictionnaires. &
	Arborescence proposée avec niveaux hiérarchiques. \newline
	Travail à poursuivre sur définitions, synonymes, normalisation typographique. \\
	\hline
	\rowcolor{lightgray}
	\textbf{Périodes et Valeurs} & 
	Travail en deux temps. \newline
	Révision des localisations, formats, supports. \newline
	Hiérarchisation en 3 branches : numérique, événementielle, dynastique. &
	Harmonisation des couleurs, acronymes, typographie. \newline
	Clarification entre mots-clés et valeurs. \newline
	Normalisation des dates et appellations. \\
	\hline
	\textbf{Mots-clés et Noms communs} &
	Structuration par généralités. \newline
	Identification de termes maîtres communs. &
	Consolidation des termes maîtres (ex. Aéronautique, Arts, Environnement). \newline
	Définir les mots-clés subjectifs. \newline
	Règles de rédaction : singulier, majuscule initiale, accents. \\
	\hline
	\rowcolor{lightgray}
	\textbf{Constructeurs} &
	Méthodologie pour nommage unifié. \newline
	Validation des nouveaux termes. &
	Hiérarchie respectée, sens historique. \newline
	Règles : acronymes sans points, pas de constructeur entre parenthèses. \newline
	Périmètre : avions, matériel aéronautique, spatial. \\
\end{longtable}